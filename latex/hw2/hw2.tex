\documentclass[11pt]{article}

    \usepackage{fontspec}
    \usepackage[slantfont, boldfont]{xeCJK}
    \setmainfont{Times New Roman} 
    \setCJKmainfont{標楷體}

    \usepackage[breakable]{tcolorbox}
    \usepackage{parskip} % Stop auto-indenting (to mimic markdown behaviour)
    

    % Basic figure setup, for now with no caption control since it's done
    % automatically by Pandoc (which extracts ![](path) syntax from Markdown).
    \usepackage{graphicx}
    % Keep aspect ratio if custom image width or height is specified
    \setkeys{Gin}{keepaspectratio}
    % Maintain compatibility with old templates. Remove in nbconvert 6.0
    \let\Oldincludegraphics\includegraphics
    % Ensure that by default, figures have no caption (until we provide a
    % proper Figure object with a Caption API and a way to capture that
    % in the conversion process - todo).
    \usepackage{caption}
    \DeclareCaptionFormat{nocaption}{}
    \captionsetup{format=nocaption,aboveskip=0pt,belowskip=0pt}

    \usepackage{float}
    \floatplacement{figure}{H} % forces figures to be placed at the correct location
    \usepackage{xcolor} % Allow colors to be defined
    \usepackage{enumerate} % Needed for markdown enumerations to work
    \usepackage{geometry} % Used to adjust the document margins
    \usepackage{amsmath} % Equations
    \usepackage{amssymb} % Equations
    \usepackage{textcomp} % defines textquotesingle
    % Hack from http://tex.stackexchange.com/a/47451/13684:
    \AtBeginDocument{%
        \def\PYZsq{\textquotesingle}% Upright quotes in Pygmentized code
    }
    \usepackage{upquote} % Upright quotes for verbatim code
    \usepackage{eurosym} % defines \euro

    \usepackage{iftex}
    \ifPDFTeX
        \usepackage[T1]{fontenc}
        \IfFileExists{alphabeta.sty}{
              \usepackage{alphabeta}
          }{
              \usepackage[mathletters]{ucs}
              \usepackage[utf8x]{inputenc}
          }
    \else
        \usepackage{fontspec}
        \usepackage{unicode-math}
    \fi

    \usepackage{fancyvrb} % verbatim replacement that allows latex
    \usepackage{grffile} % extends the file name processing of package graphics
                         % to support a larger range
    \makeatletter % fix for old versions of grffile with XeLaTeX
    \@ifpackagelater{grffile}{2019/11/01}
    {
      % Do nothing on new versions
    }
    {
      \def\Gread@@xetex#1{%
        \IfFileExists{"\Gin@base".bb}%
        {\Gread@eps{\Gin@base.bb}}%
        {\Gread@@xetex@aux#1}%
      }
    }
    \makeatother
    \usepackage[Export]{adjustbox} % Used to constrain images to a maximum size
    \adjustboxset{max size={0.9\linewidth}{0.9\paperheight}}

    % The hyperref package gives us a pdf with properly built
    % internal navigation ('pdf bookmarks' for the table of contents,
    % internal cross-reference links, web links for URLs, etc.)
    \usepackage{hyperref}
    % The default LaTeX title has an obnoxious amount of whitespace. By default,
    % titling removes some of it. It also provides customization options.
    \usepackage{titling}
    \usepackage{longtable} % longtable support required by pandoc >1.10
    \usepackage{booktabs}  % table support for pandoc > 1.12.2
    \usepackage{array}     % table support for pandoc >= 2.11.3
    \usepackage{calc}      % table minipage width calculation for pandoc >= 2.11.1
    \usepackage[inline]{enumitem} % IRkernel/repr support (it uses the enumerate* environment)
    \usepackage[normalem]{ulem} % ulem is needed to support strikethroughs (\sout)
                                % normalem makes italics be italics, not underlines
    \usepackage{soul}      % strikethrough (\st) support for pandoc >= 3.0.0
    \usepackage{mathrsfs}
    

    
    % Colors for the hyperref package
    \definecolor{urlcolor}{rgb}{0,.145,.698}
    \definecolor{linkcolor}{rgb}{.71,0.21,0.01}
    \definecolor{citecolor}{rgb}{.12,.54,.11}

    % ANSI colors
    \definecolor{ansi-black}{HTML}{3E424D}
    \definecolor{ansi-black-intense}{HTML}{282C36}
    \definecolor{ansi-red}{HTML}{E75C58}
    \definecolor{ansi-red-intense}{HTML}{B22B31}
    \definecolor{ansi-green}{HTML}{00A250}
    \definecolor{ansi-green-intense}{HTML}{007427}
    \definecolor{ansi-yellow}{HTML}{DDB62B}
    \definecolor{ansi-yellow-intense}{HTML}{B27D12}
    \definecolor{ansi-blue}{HTML}{208FFB}
    \definecolor{ansi-blue-intense}{HTML}{0065CA}
    \definecolor{ansi-magenta}{HTML}{D160C4}
    \definecolor{ansi-magenta-intense}{HTML}{A03196}
    \definecolor{ansi-cyan}{HTML}{60C6C8}
    \definecolor{ansi-cyan-intense}{HTML}{258F8F}
    \definecolor{ansi-white}{HTML}{C5C1B4}
    \definecolor{ansi-white-intense}{HTML}{A1A6B2}
    \definecolor{ansi-default-inverse-fg}{HTML}{FFFFFF}
    \definecolor{ansi-default-inverse-bg}{HTML}{000000}

    % common color for the border for error outputs.
    \definecolor{outerrorbackground}{HTML}{FFDFDF}

    % commands and environments needed by pandoc snippets
    % extracted from the output of `pandoc -s`
    \providecommand{\tightlist}{%
      \setlength{\itemsep}{0pt}\setlength{\parskip}{0pt}}
    \DefineVerbatimEnvironment{Highlighting}{Verbatim}{commandchars=\\\{\}}
    % Add ',fontsize=\small' for more characters per line
    \newenvironment{Shaded}{}{}
    \newcommand{\KeywordTok}[1]{\textcolor[rgb]{0.00,0.44,0.13}{\textbf{{#1}}}}
    \newcommand{\DataTypeTok}[1]{\textcolor[rgb]{0.56,0.13,0.00}{{#1}}}
    \newcommand{\DecValTok}[1]{\textcolor[rgb]{0.25,0.63,0.44}{{#1}}}
    \newcommand{\BaseNTok}[1]{\textcolor[rgb]{0.25,0.63,0.44}{{#1}}}
    \newcommand{\FloatTok}[1]{\textcolor[rgb]{0.25,0.63,0.44}{{#1}}}
    \newcommand{\CharTok}[1]{\textcolor[rgb]{0.25,0.44,0.63}{{#1}}}
    \newcommand{\StringTok}[1]{\textcolor[rgb]{0.25,0.44,0.63}{{#1}}}
    \newcommand{\CommentTok}[1]{\textcolor[rgb]{0.38,0.63,0.69}{\textit{{#1}}}}
    \newcommand{\OtherTok}[1]{\textcolor[rgb]{0.00,0.44,0.13}{{#1}}}
    \newcommand{\AlertTok}[1]{\textcolor[rgb]{1.00,0.00,0.00}{\textbf{{#1}}}}
    \newcommand{\FunctionTok}[1]{\textcolor[rgb]{0.02,0.16,0.49}{{#1}}}
    \newcommand{\RegionMarkerTok}[1]{{#1}}
    \newcommand{\ErrorTok}[1]{\textcolor[rgb]{1.00,0.00,0.00}{\textbf{{#1}}}}
    \newcommand{\NormalTok}[1]{{#1}}

    % Additional commands for more recent versions of Pandoc
    \newcommand{\ConstantTok}[1]{\textcolor[rgb]{0.53,0.00,0.00}{{#1}}}
    \newcommand{\SpecialCharTok}[1]{\textcolor[rgb]{0.25,0.44,0.63}{{#1}}}
    \newcommand{\VerbatimStringTok}[1]{\textcolor[rgb]{0.25,0.44,0.63}{{#1}}}
    \newcommand{\SpecialStringTok}[1]{\textcolor[rgb]{0.73,0.40,0.53}{{#1}}}
    \newcommand{\ImportTok}[1]{{#1}}
    \newcommand{\DocumentationTok}[1]{\textcolor[rgb]{0.73,0.13,0.13}{\textit{{#1}}}}
    \newcommand{\AnnotationTok}[1]{\textcolor[rgb]{0.38,0.63,0.69}{\textbf{\textit{{#1}}}}}
    \newcommand{\CommentVarTok}[1]{\textcolor[rgb]{0.38,0.63,0.69}{\textbf{\textit{{#1}}}}}
    \newcommand{\VariableTok}[1]{\textcolor[rgb]{0.10,0.09,0.49}{{#1}}}
    \newcommand{\ControlFlowTok}[1]{\textcolor[rgb]{0.00,0.44,0.13}{\textbf{{#1}}}}
    \newcommand{\OperatorTok}[1]{\textcolor[rgb]{0.40,0.40,0.40}{{#1}}}
    \newcommand{\BuiltInTok}[1]{{#1}}
    \newcommand{\ExtensionTok}[1]{{#1}}
    \newcommand{\PreprocessorTok}[1]{\textcolor[rgb]{0.74,0.48,0.00}{{#1}}}
    \newcommand{\AttributeTok}[1]{\textcolor[rgb]{0.49,0.56,0.16}{{#1}}}
    \newcommand{\InformationTok}[1]{\textcolor[rgb]{0.38,0.63,0.69}{\textbf{\textit{{#1}}}}}
    \newcommand{\WarningTok}[1]{\textcolor[rgb]{0.38,0.63,0.69}{\textbf{\textit{{#1}}}}}
    \makeatletter
    \newsavebox\pandoc@box
    \newcommand*\pandocbounded[1]{%
      \sbox\pandoc@box{#1}%
      % scaling factors for width and height
      \Gscale@div\@tempa\textheight{\dimexpr\ht\pandoc@box+\dp\pandoc@box\relax}%
      \Gscale@div\@tempb\linewidth{\wd\pandoc@box}%
      % select the smaller of both
      \ifdim\@tempb\p@<\@tempa\p@
        \let\@tempa\@tempb
      \fi
      % scaling accordingly (\@tempa < 1)
      \ifdim\@tempa\p@<\p@
        \scalebox{\@tempa}{\usebox\pandoc@box}%
      % scaling not needed, use as it is
      \else
        \usebox{\pandoc@box}%
      \fi
    }
    \makeatother

    % Define a nice break command that doesn't care if a line doesn't already
    % exist.
    \def\br{\hspace*{\fill} \\* }
    % Math Jax compatibility definitions
    \def\gt{>}
    \def\lt{<}
    \let\Oldtex\TeX
    \let\Oldlatex\LaTeX
    \renewcommand{\TeX}{\textrm{\Oldtex}}
    \renewcommand{\LaTeX}{\textrm{\Oldlatex}}
    % Document parameters
    % Document title
    \title{AI Introduction Homework 2}
    \author{4112064214 侯竣奇}
    
    
    
    
    
    
    
% Pygments definitions
\makeatletter
\def\PY@reset{\let\PY@it=\relax \let\PY@bf=\relax%
    \let\PY@ul=\relax \let\PY@tc=\relax%
    \let\PY@bc=\relax \let\PY@ff=\relax}
\def\PY@tok#1{\csname PY@tok@#1\endcsname}
\def\PY@toks#1+{\ifx\relax#1\empty\else%
    \PY@tok{#1}\expandafter\PY@toks\fi}
\def\PY@do#1{\PY@bc{\PY@tc{\PY@ul{%
    \PY@it{\PY@bf{\PY@ff{#1}}}}}}}
\def\PY#1#2{\PY@reset\PY@toks#1+\relax+\PY@do{#2}}

\@namedef{PY@tok@w}{\def\PY@tc##1{\textcolor[rgb]{0.73,0.73,0.73}{##1}}}
\@namedef{PY@tok@c}{\let\PY@it=\textit\def\PY@tc##1{\textcolor[rgb]{0.24,0.48,0.48}{##1}}}
\@namedef{PY@tok@cp}{\def\PY@tc##1{\textcolor[rgb]{0.61,0.40,0.00}{##1}}}
\@namedef{PY@tok@k}{\let\PY@bf=\textbf\def\PY@tc##1{\textcolor[rgb]{0.00,0.50,0.00}{##1}}}
\@namedef{PY@tok@kp}{\def\PY@tc##1{\textcolor[rgb]{0.00,0.50,0.00}{##1}}}
\@namedef{PY@tok@kt}{\def\PY@tc##1{\textcolor[rgb]{0.69,0.00,0.25}{##1}}}
\@namedef{PY@tok@o}{\def\PY@tc##1{\textcolor[rgb]{0.40,0.40,0.40}{##1}}}
\@namedef{PY@tok@ow}{\let\PY@bf=\textbf\def\PY@tc##1{\textcolor[rgb]{0.67,0.13,1.00}{##1}}}
\@namedef{PY@tok@nb}{\def\PY@tc##1{\textcolor[rgb]{0.00,0.50,0.00}{##1}}}
\@namedef{PY@tok@nf}{\def\PY@tc##1{\textcolor[rgb]{0.00,0.00,1.00}{##1}}}
\@namedef{PY@tok@nc}{\let\PY@bf=\textbf\def\PY@tc##1{\textcolor[rgb]{0.00,0.00,1.00}{##1}}}
\@namedef{PY@tok@nn}{\let\PY@bf=\textbf\def\PY@tc##1{\textcolor[rgb]{0.00,0.00,1.00}{##1}}}
\@namedef{PY@tok@ne}{\let\PY@bf=\textbf\def\PY@tc##1{\textcolor[rgb]{0.80,0.25,0.22}{##1}}}
\@namedef{PY@tok@nv}{\def\PY@tc##1{\textcolor[rgb]{0.10,0.09,0.49}{##1}}}
\@namedef{PY@tok@no}{\def\PY@tc##1{\textcolor[rgb]{0.53,0.00,0.00}{##1}}}
\@namedef{PY@tok@nl}{\def\PY@tc##1{\textcolor[rgb]{0.46,0.46,0.00}{##1}}}
\@namedef{PY@tok@ni}{\let\PY@bf=\textbf\def\PY@tc##1{\textcolor[rgb]{0.44,0.44,0.44}{##1}}}
\@namedef{PY@tok@na}{\def\PY@tc##1{\textcolor[rgb]{0.41,0.47,0.13}{##1}}}
\@namedef{PY@tok@nt}{\let\PY@bf=\textbf\def\PY@tc##1{\textcolor[rgb]{0.00,0.50,0.00}{##1}}}
\@namedef{PY@tok@nd}{\def\PY@tc##1{\textcolor[rgb]{0.67,0.13,1.00}{##1}}}
\@namedef{PY@tok@s}{\def\PY@tc##1{\textcolor[rgb]{0.73,0.13,0.13}{##1}}}
\@namedef{PY@tok@sd}{\let\PY@it=\textit\def\PY@tc##1{\textcolor[rgb]{0.73,0.13,0.13}{##1}}}
\@namedef{PY@tok@si}{\let\PY@bf=\textbf\def\PY@tc##1{\textcolor[rgb]{0.64,0.35,0.47}{##1}}}
\@namedef{PY@tok@se}{\let\PY@bf=\textbf\def\PY@tc##1{\textcolor[rgb]{0.67,0.36,0.12}{##1}}}
\@namedef{PY@tok@sr}{\def\PY@tc##1{\textcolor[rgb]{0.64,0.35,0.47}{##1}}}
\@namedef{PY@tok@ss}{\def\PY@tc##1{\textcolor[rgb]{0.10,0.09,0.49}{##1}}}
\@namedef{PY@tok@sx}{\def\PY@tc##1{\textcolor[rgb]{0.00,0.50,0.00}{##1}}}
\@namedef{PY@tok@m}{\def\PY@tc##1{\textcolor[rgb]{0.40,0.40,0.40}{##1}}}
\@namedef{PY@tok@gh}{\let\PY@bf=\textbf\def\PY@tc##1{\textcolor[rgb]{0.00,0.00,0.50}{##1}}}
\@namedef{PY@tok@gu}{\let\PY@bf=\textbf\def\PY@tc##1{\textcolor[rgb]{0.50,0.00,0.50}{##1}}}
\@namedef{PY@tok@gd}{\def\PY@tc##1{\textcolor[rgb]{0.63,0.00,0.00}{##1}}}
\@namedef{PY@tok@gi}{\def\PY@tc##1{\textcolor[rgb]{0.00,0.52,0.00}{##1}}}
\@namedef{PY@tok@gr}{\def\PY@tc##1{\textcolor[rgb]{0.89,0.00,0.00}{##1}}}
\@namedef{PY@tok@ge}{\let\PY@it=\textit}
\@namedef{PY@tok@gs}{\let\PY@bf=\textbf}
\@namedef{PY@tok@ges}{\let\PY@bf=\textbf\let\PY@it=\textit}
\@namedef{PY@tok@gp}{\let\PY@bf=\textbf\def\PY@tc##1{\textcolor[rgb]{0.00,0.00,0.50}{##1}}}
\@namedef{PY@tok@go}{\def\PY@tc##1{\textcolor[rgb]{0.44,0.44,0.44}{##1}}}
\@namedef{PY@tok@gt}{\def\PY@tc##1{\textcolor[rgb]{0.00,0.27,0.87}{##1}}}
\@namedef{PY@tok@err}{\def\PY@bc##1{{\setlength{\fboxsep}{\string -\fboxrule}\fcolorbox[rgb]{1.00,0.00,0.00}{1,1,1}{\strut ##1}}}}
\@namedef{PY@tok@kc}{\let\PY@bf=\textbf\def\PY@tc##1{\textcolor[rgb]{0.00,0.50,0.00}{##1}}}
\@namedef{PY@tok@kd}{\let\PY@bf=\textbf\def\PY@tc##1{\textcolor[rgb]{0.00,0.50,0.00}{##1}}}
\@namedef{PY@tok@kn}{\let\PY@bf=\textbf\def\PY@tc##1{\textcolor[rgb]{0.00,0.50,0.00}{##1}}}
\@namedef{PY@tok@kr}{\let\PY@bf=\textbf\def\PY@tc##1{\textcolor[rgb]{0.00,0.50,0.00}{##1}}}
\@namedef{PY@tok@bp}{\def\PY@tc##1{\textcolor[rgb]{0.00,0.50,0.00}{##1}}}
\@namedef{PY@tok@fm}{\def\PY@tc##1{\textcolor[rgb]{0.00,0.00,1.00}{##1}}}
\@namedef{PY@tok@vc}{\def\PY@tc##1{\textcolor[rgb]{0.10,0.09,0.49}{##1}}}
\@namedef{PY@tok@vg}{\def\PY@tc##1{\textcolor[rgb]{0.10,0.09,0.49}{##1}}}
\@namedef{PY@tok@vi}{\def\PY@tc##1{\textcolor[rgb]{0.10,0.09,0.49}{##1}}}
\@namedef{PY@tok@vm}{\def\PY@tc##1{\textcolor[rgb]{0.10,0.09,0.49}{##1}}}
\@namedef{PY@tok@sa}{\def\PY@tc##1{\textcolor[rgb]{0.73,0.13,0.13}{##1}}}
\@namedef{PY@tok@sb}{\def\PY@tc##1{\textcolor[rgb]{0.73,0.13,0.13}{##1}}}
\@namedef{PY@tok@sc}{\def\PY@tc##1{\textcolor[rgb]{0.73,0.13,0.13}{##1}}}
\@namedef{PY@tok@dl}{\def\PY@tc##1{\textcolor[rgb]{0.73,0.13,0.13}{##1}}}
\@namedef{PY@tok@s2}{\def\PY@tc##1{\textcolor[rgb]{0.73,0.13,0.13}{##1}}}
\@namedef{PY@tok@sh}{\def\PY@tc##1{\textcolor[rgb]{0.73,0.13,0.13}{##1}}}
\@namedef{PY@tok@s1}{\def\PY@tc##1{\textcolor[rgb]{0.73,0.13,0.13}{##1}}}
\@namedef{PY@tok@mb}{\def\PY@tc##1{\textcolor[rgb]{0.40,0.40,0.40}{##1}}}
\@namedef{PY@tok@mf}{\def\PY@tc##1{\textcolor[rgb]{0.40,0.40,0.40}{##1}}}
\@namedef{PY@tok@mh}{\def\PY@tc##1{\textcolor[rgb]{0.40,0.40,0.40}{##1}}}
\@namedef{PY@tok@mi}{\def\PY@tc##1{\textcolor[rgb]{0.40,0.40,0.40}{##1}}}
\@namedef{PY@tok@il}{\def\PY@tc##1{\textcolor[rgb]{0.40,0.40,0.40}{##1}}}
\@namedef{PY@tok@mo}{\def\PY@tc##1{\textcolor[rgb]{0.40,0.40,0.40}{##1}}}
\@namedef{PY@tok@ch}{\let\PY@it=\textit\def\PY@tc##1{\textcolor[rgb]{0.24,0.48,0.48}{##1}}}
\@namedef{PY@tok@cm}{\let\PY@it=\textit\def\PY@tc##1{\textcolor[rgb]{0.24,0.48,0.48}{##1}}}
\@namedef{PY@tok@cpf}{\let\PY@it=\textit\def\PY@tc##1{\textcolor[rgb]{0.24,0.48,0.48}{##1}}}
\@namedef{PY@tok@c1}{\let\PY@it=\textit\def\PY@tc##1{\textcolor[rgb]{0.24,0.48,0.48}{##1}}}
\@namedef{PY@tok@cs}{\let\PY@it=\textit\def\PY@tc##1{\textcolor[rgb]{0.24,0.48,0.48}{##1}}}

\def\PYZbs{\char`\\}
\def\PYZus{\char`\_}
\def\PYZob{\char`\{}
\def\PYZcb{\char`\}}
\def\PYZca{\char`\^}
\def\PYZam{\char`\&}
\def\PYZlt{\char`\<}
\def\PYZgt{\char`\>}
\def\PYZsh{\char`\#}
\def\PYZpc{\char`\%}
\def\PYZdl{\char`\$}
\def\PYZhy{\char`\-}
\def\PYZsq{\char`\'}
\def\PYZdq{\char`\"}
\def\PYZti{\char`\~}
% for compatibility with earlier versions
\def\PYZat{@}
\def\PYZlb{[}
\def\PYZrb{]}
\makeatother


    % For linebreaks inside Verbatim environment from package fancyvrb.
    \makeatletter
        \newbox\Wrappedcontinuationbox
        \newbox\Wrappedvisiblespacebox
        \newcommand*\Wrappedvisiblespace {\textcolor{red}{\textvisiblespace}}
        \newcommand*\Wrappedcontinuationsymbol {\textcolor{red}{\llap{\tiny$\m@th\hookrightarrow$}}}
        \newcommand*\Wrappedcontinuationindent {3ex }
        \newcommand*\Wrappedafterbreak {\kern\Wrappedcontinuationindent\copy\Wrappedcontinuationbox}
        % Take advantage of the already applied Pygments mark-up to insert
        % potential linebreaks for TeX processing.
        %        {, <, #, %, $, ' and ": go to next line.
        %        _, }, ^, &, >, - and ~: stay at end of broken line.
        % Use of \textquotesingle for straight quote.
        \newcommand*\Wrappedbreaksatspecials {%
            \def\PYGZus{\discretionary{\char`\_}{\Wrappedafterbreak}{\char`\_}}%
            \def\PYGZob{\discretionary{}{\Wrappedafterbreak\char`\{}{\char`\{}}%
            \def\PYGZcb{\discretionary{\char`\}}{\Wrappedafterbreak}{\char`\}}}%
            \def\PYGZca{\discretionary{\char`\^}{\Wrappedafterbreak}{\char`\^}}%
            \def\PYGZam{\discretionary{\char`\&}{\Wrappedafterbreak}{\char`\&}}%
            \def\PYGZlt{\discretionary{}{\Wrappedafterbreak\char`\<}{\char`\<}}%
            \def\PYGZgt{\discretionary{\char`\>}{\Wrappedafterbreak}{\char`\>}}%
            \def\PYGZsh{\discretionary{}{\Wrappedafterbreak\char`\#}{\char`\#}}%
            \def\PYGZpc{\discretionary{}{\Wrappedafterbreak\char`\%}{\char`\%}}%
            \def\PYGZdl{\discretionary{}{\Wrappedafterbreak\char`\$}{\char`\$}}%
            \def\PYGZhy{\discretionary{\char`\-}{\Wrappedafterbreak}{\char`\-}}%
            \def\PYGZsq{\discretionary{}{\Wrappedafterbreak\textquotesingle}{\textquotesingle}}%
            \def\PYGZdq{\discretionary{}{\Wrappedafterbreak\char`\"}{\char`\"}}%
            \def\PYGZti{\discretionary{\char`\~}{\Wrappedafterbreak}{\char`\~}}%
        }
        % Some characters . , ; ? ! / are not pygmentized.
        % This macro makes them "active" and they will insert potential linebreaks
        \newcommand*\Wrappedbreaksatpunct {%
            \lccode`\~`\.\lowercase{\def~}{\discretionary{\hbox{\char`\.}}{\Wrappedafterbreak}{\hbox{\char`\.}}}%
            \lccode`\~`\,\lowercase{\def~}{\discretionary{\hbox{\char`\,}}{\Wrappedafterbreak}{\hbox{\char`\,}}}%
            \lccode`\~`\;\lowercase{\def~}{\discretionary{\hbox{\char`\;}}{\Wrappedafterbreak}{\hbox{\char`\;}}}%
            \lccode`\~`\:\lowercase{\def~}{\discretionary{\hbox{\char`\:}}{\Wrappedafterbreak}{\hbox{\char`\:}}}%
            \lccode`\~`\?\lowercase{\def~}{\discretionary{\hbox{\char`\?}}{\Wrappedafterbreak}{\hbox{\char`\?}}}%
            \lccode`\~`\!\lowercase{\def~}{\discretionary{\hbox{\char`\!}}{\Wrappedafterbreak}{\hbox{\char`\!}}}%
            \lccode`\~`\/\lowercase{\def~}{\discretionary{\hbox{\char`\/}}{\Wrappedafterbreak}{\hbox{\char`\/}}}%
            \catcode`\.\active
            \catcode`\,\active
            \catcode`\;\active
            \catcode`\:\active
            \catcode`\?\active
            \catcode`\!\active
            \catcode`\/\active
            \lccode`\~`\~
        }
    \makeatother

    \let\OriginalVerbatim=\Verbatim
    \makeatletter
    \renewcommand{\Verbatim}[1][1]{%
        %\parskip\z@skip
        \sbox\Wrappedcontinuationbox {\Wrappedcontinuationsymbol}%
        \sbox\Wrappedvisiblespacebox {\FV@SetupFont\Wrappedvisiblespace}%
        \def\FancyVerbFormatLine ##1{\hsize\linewidth
            \vtop{\raggedright\hyphenpenalty\z@\exhyphenpenalty\z@
                \doublehyphendemerits\z@\finalhyphendemerits\z@
                \strut ##1\strut}%
        }%
        % If the linebreak is at a space, the latter will be displayed as visible
        % space at end of first line, and a continuation symbol starts next line.
        % Stretch/shrink are however usually zero for typewriter font.
        \def\FV@Space {%
            \nobreak\hskip\z@ plus\fontdimen3\font minus\fontdimen4\font
            \discretionary{\copy\Wrappedvisiblespacebox}{\Wrappedafterbreak}
            {\kern\fontdimen2\font}%
        }%

        % Allow breaks at special characters using \PYG... macros.
        \Wrappedbreaksatspecials
        % Breaks at punctuation characters . , ; ? ! and / need catcode=\active
        \OriginalVerbatim[#1,codes*=\Wrappedbreaksatpunct]%
    }
    \makeatother

    % Exact colors from NB
    \definecolor{incolor}{HTML}{303F9F}
    \definecolor{outcolor}{HTML}{D84315}
    \definecolor{cellborder}{HTML}{CFCFCF}
    \definecolor{cellbackground}{HTML}{F7F7F7}

    % prompt
    \makeatletter
    \newcommand{\boxspacing}{\kern\kvtcb@left@rule\kern\kvtcb@boxsep}
    \makeatother
    \newcommand{\prompt}[4]{
        {\ttfamily\llap{{\color{#2}[#3]:\hspace{3pt}#4}}\vspace{-\baselineskip}}
    }
    

    
    % Prevent overflowing lines due to hard-to-break entities
    \sloppy
    % Setup hyperref package
    \hypersetup{
      breaklinks=true,  % so long urls are correctly broken across lines
      colorlinks=true,
      urlcolor=urlcolor,
      linkcolor=linkcolor,
      citecolor=citecolor,
      }
    % Slightly bigger margins than the latex defaults
    
    \geometry{verbose,tmargin=1in,bmargin=1in,lmargin=1in,rmargin=1in}
    
    

\begin{document}
    
    \maketitle
    
    

    
    \section{Summary}\label{summary}

這個 Jupyter Notebook 的目標是使用 NumPy 從頭開始實作一個簡單的
ANN,用於 Pima Indians Diabetes 資料集的糖尿病分類任務。涵蓋了以下步驟:

\begin{enumerate}
\def\labelenumi{\arabic{enumi}.}
\tightlist
\item
  \textbf{資料處理}: 從 Kaggle Hub 載入資料集、處理缺失值 (將特定欄位的
  0 值替換為中位數) 、將特徵和目標變數分開、打亂資料順序並分割為訓練集
  (80\%) 和測試集 (20\%) ,最後使用訓練集的最小最大值對資料進行 Min-Max
  標準化。
\item
  \textbf{建立神經網路}: 實作包含一個隱藏層 (使用 ReLU 或
  Sig\textbar moid 激活函數) 和一個輸出層 (使用 Sigmoid 激活函數) 的簡單
  ANN。使用平方誤差作為損失函數。
\item
  \textbf{訓練}: 使用隨機梯度下降法
  (SGD),一次處理一個樣本,對模型進行訓練。
\item
  \textbf{評估}:
  繪製訓練過程中的損失曲線,並在測試集上評估模型的準確率和混淆矩陣。
\end{enumerate}

    \begin{tcolorbox}[breakable, size=fbox, boxrule=1pt, pad at break*=1mm,colback=cellbackground, colframe=cellborder]
\prompt{In}{incolor}{1}{\boxspacing}
\begin{Verbatim}[commandchars=\\\{\}]
\PY{k+kn}{from}\PY{+w}{ }\PY{n+nn}{enum}\PY{+w}{ }\PY{k+kn}{import} \PY{n}{Enum}
\PY{c+c1}{\PYZsh{} 用 Enum 來表示 Activation Function,只能二擇一}
\PY{k}{class}\PY{+w}{ }\PY{n+nc}{ActivationFunction}\PY{p}{(}\PY{n}{Enum}\PY{p}{)}\PY{p}{:}
    \PY{n}{SIGMOID} \PY{o}{=} \PY{l+s+s1}{\PYZsq{}}\PY{l+s+s1}{sigmoid}\PY{l+s+s1}{\PYZsq{}}
    \PY{n}{RELU} \PY{o}{=} \PY{l+s+s1}{\PYZsq{}}\PY{l+s+s1}{relu}\PY{l+s+s1}{\PYZsq{}}
\end{Verbatim}
\end{tcolorbox}

    \section{Step 0: 設定參數}\label{step-0-ux8a2dux5b9aux53c3ux6578}

可以在這裡設定一些 const 的參數值

    \begin{tcolorbox}[breakable, size=fbox, boxrule=1pt, pad at break*=1mm,colback=cellbackground, colframe=cellborder]
\prompt{In}{incolor}{2}{\boxspacing}
\begin{Verbatim}[commandchars=\\\{\}]
\PY{n}{TEST\PYZus{}SPLIT} \PY{o}{=} \PY{l+m+mf}{0.2} \PY{c+c1}{\PYZsh{} 分割測試集的比例}
\PY{n}{RANDOM\PYZus{}SEED} \PY{o}{=} \PY{l+m+mi}{42} \PY{c+c1}{\PYZsh{} 亂數種子}

\PY{n}{INPUT\PYZus{}SIZE} \PY{o}{=} \PY{l+m+mi}{8}           \PY{c+c1}{\PYZsh{} 特徵數量 (固定為 8)}
\PY{n}{HIDDEN\PYZus{}SIZE} \PY{o}{=} \PY{l+m+mi}{16}         \PY{c+c1}{\PYZsh{} 隱藏層大小 (可調整,例如 8, 16, 32)}
\PY{n}{OUTPUT\PYZus{}SIZE} \PY{o}{=} \PY{l+m+mi}{1}          \PY{c+c1}{\PYZsh{} 輸出層大小 (固定為 1,二元分類)}
\PY{n}{HIDDEN\PYZus{}ACTIVATION} \PY{o}{=} \PY{n}{ActivationFunction}\PY{o}{.}\PY{n}{RELU} \PY{c+c1}{\PYZsh{} 隱藏層激活函數 (relu 或 sigmoid)}
\PY{n}{EPOCHS} \PY{o}{=} \PY{l+m+mi}{1500}            \PY{c+c1}{\PYZsh{} 訓練輪數 (可調整,例如 1000, 2000)}
\PY{n}{LEARNING\PYZus{}RATE} \PY{o}{=} \PY{l+m+mf}{0.01}     \PY{c+c1}{\PYZsh{} 學習率 (可調整,例如 0.1, 0.01, 0.001)}
\PY{n}{VERBOSE\PYZus{}STEP} \PY{o}{=} \PY{l+m+mi}{100}       \PY{c+c1}{\PYZsh{} 每隔多少輪印出一次損失}
\end{Verbatim}
\end{tcolorbox}

    \section{Step 1: Data Processing}\label{step-1-data-processing}

    \subsection{1.1 取得資料}\label{ux53d6ux5f97ux8cc7ux6599}

    \begin{tcolorbox}[breakable, size=fbox, boxrule=1pt, pad at break*=1mm,colback=cellbackground, colframe=cellborder]
\prompt{In}{incolor}{3}{\boxspacing}
\begin{Verbatim}[commandchars=\\\{\}]
\PY{k+kn}{import}\PY{+w}{ }\PY{n+nn}{kagglehub}
\end{Verbatim}
\end{tcolorbox}

    \begin{Verbatim}[commandchars=\\\{\}]
d:\textbackslash{}Repos\textbackslash{}nchu-ai-introduction\textbackslash{}.venv\textbackslash{}Lib\textbackslash{}site-packages\textbackslash{}tqdm\textbackslash{}auto.py:21:
TqdmWarning: IProgress not found. Please update jupyter and ipywidgets. See
https://ipywidgets.readthedocs.io/en/stable/user\_install.html
  from .autonotebook import tqdm as notebook\_tqdm
    \end{Verbatim}

    \begin{tcolorbox}[breakable, size=fbox, boxrule=1pt, pad at break*=1mm,colback=cellbackground, colframe=cellborder]
\prompt{In}{incolor}{4}{\boxspacing}
\begin{Verbatim}[commandchars=\\\{\}]
\PY{n}{path} \PY{o}{=} \PY{n}{kagglehub}\PY{o}{.}\PY{n}{dataset\PYZus{}download}\PY{p}{(}\PY{l+s+s2}{\PYZdq{}}\PY{l+s+s2}{uciml/pima\PYZhy{}indians\PYZhy{}diabetes\PYZhy{}database}\PY{l+s+s2}{\PYZdq{}}\PY{p}{)}
\PY{n}{path}
\end{Verbatim}
\end{tcolorbox}

            \begin{tcolorbox}[breakable, size=fbox, boxrule=.5pt, pad at break*=1mm, opacityfill=0]
\prompt{Out}{outcolor}{4}{\boxspacing}
\begin{Verbatim}[commandchars=\\\{\}]
'C:\textbackslash{}\textbackslash{}Users\textbackslash{}\textbackslash{}User\textbackslash{}\textbackslash{}.cache\textbackslash{}\textbackslash{}kagglehub\textbackslash{}\textbackslash{}datasets\textbackslash{}\textbackslash{}uciml\textbackslash{}\textbackslash{}pima-indians-diabetes-
database\textbackslash{}\textbackslash{}versions\textbackslash{}\textbackslash{}1'
\end{Verbatim}
\end{tcolorbox}
        
    \begin{tcolorbox}[breakable, size=fbox, boxrule=1pt, pad at break*=1mm,colback=cellbackground, colframe=cellborder]
\prompt{In}{incolor}{5}{\boxspacing}
\begin{Verbatim}[commandchars=\\\{\}]
\PY{k+kn}{import}\PY{+w}{ }\PY{n+nn}{pandas}\PY{+w}{ }\PY{k}{as}\PY{+w}{ }\PY{n+nn}{pd}
\PY{n}{df} \PY{o}{=} \PY{n}{pd}\PY{o}{.}\PY{n}{read\PYZus{}csv}\PY{p}{(}\PY{n}{path} \PY{o}{+} \PY{l+s+s2}{\PYZdq{}}\PY{l+s+s2}{/diabetes.csv}\PY{l+s+s2}{\PYZdq{}}\PY{p}{)}
\PY{n}{df}\PY{o}{.}\PY{n}{head}\PY{p}{(}\PY{p}{)}
\end{Verbatim}
\end{tcolorbox}

            \begin{tcolorbox}[breakable, size=fbox, boxrule=.5pt, pad at break*=1mm, opacityfill=0]
\prompt{Out}{outcolor}{5}{\boxspacing}
\begin{Verbatim}[commandchars=\\\{\}]
   Pregnancies  Glucose  BloodPressure  SkinThickness  Insulin   BMI  \textbackslash{}
0            6      148             72             35        0  33.6
1            1       85             66             29        0  26.6
2            8      183             64              0        0  23.3
3            1       89             66             23       94  28.1
4            0      137             40             35      168  43.1

   DiabetesPedigreeFunction  Age  Outcome
0                     0.627   50        1
1                     0.351   31        0
2                     0.672   32        1
3                     0.167   21        0
4                     2.288   33        1
\end{Verbatim}
\end{tcolorbox}
        
    \begin{tcolorbox}[breakable, size=fbox, boxrule=1pt, pad at break*=1mm,colback=cellbackground, colframe=cellborder]
\prompt{In}{incolor}{6}{\boxspacing}
\begin{Verbatim}[commandchars=\\\{\}]
\PY{n}{df}\PY{o}{.}\PY{n}{describe}\PY{p}{(}\PY{p}{)}
\end{Verbatim}
\end{tcolorbox}

            \begin{tcolorbox}[breakable, size=fbox, boxrule=.5pt, pad at break*=1mm, opacityfill=0]
\prompt{Out}{outcolor}{6}{\boxspacing}
\begin{Verbatim}[commandchars=\\\{\}]
       Pregnancies     Glucose  BloodPressure  SkinThickness     Insulin  \textbackslash{}
count   768.000000  768.000000     768.000000     768.000000  768.000000
mean      3.845052  120.894531      69.105469      20.536458   79.799479
std       3.369578   31.972618      19.355807      15.952218  115.244002
min       0.000000    0.000000       0.000000       0.000000    0.000000
25\%       1.000000   99.000000      62.000000       0.000000    0.000000
50\%       3.000000  117.000000      72.000000      23.000000   30.500000
75\%       6.000000  140.250000      80.000000      32.000000  127.250000
max      17.000000  199.000000     122.000000      99.000000  846.000000

              BMI  DiabetesPedigreeFunction         Age     Outcome
count  768.000000                768.000000  768.000000  768.000000
mean    31.992578                  0.471876   33.240885    0.348958
std      7.884160                  0.331329   11.760232    0.476951
min      0.000000                  0.078000   21.000000    0.000000
25\%     27.300000                  0.243750   24.000000    0.000000
50\%     32.000000                  0.372500   29.000000    0.000000
75\%     36.600000                  0.626250   41.000000    1.000000
max     67.100000                  2.420000   81.000000    1.000000
\end{Verbatim}
\end{tcolorbox}
        
    \subsection{1.2 處理缺失值}\label{ux8655ux7406ux7f3aux5931ux503c}

    \begin{tcolorbox}[breakable, size=fbox, boxrule=1pt, pad at break*=1mm,colback=cellbackground, colframe=cellborder]
\prompt{In}{incolor}{7}{\boxspacing}
\begin{Verbatim}[commandchars=\\\{\}]
\PY{n}{df}\PY{o}{.}\PY{n}{isnull}\PY{p}{(}\PY{p}{)}\PY{o}{.}\PY{n}{sum}\PY{p}{(}\PY{p}{)}
\end{Verbatim}
\end{tcolorbox}

            \begin{tcolorbox}[breakable, size=fbox, boxrule=.5pt, pad at break*=1mm, opacityfill=0]
\prompt{Out}{outcolor}{7}{\boxspacing}
\begin{Verbatim}[commandchars=\\\{\}]
Pregnancies                 0
Glucose                     0
BloodPressure               0
SkinThickness               0
Insulin                     0
BMI                         0
DiabetesPedigreeFunction    0
Age                         0
Outcome                     0
dtype: int64
\end{Verbatim}
\end{tcolorbox}
        
    從 \texttt{df.describe()} 結果可以看到沒有 null 值\\
接著處理缺失值直接被填為 0 的情況

    各個 Feature 如下:

\begin{itemize}
\tightlist
\item
  Pregnacies 懷孕次數:可能為 0
\item
  Glucose:血糖:不可能為 0
\item
  BloodPressure 血壓:不可能為 0
\item
  SkinThickness 皮膚厚度:不可能為 0
\item
  Insulin 胰島素濃度:極罕見情況下才會為 0
\item
  BMI:不可能為 0
\item
  DiabetesPedigreeFunction 糖尿病家族函數:在 df.describe() 可以看到沒有
  0
\item
  Age 年齡:這個資料集中最少為 21 歲,df.describe() 中可看到最小是 21
  沒有問題
\end{itemize}

    \begin{tcolorbox}[breakable, size=fbox, boxrule=1pt, pad at break*=1mm,colback=cellbackground, colframe=cellborder]
\prompt{In}{incolor}{8}{\boxspacing}
\begin{Verbatim}[commandchars=\\\{\}]
\PY{c+c1}{\PYZsh{} 該處理 0 值的欄位:}
\PY{n}{cols\PYZus{}to\PYZus{}impute} \PY{o}{=} \PY{p}{[}\PY{l+s+s1}{\PYZsq{}}\PY{l+s+s1}{Glucose}\PY{l+s+s1}{\PYZsq{}}\PY{p}{,} \PY{l+s+s1}{\PYZsq{}}\PY{l+s+s1}{BloodPressure}\PY{l+s+s1}{\PYZsq{}}\PY{p}{,} \PY{l+s+s1}{\PYZsq{}}\PY{l+s+s1}{SkinThickness}\PY{l+s+s1}{\PYZsq{}}\PY{p}{,} \PY{l+s+s1}{\PYZsq{}}\PY{l+s+s1}{Insulin}\PY{l+s+s1}{\PYZsq{}}\PY{p}{,} \PY{l+s+s1}{\PYZsq{}}\PY{l+s+s1}{BMI}\PY{l+s+s1}{\PYZsq{}}\PY{p}{]}
\PY{k}{for} \PY{n}{col} \PY{o+ow}{in} \PY{n}{cols\PYZus{}to\PYZus{}impute}\PY{p}{:}
    \PY{c+c1}{\PYZsh{} 計算非零值的中位數}
    \PY{n}{median\PYZus{}val} \PY{o}{=} \PY{n}{df}\PY{p}{[}\PY{n}{df}\PY{p}{[}\PY{n}{col}\PY{p}{]} \PY{o}{!=} \PY{l+m+mi}{0}\PY{p}{]}\PY{p}{[}\PY{n}{col}\PY{p}{]}\PY{o}{.}\PY{n}{median}\PY{p}{(}\PY{p}{)}
    \PY{c+c1}{\PYZsh{} 將 0 替換為中位數}
    \PY{n}{df}\PY{p}{[}\PY{n}{col}\PY{p}{]} \PY{o}{=} \PY{n}{df}\PY{p}{[}\PY{n}{col}\PY{p}{]}\PY{o}{.}\PY{n}{replace}\PY{p}{(}\PY{l+m+mi}{0}\PY{p}{,} \PY{n}{median\PYZus{}val}\PY{p}{)}
\end{Verbatim}
\end{tcolorbox}

    \begin{tcolorbox}[breakable, size=fbox, boxrule=1pt, pad at break*=1mm,colback=cellbackground, colframe=cellborder]
\prompt{In}{incolor}{9}{\boxspacing}
\begin{Verbatim}[commandchars=\\\{\}]
\PY{n}{df}\PY{o}{.}\PY{n}{describe}\PY{p}{(}\PY{p}{)} \PY{c+c1}{\PYZsh{} 看一下處理完的資料}
\end{Verbatim}
\end{tcolorbox}

            \begin{tcolorbox}[breakable, size=fbox, boxrule=.5pt, pad at break*=1mm, opacityfill=0]
\prompt{Out}{outcolor}{9}{\boxspacing}
\begin{Verbatim}[commandchars=\\\{\}]
       Pregnancies     Glucose  BloodPressure  SkinThickness     Insulin  \textbackslash{}
count   768.000000  768.000000     768.000000     768.000000  768.000000
mean      3.845052  121.656250      72.386719      29.108073  140.671875
std       3.369578   30.438286      12.096642       8.791221   86.383060
min       0.000000   44.000000      24.000000       7.000000   14.000000
25\%       1.000000   99.750000      64.000000      25.000000  121.500000
50\%       3.000000  117.000000      72.000000      29.000000  125.000000
75\%       6.000000  140.250000      80.000000      32.000000  127.250000
max      17.000000  199.000000     122.000000      99.000000  846.000000

              BMI  DiabetesPedigreeFunction         Age     Outcome
count  768.000000                768.000000  768.000000  768.000000
mean    32.455208                  0.471876   33.240885    0.348958
std      6.875177                  0.331329   11.760232    0.476951
min     18.200000                  0.078000   21.000000    0.000000
25\%     27.500000                  0.243750   24.000000    0.000000
50\%     32.300000                  0.372500   29.000000    0.000000
75\%     36.600000                  0.626250   41.000000    1.000000
max     67.100000                  2.420000   81.000000    1.000000
\end{Verbatim}
\end{tcolorbox}
        
    \subsection{1.3 分離 feature 與
outcome}\label{ux5206ux96e2-feature-ux8207-outcome}

    \begin{tcolorbox}[breakable, size=fbox, boxrule=1pt, pad at break*=1mm,colback=cellbackground, colframe=cellborder]
\prompt{In}{incolor}{10}{\boxspacing}
\begin{Verbatim}[commandchars=\\\{\}]
\PY{k+kn}{import}\PY{+w}{ }\PY{n+nn}{numpy}\PY{+w}{ }\PY{k}{as}\PY{+w}{ }\PY{n+nn}{np}
\end{Verbatim}
\end{tcolorbox}

    \begin{tcolorbox}[breakable, size=fbox, boxrule=1pt, pad at break*=1mm,colback=cellbackground, colframe=cellborder]
\prompt{In}{incolor}{11}{\boxspacing}
\begin{Verbatim}[commandchars=\\\{\}]
\PY{n}{x} \PY{o}{=} \PY{n}{df}\PY{o}{.}\PY{n}{drop}\PY{p}{(}\PY{l+s+s1}{\PYZsq{}}\PY{l+s+s1}{Outcome}\PY{l+s+s1}{\PYZsq{}}\PY{p}{,} \PY{n}{axis}\PY{o}{=}\PY{l+m+mi}{1}\PY{p}{)}\PY{o}{.}\PY{n}{values}
\PY{n}{y} \PY{o}{=} \PY{n}{np}\PY{o}{.}\PY{n}{array}\PY{p}{(}\PY{n}{df}\PY{p}{[}\PY{l+s+s1}{\PYZsq{}}\PY{l+s+s1}{Outcome}\PY{l+s+s1}{\PYZsq{}}\PY{p}{]}\PY{o}{.}\PY{n}{values}\PY{p}{)}\PY{o}{.}\PY{n}{reshape}\PY{p}{(}\PY{o}{\PYZhy{}}\PY{l+m+mi}{1}\PY{p}{,} \PY{l+m+mi}{1}\PY{p}{)} \PY{c+c1}{\PYZsh{} 確保 y 是 (n\PYZus{}samples, 1) 的形狀}
\end{Verbatim}
\end{tcolorbox}

    \begin{tcolorbox}[breakable, size=fbox, boxrule=1pt, pad at break*=1mm,colback=cellbackground, colframe=cellborder]
\prompt{In}{incolor}{12}{\boxspacing}
\begin{Verbatim}[commandchars=\\\{\}]
\PY{n}{x}
\end{Verbatim}
\end{tcolorbox}

            \begin{tcolorbox}[breakable, size=fbox, boxrule=.5pt, pad at break*=1mm, opacityfill=0]
\prompt{Out}{outcolor}{12}{\boxspacing}
\begin{Verbatim}[commandchars=\\\{\}]
array([[  6.   , 148.   ,  72.   , {\ldots},  33.6  ,   0.627,  50.   ],
       [  1.   ,  85.   ,  66.   , {\ldots},  26.6  ,   0.351,  31.   ],
       [  8.   , 183.   ,  64.   , {\ldots},  23.3  ,   0.672,  32.   ],
       {\ldots},
       [  5.   , 121.   ,  72.   , {\ldots},  26.2  ,   0.245,  30.   ],
       [  1.   , 126.   ,  60.   , {\ldots},  30.1  ,   0.349,  47.   ],
       [  1.   ,  93.   ,  70.   , {\ldots},  30.4  ,   0.315,  23.   ]],
      shape=(768, 8))
\end{Verbatim}
\end{tcolorbox}
        
    \begin{tcolorbox}[breakable, size=fbox, boxrule=1pt, pad at break*=1mm,colback=cellbackground, colframe=cellborder]
\prompt{In}{incolor}{13}{\boxspacing}
\begin{Verbatim}[commandchars=\\\{\}]
\PY{n}{y}\PY{p}{[}\PY{p}{:}\PY{l+m+mi}{5}\PY{p}{]}
\end{Verbatim}
\end{tcolorbox}

            \begin{tcolorbox}[breakable, size=fbox, boxrule=.5pt, pad at break*=1mm, opacityfill=0]
\prompt{Out}{outcolor}{13}{\boxspacing}
\begin{Verbatim}[commandchars=\\\{\}]
array([[1],
       [0],
       [1],
       [0],
       [1]])
\end{Verbatim}
\end{tcolorbox}
        
    \begin{tcolorbox}[breakable, size=fbox, boxrule=1pt, pad at break*=1mm,colback=cellbackground, colframe=cellborder]
\prompt{In}{incolor}{14}{\boxspacing}
\begin{Verbatim}[commandchars=\\\{\}]
\PY{n+nb}{print}\PY{p}{(}\PY{n}{x}\PY{o}{.}\PY{n}{shape}\PY{p}{)}
\PY{n+nb}{print}\PY{p}{(}\PY{n}{y}\PY{o}{.}\PY{n}{shape}\PY{p}{)}
\end{Verbatim}
\end{tcolorbox}

    \begin{Verbatim}[commandchars=\\\{\}]
(768, 8)
(768, 1)
    \end{Verbatim}

    可以看到 X 跟 y 可以一一對應,正確

    \subsection{1.4
打亂資料並分割}\label{ux6253ux4e82ux8cc7ux6599ux4e26ux5206ux5272}

    \begin{tcolorbox}[breakable, size=fbox, boxrule=1pt, pad at break*=1mm,colback=cellbackground, colframe=cellborder]
\prompt{In}{incolor}{15}{\boxspacing}
\begin{Verbatim}[commandchars=\\\{\}]
\PY{n}{np}\PY{o}{.}\PY{n}{random}\PY{o}{.}\PY{n}{seed}\PY{p}{(}\PY{n}{RANDOM\PYZus{}SEED}\PY{p}{)}
\PY{n}{indices} \PY{o}{=} \PY{n}{np}\PY{o}{.}\PY{n}{arange}\PY{p}{(}\PY{n}{x}\PY{o}{.}\PY{n}{shape}\PY{p}{[}\PY{l+m+mi}{0}\PY{p}{]}\PY{p}{)} \PY{c+c1}{\PYZsh{} 產生一個索引列表}
\PY{n}{np}\PY{o}{.}\PY{n}{random}\PY{o}{.}\PY{n}{shuffle}\PY{p}{(}\PY{n}{indices}\PY{p}{)} \PY{c+c1}{\PYZsh{} 打亂索引列表}
\PY{n}{x} \PY{o}{=} \PY{n}{x}\PY{p}{[}\PY{n}{indices}\PY{p}{]} \PY{c+c1}{\PYZsh{} 使用打亂後的索引來排 X}
\PY{n}{y} \PY{o}{=} \PY{n}{y}\PY{p}{[}\PY{n}{indices}\PY{p}{]} \PY{c+c1}{\PYZsh{} 使用打亂後的索引來排 y}
\end{Verbatim}
\end{tcolorbox}

    \begin{tcolorbox}[breakable, size=fbox, boxrule=1pt, pad at break*=1mm,colback=cellbackground, colframe=cellborder]
\prompt{In}{incolor}{16}{\boxspacing}
\begin{Verbatim}[commandchars=\\\{\}]
\PY{c+c1}{\PYZsh{} 分割資料為訓練集和測試集}
\PY{n}{split\PYZus{}idx} \PY{o}{=} \PY{n+nb}{int}\PY{p}{(}\PY{n}{x}\PY{o}{.}\PY{n}{shape}\PY{p}{[}\PY{l+m+mi}{0}\PY{p}{]} \PY{o}{*} \PY{p}{(}\PY{l+m+mi}{1} \PY{o}{\PYZhy{}} \PY{n}{TEST\PYZus{}SPLIT}\PY{p}{)}\PY{p}{)}
\PY{n}{X\PYZus{}train}\PY{p}{,} \PY{n}{X\PYZus{}test} \PY{o}{=} \PY{n}{x}\PY{p}{[}\PY{p}{:}\PY{n}{split\PYZus{}idx}\PY{p}{]}\PY{p}{,} \PY{n}{x}\PY{p}{[}\PY{n}{split\PYZus{}idx}\PY{p}{:}\PY{p}{]}
\PY{n}{y\PYZus{}train}\PY{p}{,} \PY{n}{y\PYZus{}test} \PY{o}{=} \PY{n}{y}\PY{p}{[}\PY{p}{:}\PY{n}{split\PYZus{}idx}\PY{p}{]}\PY{p}{,} \PY{n}{y}\PY{p}{[}\PY{n}{split\PYZus{}idx}\PY{p}{:}\PY{p}{]}
\end{Verbatim}
\end{tcolorbox}

    在分割完資料集以後我們需要把資料進行 Normalize

\begin{itemize}
\tightlist
\item
  值得注意的是這裡我只使用\textbf{訓練集}來計算 min,
  max,再一起應用到訓練集及測試集
\item
  這樣可以避免資料探勘的常見錯誤:拿未來的資料(測試集的資料我們訓練時不應該知道)來做訓練
\end{itemize}

    \begin{tcolorbox}[breakable, size=fbox, boxrule=1pt, pad at break*=1mm,colback=cellbackground, colframe=cellborder]
\prompt{In}{incolor}{17}{\boxspacing}
\begin{Verbatim}[commandchars=\\\{\}]
\PY{c+c1}{\PYZsh{} 標準化 (Min\PYZhy{}Max Scaling 到 0\PYZhy{}1 的範圍)}
\PY{c+c1}{\PYZsh{} 注意:只在訓練集上計算 min 和 max,然後應用到訓練集和測試集}
\PY{n}{min\PYZus{}vals} \PY{o}{=} \PY{n}{X\PYZus{}train}\PY{o}{.}\PY{n}{min}\PY{p}{(}\PY{n}{axis}\PY{o}{=}\PY{l+m+mi}{0}\PY{p}{)}
\PY{n}{max\PYZus{}vals} \PY{o}{=} \PY{n}{X\PYZus{}train}\PY{o}{.}\PY{n}{max}\PY{p}{(}\PY{n}{axis}\PY{o}{=}\PY{l+m+mi}{0}\PY{p}{)}

\PY{c+c1}{\PYZsh{} 防止除以零}
\PY{n}{range\PYZus{}vals} \PY{o}{=} \PY{n}{max\PYZus{}vals} \PY{o}{\PYZhy{}} \PY{n}{min\PYZus{}vals}
\PY{n}{range\PYZus{}vals}\PY{p}{[}\PY{n}{range\PYZus{}vals} \PY{o}{==} \PY{l+m+mi}{0}\PY{p}{]} \PY{o}{=} \PY{l+m+mi}{1} \PY{c+c1}{\PYZsh{} 如果某特徵所有值都相同時,避免除以零}

\PY{n}{X\PYZus{}train} \PY{o}{=} \PY{p}{(}\PY{n}{X\PYZus{}train} \PY{o}{\PYZhy{}} \PY{n}{min\PYZus{}vals}\PY{p}{)} \PY{o}{/} \PY{n}{range\PYZus{}vals}
\PY{n}{X\PYZus{}test} \PY{o}{=} \PY{p}{(}\PY{n}{X\PYZus{}test} \PY{o}{\PYZhy{}} \PY{n}{min\PYZus{}vals}\PY{p}{)} \PY{o}{/} \PY{n}{range\PYZus{}vals} \PY{c+c1}{\PYZsh{} 使用訓練集的 min/max 標準化測試集}
\end{Verbatim}
\end{tcolorbox}

    \begin{tcolorbox}[breakable, size=fbox, boxrule=1pt, pad at break*=1mm,colback=cellbackground, colframe=cellborder]
\prompt{In}{incolor}{18}{\boxspacing}
\begin{Verbatim}[commandchars=\\\{\}]
\PY{n+nb}{print}\PY{p}{(}\PY{l+s+s2}{\PYZdq{}}\PY{l+s+s2}{資料載入與預處理完成。}\PY{l+s+s2}{\PYZdq{}}\PY{p}{)}
\PY{n+nb}{print}\PY{p}{(}\PY{l+s+sa}{f}\PY{l+s+s2}{\PYZdq{}}\PY{l+s+s2}{訓練集大小: }\PY{l+s+si}{\PYZob{}}\PY{n}{X\PYZus{}train}\PY{o}{.}\PY{n}{shape}\PY{p}{[}\PY{l+m+mi}{0}\PY{p}{]}\PY{l+s+si}{\PYZcb{}}\PY{l+s+s2}{ 樣本}\PY{l+s+s2}{\PYZdq{}}\PY{p}{)}
\PY{n+nb}{print}\PY{p}{(}\PY{l+s+sa}{f}\PY{l+s+s2}{\PYZdq{}}\PY{l+s+s2}{測試集大小: }\PY{l+s+si}{\PYZob{}}\PY{n}{X\PYZus{}test}\PY{o}{.}\PY{n}{shape}\PY{p}{[}\PY{l+m+mi}{0}\PY{p}{]}\PY{l+s+si}{\PYZcb{}}\PY{l+s+s2}{ 樣本}\PY{l+s+s2}{\PYZdq{}}\PY{p}{)}
\PY{n+nb}{print}\PY{p}{(}\PY{l+s+sa}{f}\PY{l+s+s2}{\PYZdq{}}\PY{l+s+s2}{特徵數量: }\PY{l+s+si}{\PYZob{}}\PY{n}{X\PYZus{}train}\PY{o}{.}\PY{n}{shape}\PY{p}{[}\PY{l+m+mi}{1}\PY{p}{]}\PY{l+s+si}{\PYZcb{}}\PY{l+s+s2}{\PYZdq{}}\PY{p}{)}
\end{Verbatim}
\end{tcolorbox}

    \begin{Verbatim}[commandchars=\\\{\}]
資料載入與預處理完成。
訓練集大小: 614 樣本
測試集大小: 154 樣本
特徵數量: 8
    \end{Verbatim}

    \section{Step 2 + 3: Build Neural Network \& Training
Loop}\label{step-2-3-build-neural-network-training-loop}

這一步我們創建一個 SimpleANN,並且把 Training 實作在 Class 裡

    \subsection{2.1 Activation Function}\label{activation-function}

    \begin{tcolorbox}[breakable, size=fbox, boxrule=1pt, pad at break*=1mm,colback=cellbackground, colframe=cellborder]
\prompt{In}{incolor}{19}{\boxspacing}
\begin{Verbatim}[commandchars=\\\{\}]
\PY{k}{def}\PY{+w}{ }\PY{n+nf}{sigmoid}\PY{p}{(}\PY{n}{x}\PY{p}{)}\PY{p}{:}
\PY{+w}{    }\PY{l+s+sd}{\PYZdq{}\PYZdq{}\PYZdq{}Sigmoid activation function\PYZdq{}\PYZdq{}\PYZdq{}}
    \PY{c+c1}{\PYZsh{} 防止 overflow}
    \PY{n}{x} \PY{o}{=} \PY{n}{np}\PY{o}{.}\PY{n}{clip}\PY{p}{(}\PY{n}{x}\PY{p}{,} \PY{o}{\PYZhy{}}\PY{l+m+mi}{500}\PY{p}{,} \PY{l+m+mi}{500}\PY{p}{)}
    \PY{k}{return} \PY{l+m+mi}{1} \PY{o}{/} \PY{p}{(}\PY{l+m+mi}{1} \PY{o}{+} \PY{n}{np}\PY{o}{.}\PY{n}{exp}\PY{p}{(}\PY{o}{\PYZhy{}}\PY{n}{x}\PY{p}{)}\PY{p}{)}

\PY{k}{def}\PY{+w}{ }\PY{n+nf}{sigmoid\PYZus{}derivative}\PY{p}{(}\PY{n}{x}\PY{p}{)}\PY{p}{:}
\PY{+w}{    }\PY{l+s+sd}{\PYZdq{}\PYZdq{}\PYZdq{}Sigmoid activation function 的 derivative\PYZdq{}\PYZdq{}\PYZdq{}}
    \PY{n}{s} \PY{o}{=} \PY{n}{sigmoid}\PY{p}{(}\PY{n}{x}\PY{p}{)}
    \PY{k}{return} \PY{n}{s} \PY{o}{*} \PY{p}{(}\PY{l+m+mi}{1} \PY{o}{\PYZhy{}} \PY{n}{s}\PY{p}{)}

\PY{k}{def}\PY{+w}{ }\PY{n+nf}{relu}\PY{p}{(}\PY{n}{x}\PY{p}{)}\PY{p}{:}
\PY{+w}{    }\PY{l+s+sd}{\PYZdq{}\PYZdq{}\PYZdq{}ReLU activation function\PYZdq{}\PYZdq{}\PYZdq{}}
    \PY{k}{return} \PY{n}{np}\PY{o}{.}\PY{n}{maximum}\PY{p}{(}\PY{l+m+mi}{0}\PY{p}{,} \PY{n}{x}\PY{p}{)}

\PY{k}{def}\PY{+w}{ }\PY{n+nf}{relu\PYZus{}derivative}\PY{p}{(}\PY{n}{x}\PY{p}{)}\PY{p}{:}
\PY{+w}{    }\PY{l+s+sd}{\PYZdq{}\PYZdq{}\PYZdq{}ReLU activation function 的 derivative\PYZdq{}\PYZdq{}\PYZdq{}}
    \PY{k}{return} \PY{n}{np}\PY{o}{.}\PY{n}{where}\PY{p}{(}\PY{n}{x} \PY{o}{\PYZgt{}} \PY{l+m+mi}{0}\PY{p}{,} \PY{l+m+mi}{1}\PY{p}{,} \PY{l+m+mi}{0}\PY{p}{)}
\end{Verbatim}
\end{tcolorbox}

    \subsection{2.2 Loss Function}\label{loss-function}

    \begin{tcolorbox}[breakable, size=fbox, boxrule=1pt, pad at break*=1mm,colback=cellbackground, colframe=cellborder]
\prompt{In}{incolor}{20}{\boxspacing}
\begin{Verbatim}[commandchars=\\\{\}]
\PY{k}{def}\PY{+w}{ }\PY{n+nf}{square\PYZus{}error\PYZus{}loss}\PY{p}{(}\PY{n}{y\PYZus{}true}\PY{p}{,} \PY{n}{y\PYZus{}pred}\PY{p}{)}\PY{p}{:}
\PY{+w}{    }\PY{l+s+sd}{\PYZdq{}\PYZdq{}\PYZdq{}平方誤差損失函數\PYZdq{}\PYZdq{}\PYZdq{}}
    \PY{k}{return} \PY{l+m+mf}{0.5} \PY{o}{*} \PY{n}{np}\PY{o}{.}\PY{n}{mean}\PY{p}{(}\PY{p}{(}\PY{n}{y\PYZus{}pred} \PY{o}{\PYZhy{}} \PY{n}{y\PYZus{}true}\PY{p}{)}\PY{o}{*}\PY{o}{*}\PY{l+m+mi}{2}\PY{p}{)}

\PY{k}{def}\PY{+w}{ }\PY{n+nf}{square\PYZus{}error\PYZus{}loss\PYZus{}derivative}\PY{p}{(}\PY{n}{y\PYZus{}true}\PY{p}{,} \PY{n}{y\PYZus{}pred}\PY{p}{)}\PY{p}{:}
\PY{+w}{    }\PY{l+s+sd}{\PYZdq{}\PYZdq{}\PYZdq{}平方誤差損失函數的導數\PYZdq{}\PYZdq{}\PYZdq{}}
    \PY{k}{return} \PY{n}{y\PYZus{}pred} \PY{o}{\PYZhy{}} \PY{n}{y\PYZus{}true}
\end{Verbatim}
\end{tcolorbox}

    \subsection{2.3 實作 SimpleANN
Class}\label{ux5be6ux4f5c-simpleann-class}

    \begin{tcolorbox}[breakable, size=fbox, boxrule=1pt, pad at break*=1mm,colback=cellbackground, colframe=cellborder]
\prompt{In}{incolor}{21}{\boxspacing}
\begin{Verbatim}[commandchars=\\\{\}]
\PY{k+kn}{import}\PY{+w}{ }\PY{n+nn}{time}
\end{Verbatim}
\end{tcolorbox}

    \begin{tcolorbox}[breakable, size=fbox, boxrule=1pt, pad at break*=1mm,colback=cellbackground, colframe=cellborder]
\prompt{In}{incolor}{22}{\boxspacing}
\begin{Verbatim}[commandchars=\\\{\}]
\PY{k}{class}\PY{+w}{ }\PY{n+nc}{SimpleANN}\PY{p}{:}
\PY{+w}{    }\PY{l+s+sd}{\PYZdq{}\PYZdq{}\PYZdq{}}
\PY{l+s+sd}{    一個簡單的 NumPy 實現的 ANN。}
\PY{l+s+sd}{    \PYZdq{}\PYZdq{}\PYZdq{}}

    \PY{k}{def}\PY{+w}{ }\PY{n+nf+fm}{\PYZus{}\PYZus{}init\PYZus{}\PYZus{}}\PY{p}{(}
        \PY{n+nb+bp}{self}\PY{p}{,}
        \PY{n}{input\PYZus{}size}\PY{p}{,}
        \PY{n}{hidden\PYZus{}size}\PY{p}{,}
        \PY{n}{output\PYZus{}size}\PY{p}{,}
        \PY{n}{hidden\PYZus{}activation}\PY{p}{:} \PY{n}{ActivationFunction} \PY{o}{=} \PY{n}{ActivationFunction}\PY{o}{.}\PY{n}{RELU}\PY{p}{,}
        \PY{n}{seed}\PY{o}{=}\PY{l+m+mi}{42}\PY{p}{,}
    \PY{p}{)}\PY{p}{:}
\PY{+w}{        }\PY{l+s+sd}{\PYZdq{}\PYZdq{}\PYZdq{}}
\PY{l+s+sd}{        初始化神經網路。}

\PY{l+s+sd}{        Args:}
\PY{l+s+sd}{            input\PYZus{}size (int): 輸入層神經元數量 (特徵數)。}
\PY{l+s+sd}{            hidden\PYZus{}size (int): 隱藏層神經元數量。}
\PY{l+s+sd}{            output\PYZus{}size (int): 輸出層神經元數量 (通常為 1 用於二元分類)。}
\PY{l+s+sd}{            hidden\PYZus{}activation (str): 隱藏層激活函數 (relu 或 sigmoid),是 Enum。}
\PY{l+s+sd}{            seed (int): 權重初始化的隨機種子。}
\PY{l+s+sd}{        \PYZdq{}\PYZdq{}\PYZdq{}}

        \PY{n+nb+bp}{self}\PY{o}{.}\PY{n}{input\PYZus{}size} \PY{o}{=} \PY{n}{input\PYZus{}size}
        \PY{n+nb+bp}{self}\PY{o}{.}\PY{n}{hidden\PYZus{}size} \PY{o}{=} \PY{n}{hidden\PYZus{}size}
        \PY{n+nb+bp}{self}\PY{o}{.}\PY{n}{output\PYZus{}size} \PY{o}{=} \PY{n}{output\PYZus{}size}
        \PY{n+nb+bp}{self}\PY{o}{.}\PY{n}{seed} \PY{o}{=} \PY{n}{seed}
        \PY{n}{np}\PY{o}{.}\PY{n}{random}\PY{o}{.}\PY{n}{seed}\PY{p}{(}\PY{n+nb+bp}{self}\PY{o}{.}\PY{n}{seed}\PY{p}{)}

        \PY{c+c1}{\PYZsh{} 權重初始化 (Xavier/Glorot initialization 變體)}
        \PY{c+c1}{\PYZsh{} 這有助於防止梯度消失或爆炸}
        \PY{n}{limit\PYZus{}w1} \PY{o}{=} \PY{n}{np}\PY{o}{.}\PY{n}{sqrt}\PY{p}{(}\PY{l+m+mf}{6.0} \PY{o}{/} \PY{p}{(}\PY{n}{input\PYZus{}size} \PY{o}{+} \PY{n}{hidden\PYZus{}size}\PY{p}{)}\PY{p}{)}
        \PY{n+nb+bp}{self}\PY{o}{.}\PY{n}{W1} \PY{o}{=} \PY{n}{np}\PY{o}{.}\PY{n}{random}\PY{o}{.}\PY{n}{uniform}\PY{p}{(}\PY{o}{\PYZhy{}}\PY{n}{limit\PYZus{}w1}\PY{p}{,} \PY{n}{limit\PYZus{}w1}\PY{p}{,} \PY{p}{(}\PY{n+nb+bp}{self}\PY{o}{.}\PY{n}{input\PYZus{}size}\PY{p}{,} \PY{n+nb+bp}{self}\PY{o}{.}\PY{n}{hidden\PYZus{}size}\PY{p}{)}\PY{p}{)}
        \PY{n+nb+bp}{self}\PY{o}{.}\PY{n}{b1} \PY{o}{=} \PY{n}{np}\PY{o}{.}\PY{n}{zeros}\PY{p}{(}\PY{p}{(}\PY{l+m+mi}{1}\PY{p}{,} \PY{n+nb+bp}{self}\PY{o}{.}\PY{n}{hidden\PYZus{}size}\PY{p}{)}\PY{p}{)}  \PY{c+c1}{\PYZsh{} 初始化 bias 為 0}

        \PY{n}{limit\PYZus{}w2} \PY{o}{=} \PY{n}{np}\PY{o}{.}\PY{n}{sqrt}\PY{p}{(}\PY{l+m+mf}{6.0} \PY{o}{/} \PY{p}{(}\PY{n}{hidden\PYZus{}size} \PY{o}{+} \PY{n}{output\PYZus{}size}\PY{p}{)}\PY{p}{)}
        \PY{n+nb+bp}{self}\PY{o}{.}\PY{n}{W2} \PY{o}{=} \PY{n}{np}\PY{o}{.}\PY{n}{random}\PY{o}{.}\PY{n}{uniform}\PY{p}{(}\PY{o}{\PYZhy{}}\PY{n}{limit\PYZus{}w2}\PY{p}{,} \PY{n}{limit\PYZus{}w2}\PY{p}{,} \PY{p}{(}\PY{n+nb+bp}{self}\PY{o}{.}\PY{n}{hidden\PYZus{}size}\PY{p}{,} \PY{n+nb+bp}{self}\PY{o}{.}\PY{n}{output\PYZus{}size}\PY{p}{)}\PY{p}{)}
        \PY{n+nb+bp}{self}\PY{o}{.}\PY{n}{b2} \PY{o}{=} \PY{n}{np}\PY{o}{.}\PY{n}{zeros}\PY{p}{(}\PY{p}{(}\PY{l+m+mi}{1}\PY{p}{,} \PY{n+nb+bp}{self}\PY{o}{.}\PY{n}{output\PYZus{}size}\PY{p}{)}\PY{p}{)}  \PY{c+c1}{\PYZsh{} 初始化 bias 為 0}

        \PY{c+c1}{\PYZsh{} 選擇隱藏層 Activation Function}
        \PY{k}{if} \PY{n}{hidden\PYZus{}activation} \PY{o}{==} \PY{n}{ActivationFunction}\PY{o}{.}\PY{n}{RELU}\PY{p}{:}
            \PY{n+nb+bp}{self}\PY{o}{.}\PY{n}{hidden\PYZus{}activation} \PY{o}{=} \PY{n}{relu}
            \PY{n+nb+bp}{self}\PY{o}{.}\PY{n}{hidden\PYZus{}activation\PYZus{}derivative} \PY{o}{=} \PY{n}{relu\PYZus{}derivative}
        \PY{k}{elif} \PY{n}{hidden\PYZus{}activation} \PY{o}{==} \PY{n}{ActivationFunction}\PY{o}{.}\PY{n}{SIGMOID}\PY{p}{:}
            \PY{n+nb+bp}{self}\PY{o}{.}\PY{n}{hidden\PYZus{}activation} \PY{o}{=} \PY{n}{sigmoid}
            \PY{n+nb+bp}{self}\PY{o}{.}\PY{n}{hidden\PYZus{}activation\PYZus{}derivative} \PY{o}{=} \PY{n}{sigmoid\PYZus{}derivative}
        \PY{k}{else}\PY{p}{:}
            \PY{k}{raise} \PY{n+ne}{ValueError}\PY{p}{(}\PY{l+s+s2}{\PYZdq{}}\PY{l+s+s2}{隱藏層激活函數必須是 ActivationFunction.RELU }\PY{l+s+s2}{\PYZdq{}} \PY{o}{+} \PY{l+s+s2}{\PYZdq{}}\PY{l+s+s2}{或 ActivationFunction.SIGMOID}\PY{l+s+s2}{\PYZdq{}}\PY{p}{)}

        \PY{c+c1}{\PYZsh{} 輸出層固定使用 Sigmoid}
        \PY{n+nb+bp}{self}\PY{o}{.}\PY{n}{output\PYZus{}activation} \PY{o}{=} \PY{n}{sigmoid}
        \PY{n+nb+bp}{self}\PY{o}{.}\PY{n}{output\PYZus{}activation\PYZus{}derivative} \PY{o}{=} \PY{n}{sigmoid\PYZus{}derivative}

        \PY{n+nb}{print}\PY{p}{(}\PY{l+s+s2}{\PYZdq{}}\PY{l+s+s2}{神經網路初始化完成:}\PY{l+s+s2}{\PYZdq{}}\PY{p}{)}
        \PY{n+nb}{print}\PY{p}{(}\PY{l+s+sa}{f}\PY{l+s+s2}{\PYZdq{}}\PY{l+s+s2}{  \PYZhy{} 輸入層大小: }\PY{l+s+si}{\PYZob{}}\PY{n}{input\PYZus{}size}\PY{l+s+si}{\PYZcb{}}\PY{l+s+s2}{\PYZdq{}}\PY{p}{)}
        \PY{n+nb}{print}\PY{p}{(}\PY{l+s+sa}{f}\PY{l+s+s2}{\PYZdq{}}\PY{l+s+s2}{  \PYZhy{} 隱藏層大小: }\PY{l+s+si}{\PYZob{}}\PY{n}{hidden\PYZus{}size}\PY{l+s+si}{\PYZcb{}}\PY{l+s+s2}{\PYZdq{}}\PY{p}{)}
        \PY{n+nb}{print}\PY{p}{(}\PY{l+s+sa}{f}\PY{l+s+s2}{\PYZdq{}}\PY{l+s+s2}{  \PYZhy{} 輸出層大小: }\PY{l+s+si}{\PYZob{}}\PY{n}{output\PYZus{}size}\PY{l+s+si}{\PYZcb{}}\PY{l+s+s2}{\PYZdq{}}\PY{p}{)}
        \PY{n+nb}{print}\PY{p}{(}\PY{l+s+sa}{f}\PY{l+s+s2}{\PYZdq{}}\PY{l+s+s2}{  \PYZhy{} 隱藏層激活函數: }\PY{l+s+si}{\PYZob{}}\PY{n}{hidden\PYZus{}activation}\PY{l+s+si}{\PYZcb{}}\PY{l+s+s2}{\PYZdq{}}\PY{p}{)}
        \PY{n+nb}{print}\PY{p}{(}\PY{l+s+s2}{\PYZdq{}}\PY{l+s+s2}{  \PYZhy{} 輸出層激活函數: sigmoid}\PY{l+s+s2}{\PYZdq{}}\PY{p}{)}

    \PY{k}{def}\PY{+w}{ }\PY{n+nf}{forward}\PY{p}{(}\PY{n+nb+bp}{self}\PY{p}{,} \PY{n}{X}\PY{p}{)}\PY{p}{:}  \PY{c+c1}{\PYZsh{} noqa: N803}
\PY{+w}{        }\PY{l+s+sd}{\PYZdq{}\PYZdq{}\PYZdq{}}
\PY{l+s+sd}{        執行前向傳播。}

\PY{l+s+sd}{        Args:}
\PY{l+s+sd}{            X (np.ndarray): 輸入資料,形狀為 (n\PYZus{}samples, input\PYZus{}size)。}

\PY{l+s+sd}{        Returns:}
\PY{l+s+sd}{            tuple: (hidden\PYZus{}layer\PYZus{}output, final\PYZus{}output)}
\PY{l+s+sd}{                    包含隱藏層和輸出層的輸出。}
\PY{l+s+sd}{        \PYZdq{}\PYZdq{}\PYZdq{}}
        \PY{c+c1}{\PYZsh{} 輸入層到隱藏層}
        \PY{n+nb+bp}{self}\PY{o}{.}\PY{n}{z1} \PY{o}{=} \PY{n}{X} \PY{o}{@} \PY{n+nb+bp}{self}\PY{o}{.}\PY{n}{W1} \PY{o}{+} \PY{n+nb+bp}{self}\PY{o}{.}\PY{n}{b1}
        \PY{n+nb+bp}{self}\PY{o}{.}\PY{n}{a1} \PY{o}{=} \PY{n+nb+bp}{self}\PY{o}{.}\PY{n}{hidden\PYZus{}activation}\PY{p}{(}\PY{n+nb+bp}{self}\PY{o}{.}\PY{n}{z1}\PY{p}{)}

        \PY{c+c1}{\PYZsh{} 隱藏層到輸出層}
        \PY{n+nb+bp}{self}\PY{o}{.}\PY{n}{z2} \PY{o}{=} \PY{n+nb+bp}{self}\PY{o}{.}\PY{n}{a1} \PY{o}{@} \PY{n+nb+bp}{self}\PY{o}{.}\PY{n}{W2} \PY{o}{+} \PY{n+nb+bp}{self}\PY{o}{.}\PY{n}{b2}
        \PY{n+nb+bp}{self}\PY{o}{.}\PY{n}{a2} \PY{o}{=} \PY{n+nb+bp}{self}\PY{o}{.}\PY{n}{output\PYZus{}activation}\PY{p}{(}\PY{n+nb+bp}{self}\PY{o}{.}\PY{n}{z2}\PY{p}{)}  \PY{c+c1}{\PYZsh{} 最終輸出}

        \PY{k}{return} \PY{n+nb+bp}{self}\PY{o}{.}\PY{n}{a1}\PY{p}{,} \PY{n+nb+bp}{self}\PY{o}{.}\PY{n}{a2}

    \PY{k}{def}\PY{+w}{ }\PY{n+nf}{backward}\PY{p}{(}\PY{n+nb+bp}{self}\PY{p}{,} \PY{n}{X}\PY{p}{,} \PY{n}{y\PYZus{}true}\PY{p}{,} \PY{n}{y\PYZus{}pred}\PY{p}{)}\PY{p}{:}  \PY{c+c1}{\PYZsh{} noqa: N803}
\PY{+w}{        }\PY{l+s+sd}{\PYZdq{}\PYZdq{}\PYZdq{}}
\PY{l+s+sd}{        執行反向傳播並計算梯度。}

\PY{l+s+sd}{        Args:}
\PY{l+s+sd}{            X (np.ndarray): 單個輸入樣本,形狀為 (1, input\PYZus{}size)。}
\PY{l+s+sd}{            y\PYZus{}true (np.ndarray): 單個真實標籤,形狀為 (1, output\PYZus{}size)。}
\PY{l+s+sd}{            y\PYZus{}pred (np.ndarray): 單個預測輸出,形狀為 (1, output\PYZus{}size)。}

\PY{l+s+sd}{        Returns:}
\PY{l+s+sd}{            tuple: (dW1, db1, dW2, db2)}
\PY{l+s+sd}{                    權重和偏置的梯度。}
\PY{l+s+sd}{        \PYZdq{}\PYZdq{}\PYZdq{}}
        \PY{c+c1}{\PYZsh{} 計算輸出層的誤差和梯度}
        \PY{c+c1}{\PYZsh{} 損失函數對輸出層激活值的導數 * 輸出層激活函數對其輸入的導數}
        \PY{n}{delta\PYZus{}output} \PY{o}{=} \PY{n}{square\PYZus{}error\PYZus{}loss\PYZus{}derivative}\PY{p}{(}\PY{n}{y\PYZus{}true}\PY{p}{,} \PY{n}{y\PYZus{}pred}\PY{p}{)} \PY{o}{*} \PY{n+nb+bp}{self}\PY{o}{.}\PY{n}{output\PYZus{}activation\PYZus{}derivative}\PY{p}{(}\PY{n+nb+bp}{self}\PY{o}{.}\PY{n}{z2}\PY{p}{)}
        \PY{c+c1}{\PYZsh{} (1, output\PYZus{}size)}

        \PY{c+c1}{\PYZsh{} 計算隱藏層的誤差和梯度}
        \PY{c+c1}{\PYZsh{} 輸出層誤差 @ W2 的轉置 * 隱藏層激活函數對其輸入的導數}
        \PY{n}{delta\PYZus{}hidden} \PY{o}{=} \PY{p}{(}\PY{n}{delta\PYZus{}output} \PY{o}{@} \PY{n+nb+bp}{self}\PY{o}{.}\PY{n}{W2}\PY{o}{.}\PY{n}{T}\PY{p}{)} \PY{o}{*} \PY{n+nb+bp}{self}\PY{o}{.}\PY{n}{hidden\PYZus{}activation\PYZus{}derivative}\PY{p}{(}\PY{n+nb+bp}{self}\PY{o}{.}\PY{n}{z1}\PY{p}{)}
        \PY{c+c1}{\PYZsh{} (1, hidden\PYZus{}size)}

        \PY{c+c1}{\PYZsh{} 計算權重和偏置的梯度}
        \PY{c+c1}{\PYZsh{} 梯度 = 上一層的激活值轉置 @ 當前層的誤差}
        \PY{n}{dW2} \PY{o}{=} \PY{n+nb+bp}{self}\PY{o}{.}\PY{n}{a1}\PY{o}{.}\PY{n}{T} \PY{o}{@} \PY{n}{delta\PYZus{}output}  \PY{c+c1}{\PYZsh{} noqa: N806}
        \PY{c+c1}{\PYZsh{} (hidden\PYZus{}size, 1) @ (1, output\PYZus{}size) \PYZhy{}\PYZgt{} (hidden\PYZus{}size, output\PYZus{}size)}
        \PY{n}{db2} \PY{o}{=} \PY{n}{np}\PY{o}{.}\PY{n}{sum}\PY{p}{(}\PY{n}{delta\PYZus{}output}\PY{p}{,} \PY{n}{axis}\PY{o}{=}\PY{l+m+mi}{0}\PY{p}{,} \PY{n}{keepdims}\PY{o}{=}\PY{k+kc}{True}\PY{p}{)}  \PY{c+c1}{\PYZsh{} (1, output\PYZus{}size)}

        \PY{n}{dW1} \PY{o}{=} \PY{n}{X}\PY{o}{.}\PY{n}{T} \PY{o}{@} \PY{n}{delta\PYZus{}hidden}  \PY{c+c1}{\PYZsh{} (input\PYZus{}size, 1) @ (1, hidden\PYZus{}size) \PYZhy{}\PYZgt{} (input\PYZus{}size, hidden\PYZus{}size)  \PYZsh{} noqa: N806}
        \PY{n}{db1} \PY{o}{=} \PY{n}{np}\PY{o}{.}\PY{n}{sum}\PY{p}{(}\PY{n}{delta\PYZus{}hidden}\PY{p}{,} \PY{n}{axis}\PY{o}{=}\PY{l+m+mi}{0}\PY{p}{,} \PY{n}{keepdims}\PY{o}{=}\PY{k+kc}{True}\PY{p}{)}  \PY{c+c1}{\PYZsh{} (1, hidden\PYZus{}size)}

        \PY{k}{return} \PY{n}{dW1}\PY{p}{,} \PY{n}{db1}\PY{p}{,} \PY{n}{dW2}\PY{p}{,} \PY{n}{db2}

    \PY{k}{def}\PY{+w}{ }\PY{n+nf}{update\PYZus{}weights}\PY{p}{(}\PY{n+nb+bp}{self}\PY{p}{,} \PY{n}{dW1}\PY{p}{,} \PY{n}{db1}\PY{p}{,} \PY{n}{dW2}\PY{p}{,} \PY{n}{db2}\PY{p}{,} \PY{n}{learning\PYZus{}rate}\PY{p}{)}\PY{p}{:}  \PY{c+c1}{\PYZsh{} noqa: N803}
\PY{+w}{        }\PY{l+s+sd}{\PYZdq{}\PYZdq{}\PYZdq{}}
\PY{l+s+sd}{        使用 SGD 更新 weight 和 bias。}

\PY{l+s+sd}{        Args:}
\PY{l+s+sd}{            dW1, db1, dW2, db2: 梯度。}
\PY{l+s+sd}{            learning\PYZus{}rate (float): 學習率。}
\PY{l+s+sd}{        \PYZdq{}\PYZdq{}\PYZdq{}}
        \PY{n+nb+bp}{self}\PY{o}{.}\PY{n}{W1} \PY{o}{\PYZhy{}}\PY{o}{=} \PY{n}{learning\PYZus{}rate} \PY{o}{*} \PY{n}{dW1}
        \PY{n+nb+bp}{self}\PY{o}{.}\PY{n}{b1} \PY{o}{\PYZhy{}}\PY{o}{=} \PY{n}{learning\PYZus{}rate} \PY{o}{*} \PY{n}{db1}
        \PY{n+nb+bp}{self}\PY{o}{.}\PY{n}{W2} \PY{o}{\PYZhy{}}\PY{o}{=} \PY{n}{learning\PYZus{}rate} \PY{o}{*} \PY{n}{dW2}
        \PY{n+nb+bp}{self}\PY{o}{.}\PY{n}{b2} \PY{o}{\PYZhy{}}\PY{o}{=} \PY{n}{learning\PYZus{}rate} \PY{o}{*} \PY{n}{db2}

    \PY{k}{def}\PY{+w}{ }\PY{n+nf}{train}\PY{p}{(}\PY{n+nb+bp}{self}\PY{p}{,} \PY{n}{X\PYZus{}train}\PY{p}{,} \PY{n}{y\PYZus{}train}\PY{p}{,} \PY{n}{epochs}\PY{p}{,} \PY{n}{learning\PYZus{}rate}\PY{p}{,} \PY{n}{verbose}\PY{o}{=}\PY{l+m+mi}{100}\PY{p}{)}\PY{p}{:}  \PY{c+c1}{\PYZsh{} noqa: N803}
\PY{+w}{        }\PY{l+s+sd}{\PYZdq{}\PYZdq{}\PYZdq{}}
\PY{l+s+sd}{        訓練神經網路。}

\PY{l+s+sd}{        Args:}
\PY{l+s+sd}{            X\PYZus{}train (np.ndarray): 訓練特徵。}
\PY{l+s+sd}{            y\PYZus{}train (np.ndarray): 訓練標籤。}
\PY{l+s+sd}{            epochs (int): 訓練的輪數。}
\PY{l+s+sd}{            learning\PYZus{}rate (float): 學習率。}
\PY{l+s+sd}{            verbose (int): 每隔多少輪印出一次損失。}

\PY{l+s+sd}{        Returns:}
\PY{l+s+sd}{            list: 每輪的平均損失列表。}
\PY{l+s+sd}{        \PYZdq{}\PYZdq{}\PYZdq{}}
        \PY{n}{start\PYZus{}time} \PY{o}{=} \PY{n}{time}\PY{o}{.}\PY{n}{time}\PY{p}{(}\PY{p}{)}
        \PY{n}{losses} \PY{o}{=} \PY{p}{[}\PY{p}{]}
        \PY{n}{n\PYZus{}samples} \PY{o}{=} \PY{n}{X\PYZus{}train}\PY{o}{.}\PY{n}{shape}\PY{p}{[}\PY{l+m+mi}{0}\PY{p}{]}

        \PY{n+nb}{print}\PY{p}{(}\PY{l+s+s2}{\PYZdq{}}\PY{l+s+se}{\PYZbs{}n}\PY{l+s+s2}{開始訓練...}\PY{l+s+s2}{\PYZdq{}}\PY{p}{)}
        \PY{n+nb}{print}\PY{p}{(}\PY{l+s+sa}{f}\PY{l+s+s2}{\PYZdq{}}\PY{l+s+s2}{  \PYZhy{} Epochs: }\PY{l+s+si}{\PYZob{}}\PY{n}{epochs}\PY{l+s+si}{\PYZcb{}}\PY{l+s+s2}{\PYZdq{}}\PY{p}{)}
        \PY{n+nb}{print}\PY{p}{(}\PY{l+s+sa}{f}\PY{l+s+s2}{\PYZdq{}}\PY{l+s+s2}{  \PYZhy{} Learning Rate: }\PY{l+s+si}{\PYZob{}}\PY{n}{learning\PYZus{}rate}\PY{l+s+si}{\PYZcb{}}\PY{l+s+s2}{\PYZdq{}}\PY{p}{)}
        \PY{n+nb}{print}\PY{p}{(}\PY{l+s+sa}{f}\PY{l+s+s2}{\PYZdq{}}\PY{l+s+s2}{  \PYZhy{} 訓練樣本數: }\PY{l+s+si}{\PYZob{}}\PY{n}{n\PYZus{}samples}\PY{l+s+si}{\PYZcb{}}\PY{l+s+s2}{\PYZdq{}}\PY{p}{)}

        \PY{k}{for} \PY{n}{epoch} \PY{o+ow}{in} \PY{n+nb}{range}\PY{p}{(}\PY{n}{epochs}\PY{p}{)}\PY{p}{:}
            \PY{n}{epoch\PYZus{}loss} \PY{o}{=} \PY{l+m+mi}{0}
            \PY{c+c1}{\PYZsh{} \PYZhy{}\PYZhy{}\PYZhy{} SGD: 遍歷每個樣本 \PYZhy{}\PYZhy{}\PYZhy{}}
            \PY{k}{for} \PY{n}{i} \PY{o+ow}{in} \PY{n+nb}{range}\PY{p}{(}\PY{n}{n\PYZus{}samples}\PY{p}{)}\PY{p}{:}
                \PY{c+c1}{\PYZsh{} 獲取單個樣本,並確保其形狀正確 (1, n\PYZus{}features)}
                \PY{n}{x\PYZus{}sample} \PY{o}{=} \PY{n}{X\PYZus{}train}\PY{p}{[}\PY{n}{i} \PY{p}{:} \PY{n}{i} \PY{o}{+} \PY{l+m+mi}{1}\PY{p}{,} \PY{p}{:}\PY{p}{]}
                \PY{n}{y\PYZus{}sample} \PY{o}{=} \PY{n}{y\PYZus{}train}\PY{p}{[}\PY{n}{i} \PY{p}{:} \PY{n}{i} \PY{o}{+} \PY{l+m+mi}{1}\PY{p}{,} \PY{p}{:}\PY{p}{]}

                \PY{c+c1}{\PYZsh{} 1. 前向傳播}
                \PY{n}{\PYZus{}}\PY{p}{,} \PY{n}{y\PYZus{}pred} \PY{o}{=} \PY{n+nb+bp}{self}\PY{o}{.}\PY{n}{forward}\PY{p}{(}\PY{n}{x\PYZus{}sample}\PY{p}{)}

                \PY{c+c1}{\PYZsh{} 2. 計算損失 (僅用於記錄,梯度計算在 backward 中)}
                \PY{n}{loss} \PY{o}{=} \PY{n}{square\PYZus{}error\PYZus{}loss}\PY{p}{(}\PY{n}{y\PYZus{}sample}\PY{p}{,} \PY{n}{y\PYZus{}pred}\PY{p}{)}
                \PY{n}{epoch\PYZus{}loss} \PY{o}{+}\PY{o}{=} \PY{n}{loss}

                \PY{c+c1}{\PYZsh{} 3. 反向傳播}
                \PY{n}{dW1}\PY{p}{,} \PY{n}{db1}\PY{p}{,} \PY{n}{dW2}\PY{p}{,} \PY{n}{db2} \PY{o}{=} \PY{n+nb+bp}{self}\PY{o}{.}\PY{n}{backward}\PY{p}{(}\PY{n}{x\PYZus{}sample}\PY{p}{,} \PY{n}{y\PYZus{}sample}\PY{p}{,} \PY{n}{y\PYZus{}pred}\PY{p}{)}  \PY{c+c1}{\PYZsh{} noqa: N806}

                \PY{c+c1}{\PYZsh{} 4. 更新權重 (SGD)}
                \PY{n+nb+bp}{self}\PY{o}{.}\PY{n}{update\PYZus{}weights}\PY{p}{(}\PY{n}{dW1}\PY{p}{,} \PY{n}{db1}\PY{p}{,} \PY{n}{dW2}\PY{p}{,} \PY{n}{db2}\PY{p}{,} \PY{n}{learning\PYZus{}rate}\PY{p}{)}
            \PY{c+c1}{\PYZsh{} \PYZhy{}\PYZhy{}\PYZhy{} SGD 結束 \PYZhy{}\PYZhy{}\PYZhy{}}

            \PY{c+c1}{\PYZsh{} 計算並記錄該 epoch 的平均損失}
            \PY{n}{average\PYZus{}epoch\PYZus{}loss} \PY{o}{=} \PY{n}{epoch\PYZus{}loss} \PY{o}{/} \PY{n}{n\PYZus{}samples}
            \PY{n}{losses}\PY{o}{.}\PY{n}{append}\PY{p}{(}\PY{n}{average\PYZus{}epoch\PYZus{}loss}\PY{p}{)}

            \PY{c+c1}{\PYZsh{} print 進度}
            \PY{k}{if} \PY{p}{(}\PY{n}{epoch} \PY{o}{+} \PY{l+m+mi}{1}\PY{p}{)} \PY{o}{\PYZpc{}} \PY{n}{verbose} \PY{o}{==} \PY{l+m+mi}{0} \PY{o+ow}{or} \PY{n}{epoch} \PY{o}{==} \PY{l+m+mi}{0}\PY{p}{:}
                \PY{n+nb}{print}\PY{p}{(}\PY{l+s+sa}{f}\PY{l+s+s2}{\PYZdq{}}\PY{l+s+s2}{Epoch }\PY{l+s+si}{\PYZob{}}\PY{n}{epoch}\PY{+w}{ }\PY{o}{+}\PY{+w}{ }\PY{l+m+mi}{1}\PY{l+s+si}{\PYZcb{}}\PY{l+s+s2}{/}\PY{l+s+si}{\PYZob{}}\PY{n}{epochs}\PY{l+s+si}{\PYZcb{}}\PY{l+s+s2}{, 平均損失: }\PY{l+s+si}{\PYZob{}}\PY{n}{average\PYZus{}epoch\PYZus{}loss}\PY{l+s+si}{:}\PY{l+s+s2}{.6f}\PY{l+s+si}{\PYZcb{}}\PY{l+s+s2}{\PYZdq{}}\PY{p}{)}

        \PY{n}{end\PYZus{}time} \PY{o}{=} \PY{n}{time}\PY{o}{.}\PY{n}{time}\PY{p}{(}\PY{p}{)}
        \PY{n+nb}{print}\PY{p}{(}\PY{l+s+sa}{f}\PY{l+s+s2}{\PYZdq{}}\PY{l+s+s2}{訓練完成。總耗時: }\PY{l+s+si}{\PYZob{}}\PY{n}{end\PYZus{}time}\PY{+w}{ }\PY{o}{\PYZhy{}}\PY{+w}{ }\PY{n}{start\PYZus{}time}\PY{l+s+si}{:}\PY{l+s+s2}{.2f}\PY{l+s+si}{\PYZcb{}}\PY{l+s+s2}{ 秒}\PY{l+s+s2}{\PYZdq{}}\PY{p}{)}
        \PY{k}{return} \PY{n}{losses}

    \PY{k}{def}\PY{+w}{ }\PY{n+nf}{predict}\PY{p}{(}\PY{n+nb+bp}{self}\PY{p}{,} \PY{n}{x}\PY{p}{)}\PY{p}{:}
\PY{+w}{        }\PY{l+s+sd}{\PYZdq{}\PYZdq{}\PYZdq{}}
\PY{l+s+sd}{        使用訓練好的模型進行預測。}

\PY{l+s+sd}{        Args:}
\PY{l+s+sd}{            x (np.ndarray): 輸入資料。}

\PY{l+s+sd}{        Returns:}
\PY{l+s+sd}{            np.ndarray: 二元預測結果 (0 或 1)。}
\PY{l+s+sd}{        \PYZdq{}\PYZdq{}\PYZdq{}}
        \PY{n}{\PYZus{}}\PY{p}{,} \PY{n}{final\PYZus{}output} \PY{o}{=} \PY{n+nb+bp}{self}\PY{o}{.}\PY{n}{forward}\PY{p}{(}\PY{n}{x}\PY{p}{)}
        \PY{c+c1}{\PYZsh{} 將 Sigmoid 輸出轉換為二元預測 (閾值為 0.5)}
        \PY{n}{predictions} \PY{o}{=} \PY{p}{(}\PY{n}{final\PYZus{}output} \PY{o}{\PYZgt{}} \PY{l+m+mf}{0.5}\PY{p}{)}\PY{o}{.}\PY{n}{astype}\PY{p}{(}\PY{n+nb}{int}\PY{p}{)}
        \PY{k}{return} \PY{n}{predictions}

    \PY{k}{def}\PY{+w}{ }\PY{n+nf}{evaluate}\PY{p}{(}\PY{n+nb+bp}{self}\PY{p}{,} \PY{n}{X\PYZus{}test}\PY{p}{,} \PY{n}{y\PYZus{}test}\PY{p}{)}\PY{p}{:}  \PY{c+c1}{\PYZsh{} noqa: N803}
\PY{+w}{        }\PY{l+s+sd}{\PYZdq{}\PYZdq{}\PYZdq{}}
\PY{l+s+sd}{        在測試集上評估模型性能。}

\PY{l+s+sd}{        Args:}
\PY{l+s+sd}{            X\PYZus{}test (np.ndarray): 測試特徵。}
\PY{l+s+sd}{            y\PYZus{}test (np.ndarray): 測試標籤。}

\PY{l+s+sd}{        Returns:}
\PY{l+s+sd}{            tuple: (accuracy, confusion\PYZus{}matrix)}
\PY{l+s+sd}{                準確率和混淆矩陣。}
\PY{l+s+sd}{        \PYZdq{}\PYZdq{}\PYZdq{}}
        \PY{n}{predictions} \PY{o}{=} \PY{n+nb+bp}{self}\PY{o}{.}\PY{n}{predict}\PY{p}{(}\PY{n}{X\PYZus{}test}\PY{p}{)}
        \PY{n}{accuracy} \PY{o}{=} \PY{n}{np}\PY{o}{.}\PY{n}{mean}\PY{p}{(}\PY{n}{predictions} \PY{o}{==} \PY{n}{y\PYZus{}test}\PY{p}{)}

        \PY{c+c1}{\PYZsh{} 計算 confusion matrix}
        \PY{n}{tp} \PY{o}{=} \PY{n}{np}\PY{o}{.}\PY{n}{sum}\PY{p}{(}\PY{p}{(}\PY{n}{predictions} \PY{o}{==} \PY{l+m+mi}{1}\PY{p}{)} \PY{o}{\PYZam{}} \PY{p}{(}\PY{n}{y\PYZus{}test} \PY{o}{==} \PY{l+m+mi}{1}\PY{p}{)}\PY{p}{)}
        \PY{n}{tn} \PY{o}{=} \PY{n}{np}\PY{o}{.}\PY{n}{sum}\PY{p}{(}\PY{p}{(}\PY{n}{predictions} \PY{o}{==} \PY{l+m+mi}{0}\PY{p}{)} \PY{o}{\PYZam{}} \PY{p}{(}\PY{n}{y\PYZus{}test} \PY{o}{==} \PY{l+m+mi}{0}\PY{p}{)}\PY{p}{)}
        \PY{n}{fp} \PY{o}{=} \PY{n}{np}\PY{o}{.}\PY{n}{sum}\PY{p}{(}\PY{p}{(}\PY{n}{predictions} \PY{o}{==} \PY{l+m+mi}{1}\PY{p}{)} \PY{o}{\PYZam{}} \PY{p}{(}\PY{n}{y\PYZus{}test} \PY{o}{==} \PY{l+m+mi}{0}\PY{p}{)}\PY{p}{)}
        \PY{n}{fn} \PY{o}{=} \PY{n}{np}\PY{o}{.}\PY{n}{sum}\PY{p}{(}\PY{p}{(}\PY{n}{predictions} \PY{o}{==} \PY{l+m+mi}{0}\PY{p}{)} \PY{o}{\PYZam{}} \PY{p}{(}\PY{n}{y\PYZus{}test} \PY{o}{==} \PY{l+m+mi}{1}\PY{p}{)}\PY{p}{)}

        \PY{n}{confusion\PYZus{}matrix} \PY{o}{=} \PY{n}{np}\PY{o}{.}\PY{n}{array}\PY{p}{(}\PY{p}{[}\PY{p}{[}\PY{n}{tn}\PY{p}{,} \PY{n}{fp}\PY{p}{]}\PY{p}{,} \PY{p}{[}\PY{n}{fn}\PY{p}{,} \PY{n}{tp}\PY{p}{]}\PY{p}{]}\PY{p}{)}

        \PY{n+nb}{print}\PY{p}{(}\PY{l+s+s2}{\PYZdq{}}\PY{l+s+se}{\PYZbs{}n}\PY{l+s+s2}{模型評估結果:}\PY{l+s+s2}{\PYZdq{}}\PY{p}{)}
        \PY{n+nb}{print}\PY{p}{(}\PY{l+s+sa}{f}\PY{l+s+s2}{\PYZdq{}}\PY{l+s+s2}{  \PYZhy{} 測試集準確率: }\PY{l+s+si}{\PYZob{}}\PY{n}{accuracy}\PY{l+s+si}{:}\PY{l+s+s2}{.4f}\PY{l+s+si}{\PYZcb{}}\PY{l+s+s2}{\PYZdq{}}\PY{p}{)}
        \PY{n+nb}{print}\PY{p}{(}\PY{l+s+sa}{f}\PY{l+s+s2}{\PYZdq{}}\PY{l+s+s2}{  \PYZhy{} 混淆矩陣:}\PY{l+s+se}{\PYZbs{}n}\PY{l+s+si}{\PYZob{}}\PY{n}{confusion\PYZus{}matrix}\PY{l+s+si}{\PYZcb{}}\PY{l+s+s2}{\PYZdq{}}\PY{p}{)}
        \PY{n+nb}{print}\PY{p}{(}\PY{l+s+sa}{f}\PY{l+s+s2}{\PYZdq{}}\PY{l+s+s2}{    \PYZhy{} True Negatives (TN): }\PY{l+s+si}{\PYZob{}}\PY{n}{tn}\PY{l+s+si}{\PYZcb{}}\PY{l+s+s2}{\PYZdq{}}\PY{p}{)}
        \PY{n+nb}{print}\PY{p}{(}\PY{l+s+sa}{f}\PY{l+s+s2}{\PYZdq{}}\PY{l+s+s2}{    \PYZhy{} False Positives (FP): }\PY{l+s+si}{\PYZob{}}\PY{n}{fp}\PY{l+s+si}{\PYZcb{}}\PY{l+s+s2}{\PYZdq{}}\PY{p}{)}
        \PY{n+nb}{print}\PY{p}{(}\PY{l+s+sa}{f}\PY{l+s+s2}{\PYZdq{}}\PY{l+s+s2}{    \PYZhy{} False Negatives (FN): }\PY{l+s+si}{\PYZob{}}\PY{n}{fn}\PY{l+s+si}{\PYZcb{}}\PY{l+s+s2}{\PYZdq{}}\PY{p}{)}
        \PY{n+nb}{print}\PY{p}{(}\PY{l+s+sa}{f}\PY{l+s+s2}{\PYZdq{}}\PY{l+s+s2}{    \PYZhy{} True Positives (TP): }\PY{l+s+si}{\PYZob{}}\PY{n}{tp}\PY{l+s+si}{\PYZcb{}}\PY{l+s+s2}{\PYZdq{}}\PY{p}{)}

        \PY{k}{return} \PY{n}{accuracy}\PY{p}{,} \PY{n}{confusion\PYZus{}matrix}
\end{Verbatim}
\end{tcolorbox}

    \section{Step 3: Evaluation}\label{step-3-evaluation}

註:Training 的實作已與 Step 2 一起寫在 Class
中,這邊只進行呼叫訓練的操作

    \begin{tcolorbox}[breakable, size=fbox, boxrule=1pt, pad at break*=1mm,colback=cellbackground, colframe=cellborder]
\prompt{In}{incolor}{23}{\boxspacing}
\begin{Verbatim}[commandchars=\\\{\}]
\PY{n}{model} \PY{o}{=} \PY{n}{SimpleANN}\PY{p}{(}\PY{n}{input\PYZus{}size}\PY{o}{=}\PY{n}{INPUT\PYZus{}SIZE}\PY{p}{,}
            \PY{n}{hidden\PYZus{}size}\PY{o}{=}\PY{n}{HIDDEN\PYZus{}SIZE}\PY{p}{,}
            \PY{n}{output\PYZus{}size}\PY{o}{=}\PY{n}{OUTPUT\PYZus{}SIZE}\PY{p}{,}
            \PY{n}{hidden\PYZus{}activation}\PY{o}{=}\PY{n}{HIDDEN\PYZus{}ACTIVATION}\PY{p}{,}
            \PY{n}{seed}\PY{o}{=}\PY{n}{RANDOM\PYZus{}SEED}\PY{p}{)}
\end{Verbatim}
\end{tcolorbox}

    \begin{Verbatim}[commandchars=\\\{\}]
神經網路初始化完成:
  - 輸入層大小: 8
  - 隱藏層大小: 16
  - 輸出層大小: 1
  - 隱藏層激活函數: ActivationFunction.RELU
  - 輸出層激活函數: sigmoid
    \end{Verbatim}

    \begin{tcolorbox}[breakable, size=fbox, boxrule=1pt, pad at break*=1mm,colback=cellbackground, colframe=cellborder]
\prompt{In}{incolor}{24}{\boxspacing}
\begin{Verbatim}[commandchars=\\\{\}]
\PY{c+c1}{\PYZsh{} 訓練模型}
\PY{n}{losses} \PY{o}{=} \PY{n}{model}\PY{o}{.}\PY{n}{train}\PY{p}{(}\PY{n}{X\PYZus{}train}\PY{p}{,} \PY{n}{y\PYZus{}train}\PY{p}{,} \PY{n}{epochs}\PY{o}{=}\PY{n}{EPOCHS}\PY{p}{,} \PY{n}{learning\PYZus{}rate}\PY{o}{=}\PY{n}{LEARNING\PYZus{}RATE}\PY{p}{,} \PY{n}{verbose}\PY{o}{=}\PY{n}{VERBOSE\PYZus{}STEP}\PY{p}{)}
\end{Verbatim}
\end{tcolorbox}

    \begin{Verbatim}[commandchars=\\\{\}]

開始訓練{\ldots}
  - Epochs: 1500
  - Learning Rate: 0.01
  - 訓練樣本數: 614
Epoch 1/1500, 平均損失: 0.120933
Epoch 100/1500, 平均損失: 0.074983
Epoch 200/1500, 平均損失: 0.073965
Epoch 300/1500, 平均損失: 0.072848
Epoch 400/1500, 平均損失: 0.071801
Epoch 500/1500, 平均損失: 0.070995
Epoch 600/1500, 平均損失: 0.070316
Epoch 700/1500, 平均損失: 0.069818
Epoch 800/1500, 平均損失: 0.069477
Epoch 900/1500, 平均損失: 0.069250
Epoch 1000/1500, 平均損失: 0.069071
Epoch 1100/1500, 平均損失: 0.068835
Epoch 1200/1500, 平均損失: 0.068674
Epoch 1300/1500, 平均損失: 0.068566
Epoch 1400/1500, 平均損失: 0.068467
Epoch 1500/1500, 平均損失: 0.068418
訓練完成。總耗時: 55.31 秒
    \end{Verbatim}

    \begin{tcolorbox}[breakable, size=fbox, boxrule=1pt, pad at break*=1mm,colback=cellbackground, colframe=cellborder]
\prompt{In}{incolor}{25}{\boxspacing}
\begin{Verbatim}[commandchars=\\\{\}]
\PY{c+c1}{\PYZsh{} 繪製 loss vs. epoch}
\PY{k+kn}{from}\PY{+w}{ }\PY{n+nn}{matplotlib}\PY{+w}{ }\PY{k+kn}{import} \PY{n}{pyplot} \PY{k}{as} \PY{n}{plt}
\PY{n}{plt}\PY{o}{.}\PY{n}{figure}\PY{p}{(}\PY{n}{figsize}\PY{o}{=}\PY{p}{(}\PY{l+m+mi}{10}\PY{p}{,} \PY{l+m+mi}{6}\PY{p}{)}\PY{p}{)}
\PY{n}{plt}\PY{o}{.}\PY{n}{plot}\PY{p}{(}\PY{n+nb}{range}\PY{p}{(}\PY{l+m+mi}{1}\PY{p}{,} \PY{n}{EPOCHS} \PY{o}{+} \PY{l+m+mi}{1}\PY{p}{)}\PY{p}{,} \PY{n}{losses}\PY{p}{)}
\PY{n}{plt}\PY{o}{.}\PY{n}{xlabel}\PY{p}{(}\PY{l+s+s2}{\PYZdq{}}\PY{l+s+s2}{Epoch}\PY{l+s+s2}{\PYZdq{}}\PY{p}{)}
\PY{n}{plt}\PY{o}{.}\PY{n}{ylabel}\PY{p}{(}\PY{l+s+s2}{\PYZdq{}}\PY{l+s+s2}{Average Square Error Loss}\PY{l+s+s2}{\PYZdq{}}\PY{p}{)}
\PY{n}{plt}\PY{o}{.}\PY{n}{title}\PY{p}{(}\PY{l+s+s2}{\PYZdq{}}\PY{l+s+s2}{Loss Curve}\PY{l+s+s2}{\PYZdq{}}\PY{p}{)}
\PY{n}{plt}\PY{o}{.}\PY{n}{grid}\PY{p}{(}\PY{k+kc}{True}\PY{p}{)}
\PY{n}{plt}\PY{o}{.}\PY{n}{show}\PY{p}{(}\PY{p}{)}
\end{Verbatim}
\end{tcolorbox}

    \begin{center}
    \adjustimage{max size={0.9\linewidth}{0.9\paperheight}}{output_40_0.png}
    \end{center}
    { \hspace*{\fill} \\}
    
    \section{Step 4: Evaluation}\label{step-4-evaluation}

    \begin{tcolorbox}[breakable, size=fbox, boxrule=1pt, pad at break*=1mm,colback=cellbackground, colframe=cellborder]
\prompt{In}{incolor}{26}{\boxspacing}
\begin{Verbatim}[commandchars=\\\{\}]
\PY{n}{accuracy}\PY{p}{,} \PY{n}{confusion\PYZus{}mat} \PY{o}{=} \PY{n}{model}\PY{o}{.}\PY{n}{evaluate}\PY{p}{(}\PY{n}{X\PYZus{}test}\PY{p}{,} \PY{n}{y\PYZus{}test}\PY{p}{)}
\end{Verbatim}
\end{tcolorbox}

    \begin{Verbatim}[commandchars=\\\{\}]

模型評估結果:
  - 測試集準確率: 0.7792
  - 混淆矩陣:
[[83 13]
 [21 37]]
    - True Negatives (TN): 83
    - False Positives (FP): 13
    - False Negatives (FN): 21
    - True Positives (TP): 37
    \end{Verbatim}

    \begin{tcolorbox}[breakable, size=fbox, boxrule=1pt, pad at break*=1mm,colback=cellbackground, colframe=cellborder]
\prompt{In}{incolor}{28}{\boxspacing}
\begin{Verbatim}[commandchars=\\\{\}]
\PY{n+nb}{print}\PY{p}{(}\PY{l+s+s2}{\PYZdq{}}\PY{l+s+s2}{最終 Weights:}\PY{l+s+s2}{\PYZdq{}}\PY{p}{)}
\PY{n+nb}{print}\PY{p}{(}\PY{l+s+s2}{\PYZdq{}}\PY{l+s+s2}{W1 (Input \PYZhy{}\PYZgt{} Hidden):}\PY{l+s+se}{\PYZbs{}n}\PY{l+s+s2}{\PYZdq{}}\PY{p}{,} \PY{n}{model}\PY{o}{.}\PY{n}{W1}\PY{p}{)}
\PY{n+nb}{print}\PY{p}{(}\PY{l+s+s2}{\PYZdq{}}\PY{l+s+s2}{b1 (Hidden Bias):}\PY{l+s+se}{\PYZbs{}n}\PY{l+s+s2}{\PYZdq{}}\PY{p}{,} \PY{n}{model}\PY{o}{.}\PY{n}{b1}\PY{p}{)}
\PY{n+nb}{print}\PY{p}{(}\PY{l+s+s2}{\PYZdq{}}\PY{l+s+s2}{W2 (Hidden \PYZhy{}\PYZgt{} Output):}\PY{l+s+se}{\PYZbs{}n}\PY{l+s+s2}{\PYZdq{}}\PY{p}{,} \PY{n}{model}\PY{o}{.}\PY{n}{W2}\PY{p}{)}
\PY{n+nb}{print}\PY{p}{(}\PY{l+s+s2}{\PYZdq{}}\PY{l+s+s2}{b2 (Output Bias):}\PY{l+s+se}{\PYZbs{}n}\PY{l+s+s2}{\PYZdq{}}\PY{p}{,} \PY{n}{model}\PY{o}{.}\PY{n}{b2}\PY{p}{)}
\end{Verbatim}
\end{tcolorbox}

    \begin{Verbatim}[commandchars=\\\{\}]
最終 Weights:
W1 (Input -> Hidden):
 [[-0.25741404  0.60069728  0.21440997 -0.42379298  0.086695   -0.34400548
  -1.74363813  0.17807147  0.09287509  0.59019976 -0.47941551  0.90389712
   0.84638356 -0.28766089 -1.03633838 -0.31627978]
 [-0.19654201 -0.00931814 -0.72662162 -0.8675841   0.12609567 -0.36050614
   0.66263342 -0.39880344 -0.05107465  0.4876268  -0.30032622  0.4195509
   0.45504173 -0.45354959  0.91121065 -0.33806347]
 [-0.77157299  0.46021188  0.22083237  0.81944459 -0.27301264 -0.40232789
   0.01511544  0.14796978 -0.39820967  0.00505847 -0.46561148  0.3161549
  -0.14036066  0.16252228 -0.24637666  0.02033175]
 [-0.42263687 -0.37219503  1.4651971   0.35595754  0.52014502  0.39482735
   0.24193511  0.40418033 -0.41158444  0.23756619 -0.45477271  0.38187806
   0.51132434 -0.22865097  0.31337502 -0.1417006 ]
 [-2.20543495 -0.21231267 -1.04403609  0.13609958 -0.5709749   0.48688694
   0.25049193 -0.23659268 -0.49484632 -0.57550863  0.20685734 -0.78133004
  -0.74563749 -0.42595535 -0.61583613 -0.38313694]
 [ 0.44622995 -0.02322995 -0.99596078 -0.13591232 -0.24688819 -0.17481668
   0.29061237 -0.11032765  0.36085847  0.47159579 -0.38040575  0.70642686
   0.81028029  0.0612772   0.31528508  0.00660041]
 [-0.46068974 -0.05145953 -1.7713426  -1.10807808 -0.51437547  0.13641041
  -0.82000613 -0.15779313  0.44977067 -0.25110125 -0.08961708  0.25187121
  -0.22991178 -0.42302009 -0.22406465 -0.34503545]
 [ 2.81072374  0.38853501  0.06057336 -1.4000801   0.48828698 -0.31342994
   1.05625322 -0.21815191  0.27407022  0.91098411 -0.18199653  0.2845261
   0.41966169 -0.07289221  0.58219181  0.35547229]]
b1 (Hidden Bias):
 [[-0.00242864 -0.06474897  0.74279499  1.16425757 -0.13492469  0.
  -0.06172456  0.36641835  0.01062365 -0.06341467  0.         -0.24824213
  -0.02894696  0.         -0.0183945   0.00653713]]
W2 (Hidden -> Output):
 [[-3.34886373]
 [ 0.52054996]
 [-2.33908623]
 [-1.76858693]
 [-0.74581054]
 [-0.19294193]
 [ 1.93645543]
 [-0.32322734]
 [-0.11351547]
 [ 0.8865093 ]
 [-0.16203218]
 [ 1.00650066]
 [ 1.12127219]
 [-0.29492658]
 [ 1.3135932 ]
 [-0.24667669]]
b2 (Output Bias):
 [[-1.14900166]]
    \end{Verbatim}

    \section{Conclusion}\label{conclusion}

本筆記本成功地使用 NumPy 從零開始建立並訓練了一個簡單的 ANN 模型,用於
Pima Indians Diabetes 資料集的二元分類任務。

訓練過程中的損失曲線顯示,模型的平方誤差損失隨著訓練輪數的增加而穩定下降,最終趨於平穩
。

在測試集上的評估結果如下: * \textbf{準確率 (Accuracy)}: 0.7792
(77.92\%) * \textbf{Confusion Matrix}: * True Negatives (TN): 83 * False
Positives (FP): 13 * False Negatives (FN): 21 * True Positives (TP): 37

這些結果表明,這個從頭實作的簡單 ANN
模型在糖尿病預測任務上達到了一定的效能。最終訓練完成的模型權重也已記錄下來。


    % Add a bibliography block to the postdoc
    
    
    
\end{document}
