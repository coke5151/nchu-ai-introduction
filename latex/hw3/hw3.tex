\documentclass[11pt]{article}

    \usepackage[breakable]{tcolorbox}
    \usepackage{parskip} % Stop auto-indenting (to mimic markdown behaviour)
    

    % Basic figure setup, for now with no caption control since it's done
    % automatically by Pandoc (which extracts ![](path) syntax from Markdown).
    \usepackage{graphicx}
    % Keep aspect ratio if custom image width or height is specified
    \setkeys{Gin}{keepaspectratio}
    % Maintain compatibility with old templates. Remove in nbconvert 6.0
    \let\Oldincludegraphics\includegraphics
    % Ensure that by default, figures have no caption (until we provide a
    % proper Figure object with a Caption API and a way to capture that
    % in the conversion process - todo).
    \usepackage{caption}
    \DeclareCaptionFormat{nocaption}{}
    \captionsetup{format=nocaption,aboveskip=0pt,belowskip=0pt}

    \usepackage{float}
    \floatplacement{figure}{H} % forces figures to be placed at the correct location
    \usepackage{xcolor} % Allow colors to be defined
    \usepackage{enumerate} % Needed for markdown enumerations to work
    \usepackage{geometry} % Used to adjust the document margins
    \usepackage{amsmath} % Equations
    \usepackage{amssymb} % Equations
    \usepackage{textcomp} % defines textquotesingle
    % Hack from http://tex.stackexchange.com/a/47451/13684:
    \AtBeginDocument{%
        \def\PYZsq{\textquotesingle}% Upright quotes in Pygmentized code
    }
    \usepackage{upquote} % Upright quotes for verbatim code
    \usepackage{eurosym} % defines \euro

    \usepackage{iftex}
    \ifPDFTeX
        \usepackage[T1]{fontenc}
        \IfFileExists{alphabeta.sty}{
              \usepackage{alphabeta}
          }{
              \usepackage[mathletters]{ucs}
              \usepackage[utf8x]{inputenc}
          }
    \else
        \usepackage{fontspec}
        \usepackage{unicode-math}
    \fi

    \usepackage{fancyvrb} % verbatim replacement that allows latex
    \usepackage{grffile} % extends the file name processing of package graphics
                         % to support a larger range
    \makeatletter % fix for old versions of grffile with XeLaTeX
    \@ifpackagelater{grffile}{2019/11/01}
    {
      % Do nothing on new versions
    }
    {
      \def\Gread@@xetex#1{%
        \IfFileExists{"\Gin@base".bb}%
        {\Gread@eps{\Gin@base.bb}}%
        {\Gread@@xetex@aux#1}%
      }
    }
    \makeatother
    \usepackage[Export]{adjustbox} % Used to constrain images to a maximum size
    \adjustboxset{max size={0.9\linewidth}{0.9\paperheight}}

    % The hyperref package gives us a pdf with properly built
    % internal navigation ('pdf bookmarks' for the table of contents,
    % internal cross-reference links, web links for URLs, etc.)
    \usepackage{hyperref}
    % The default LaTeX title has an obnoxious amount of whitespace. By default,
    % titling removes some of it. It also provides customization options.
    \usepackage{titling}
    \usepackage{longtable} % longtable support required by pandoc >1.10
    \usepackage{booktabs}  % table support for pandoc > 1.12.2
    \usepackage{array}     % table support for pandoc >= 2.11.3
    \usepackage{calc}      % table minipage width calculation for pandoc >= 2.11.1
    \usepackage[inline]{enumitem} % IRkernel/repr support (it uses the enumerate* environment)
    \usepackage[normalem]{ulem} % ulem is needed to support strikethroughs (\sout)
                                % normalem makes italics be italics, not underlines
    \usepackage{soul}      % strikethrough (\st) support for pandoc >= 3.0.0
    \usepackage{mathrsfs}
    

    
    % Colors for the hyperref package
    \definecolor{urlcolor}{rgb}{0,.145,.698}
    \definecolor{linkcolor}{rgb}{.71,0.21,0.01}
    \definecolor{citecolor}{rgb}{.12,.54,.11}

    % ANSI colors
    \definecolor{ansi-black}{HTML}{3E424D}
    \definecolor{ansi-black-intense}{HTML}{282C36}
    \definecolor{ansi-red}{HTML}{E75C58}
    \definecolor{ansi-red-intense}{HTML}{B22B31}
    \definecolor{ansi-green}{HTML}{00A250}
    \definecolor{ansi-green-intense}{HTML}{007427}
    \definecolor{ansi-yellow}{HTML}{DDB62B}
    \definecolor{ansi-yellow-intense}{HTML}{B27D12}
    \definecolor{ansi-blue}{HTML}{208FFB}
    \definecolor{ansi-blue-intense}{HTML}{0065CA}
    \definecolor{ansi-magenta}{HTML}{D160C4}
    \definecolor{ansi-magenta-intense}{HTML}{A03196}
    \definecolor{ansi-cyan}{HTML}{60C6C8}
    \definecolor{ansi-cyan-intense}{HTML}{258F8F}
    \definecolor{ansi-white}{HTML}{C5C1B4}
    \definecolor{ansi-white-intense}{HTML}{A1A6B2}
    \definecolor{ansi-default-inverse-fg}{HTML}{FFFFFF}
    \definecolor{ansi-default-inverse-bg}{HTML}{000000}

    % common color for the border for error outputs.
    \definecolor{outerrorbackground}{HTML}{FFDFDF}

    % commands and environments needed by pandoc snippets
    % extracted from the output of `pandoc -s`
    \providecommand{\tightlist}{%
      \setlength{\itemsep}{0pt}\setlength{\parskip}{0pt}}
    \DefineVerbatimEnvironment{Highlighting}{Verbatim}{commandchars=\\\{\}}
    % Add ',fontsize=\small' for more characters per line
    \newenvironment{Shaded}{}{}
    \newcommand{\KeywordTok}[1]{\textcolor[rgb]{0.00,0.44,0.13}{\textbf{{#1}}}}
    \newcommand{\DataTypeTok}[1]{\textcolor[rgb]{0.56,0.13,0.00}{{#1}}}
    \newcommand{\DecValTok}[1]{\textcolor[rgb]{0.25,0.63,0.44}{{#1}}}
    \newcommand{\BaseNTok}[1]{\textcolor[rgb]{0.25,0.63,0.44}{{#1}}}
    \newcommand{\FloatTok}[1]{\textcolor[rgb]{0.25,0.63,0.44}{{#1}}}
    \newcommand{\CharTok}[1]{\textcolor[rgb]{0.25,0.44,0.63}{{#1}}}
    \newcommand{\StringTok}[1]{\textcolor[rgb]{0.25,0.44,0.63}{{#1}}}
    \newcommand{\CommentTok}[1]{\textcolor[rgb]{0.38,0.63,0.69}{\textit{{#1}}}}
    \newcommand{\OtherTok}[1]{\textcolor[rgb]{0.00,0.44,0.13}{{#1}}}
    \newcommand{\AlertTok}[1]{\textcolor[rgb]{1.00,0.00,0.00}{\textbf{{#1}}}}
    \newcommand{\FunctionTok}[1]{\textcolor[rgb]{0.02,0.16,0.49}{{#1}}}
    \newcommand{\RegionMarkerTok}[1]{{#1}}
    \newcommand{\ErrorTok}[1]{\textcolor[rgb]{1.00,0.00,0.00}{\textbf{{#1}}}}
    \newcommand{\NormalTok}[1]{{#1}}

    % Additional commands for more recent versions of Pandoc
    \newcommand{\ConstantTok}[1]{\textcolor[rgb]{0.53,0.00,0.00}{{#1}}}
    \newcommand{\SpecialCharTok}[1]{\textcolor[rgb]{0.25,0.44,0.63}{{#1}}}
    \newcommand{\VerbatimStringTok}[1]{\textcolor[rgb]{0.25,0.44,0.63}{{#1}}}
    \newcommand{\SpecialStringTok}[1]{\textcolor[rgb]{0.73,0.40,0.53}{{#1}}}
    \newcommand{\ImportTok}[1]{{#1}}
    \newcommand{\DocumentationTok}[1]{\textcolor[rgb]{0.73,0.13,0.13}{\textit{{#1}}}}
    \newcommand{\AnnotationTok}[1]{\textcolor[rgb]{0.38,0.63,0.69}{\textbf{\textit{{#1}}}}}
    \newcommand{\CommentVarTok}[1]{\textcolor[rgb]{0.38,0.63,0.69}{\textbf{\textit{{#1}}}}}
    \newcommand{\VariableTok}[1]{\textcolor[rgb]{0.10,0.09,0.49}{{#1}}}
    \newcommand{\ControlFlowTok}[1]{\textcolor[rgb]{0.00,0.44,0.13}{\textbf{{#1}}}}
    \newcommand{\OperatorTok}[1]{\textcolor[rgb]{0.40,0.40,0.40}{{#1}}}
    \newcommand{\BuiltInTok}[1]{{#1}}
    \newcommand{\ExtensionTok}[1]{{#1}}
    \newcommand{\PreprocessorTok}[1]{\textcolor[rgb]{0.74,0.48,0.00}{{#1}}}
    \newcommand{\AttributeTok}[1]{\textcolor[rgb]{0.49,0.56,0.16}{{#1}}}
    \newcommand{\InformationTok}[1]{\textcolor[rgb]{0.38,0.63,0.69}{\textbf{\textit{{#1}}}}}
    \newcommand{\WarningTok}[1]{\textcolor[rgb]{0.38,0.63,0.69}{\textbf{\textit{{#1}}}}}
    \makeatletter
    \newsavebox\pandoc@box
    \newcommand*\pandocbounded[1]{%
      \sbox\pandoc@box{#1}%
      % scaling factors for width and height
      \Gscale@div\@tempa\textheight{\dimexpr\ht\pandoc@box+\dp\pandoc@box\relax}%
      \Gscale@div\@tempb\linewidth{\wd\pandoc@box}%
      % select the smaller of both
      \ifdim\@tempb\p@<\@tempa\p@
        \let\@tempa\@tempb
      \fi
      % scaling accordingly (\@tempa < 1)
      \ifdim\@tempa\p@<\p@
        \scalebox{\@tempa}{\usebox\pandoc@box}%
      % scaling not needed, use as it is
      \else
        \usebox{\pandoc@box}%
      \fi
    }
    \makeatother

    % Define a nice break command that doesn't care if a line doesn't already
    % exist.
    \def\br{\hspace*{\fill} \\* }
    % Math Jax compatibility definitions
    \def\gt{>}
    \def\lt{<}
    \let\Oldtex\TeX
    \let\Oldlatex\LaTeX
    \renewcommand{\TeX}{\textrm{\Oldtex}}
    \renewcommand{\LaTeX}{\textrm{\Oldlatex}}
    % Document parameters
    % Document title
    \title{hw3}
    
    
    
    
    
    
    
% Pygments definitions
\makeatletter
\def\PY@reset{\let\PY@it=\relax \let\PY@bf=\relax%
    \let\PY@ul=\relax \let\PY@tc=\relax%
    \let\PY@bc=\relax \let\PY@ff=\relax}
\def\PY@tok#1{\csname PY@tok@#1\endcsname}
\def\PY@toks#1+{\ifx\relax#1\empty\else%
    \PY@tok{#1}\expandafter\PY@toks\fi}
\def\PY@do#1{\PY@bc{\PY@tc{\PY@ul{%
    \PY@it{\PY@bf{\PY@ff{#1}}}}}}}
\def\PY#1#2{\PY@reset\PY@toks#1+\relax+\PY@do{#2}}

\@namedef{PY@tok@w}{\def\PY@tc##1{\textcolor[rgb]{0.73,0.73,0.73}{##1}}}
\@namedef{PY@tok@c}{\let\PY@it=\textit\def\PY@tc##1{\textcolor[rgb]{0.24,0.48,0.48}{##1}}}
\@namedef{PY@tok@cp}{\def\PY@tc##1{\textcolor[rgb]{0.61,0.40,0.00}{##1}}}
\@namedef{PY@tok@k}{\let\PY@bf=\textbf\def\PY@tc##1{\textcolor[rgb]{0.00,0.50,0.00}{##1}}}
\@namedef{PY@tok@kp}{\def\PY@tc##1{\textcolor[rgb]{0.00,0.50,0.00}{##1}}}
\@namedef{PY@tok@kt}{\def\PY@tc##1{\textcolor[rgb]{0.69,0.00,0.25}{##1}}}
\@namedef{PY@tok@o}{\def\PY@tc##1{\textcolor[rgb]{0.40,0.40,0.40}{##1}}}
\@namedef{PY@tok@ow}{\let\PY@bf=\textbf\def\PY@tc##1{\textcolor[rgb]{0.67,0.13,1.00}{##1}}}
\@namedef{PY@tok@nb}{\def\PY@tc##1{\textcolor[rgb]{0.00,0.50,0.00}{##1}}}
\@namedef{PY@tok@nf}{\def\PY@tc##1{\textcolor[rgb]{0.00,0.00,1.00}{##1}}}
\@namedef{PY@tok@nc}{\let\PY@bf=\textbf\def\PY@tc##1{\textcolor[rgb]{0.00,0.00,1.00}{##1}}}
\@namedef{PY@tok@nn}{\let\PY@bf=\textbf\def\PY@tc##1{\textcolor[rgb]{0.00,0.00,1.00}{##1}}}
\@namedef{PY@tok@ne}{\let\PY@bf=\textbf\def\PY@tc##1{\textcolor[rgb]{0.80,0.25,0.22}{##1}}}
\@namedef{PY@tok@nv}{\def\PY@tc##1{\textcolor[rgb]{0.10,0.09,0.49}{##1}}}
\@namedef{PY@tok@no}{\def\PY@tc##1{\textcolor[rgb]{0.53,0.00,0.00}{##1}}}
\@namedef{PY@tok@nl}{\def\PY@tc##1{\textcolor[rgb]{0.46,0.46,0.00}{##1}}}
\@namedef{PY@tok@ni}{\let\PY@bf=\textbf\def\PY@tc##1{\textcolor[rgb]{0.44,0.44,0.44}{##1}}}
\@namedef{PY@tok@na}{\def\PY@tc##1{\textcolor[rgb]{0.41,0.47,0.13}{##1}}}
\@namedef{PY@tok@nt}{\let\PY@bf=\textbf\def\PY@tc##1{\textcolor[rgb]{0.00,0.50,0.00}{##1}}}
\@namedef{PY@tok@nd}{\def\PY@tc##1{\textcolor[rgb]{0.67,0.13,1.00}{##1}}}
\@namedef{PY@tok@s}{\def\PY@tc##1{\textcolor[rgb]{0.73,0.13,0.13}{##1}}}
\@namedef{PY@tok@sd}{\let\PY@it=\textit\def\PY@tc##1{\textcolor[rgb]{0.73,0.13,0.13}{##1}}}
\@namedef{PY@tok@si}{\let\PY@bf=\textbf\def\PY@tc##1{\textcolor[rgb]{0.64,0.35,0.47}{##1}}}
\@namedef{PY@tok@se}{\let\PY@bf=\textbf\def\PY@tc##1{\textcolor[rgb]{0.67,0.36,0.12}{##1}}}
\@namedef{PY@tok@sr}{\def\PY@tc##1{\textcolor[rgb]{0.64,0.35,0.47}{##1}}}
\@namedef{PY@tok@ss}{\def\PY@tc##1{\textcolor[rgb]{0.10,0.09,0.49}{##1}}}
\@namedef{PY@tok@sx}{\def\PY@tc##1{\textcolor[rgb]{0.00,0.50,0.00}{##1}}}
\@namedef{PY@tok@m}{\def\PY@tc##1{\textcolor[rgb]{0.40,0.40,0.40}{##1}}}
\@namedef{PY@tok@gh}{\let\PY@bf=\textbf\def\PY@tc##1{\textcolor[rgb]{0.00,0.00,0.50}{##1}}}
\@namedef{PY@tok@gu}{\let\PY@bf=\textbf\def\PY@tc##1{\textcolor[rgb]{0.50,0.00,0.50}{##1}}}
\@namedef{PY@tok@gd}{\def\PY@tc##1{\textcolor[rgb]{0.63,0.00,0.00}{##1}}}
\@namedef{PY@tok@gi}{\def\PY@tc##1{\textcolor[rgb]{0.00,0.52,0.00}{##1}}}
\@namedef{PY@tok@gr}{\def\PY@tc##1{\textcolor[rgb]{0.89,0.00,0.00}{##1}}}
\@namedef{PY@tok@ge}{\let\PY@it=\textit}
\@namedef{PY@tok@gs}{\let\PY@bf=\textbf}
\@namedef{PY@tok@ges}{\let\PY@bf=\textbf\let\PY@it=\textit}
\@namedef{PY@tok@gp}{\let\PY@bf=\textbf\def\PY@tc##1{\textcolor[rgb]{0.00,0.00,0.50}{##1}}}
\@namedef{PY@tok@go}{\def\PY@tc##1{\textcolor[rgb]{0.44,0.44,0.44}{##1}}}
\@namedef{PY@tok@gt}{\def\PY@tc##1{\textcolor[rgb]{0.00,0.27,0.87}{##1}}}
\@namedef{PY@tok@err}{\def\PY@bc##1{{\setlength{\fboxsep}{\string -\fboxrule}\fcolorbox[rgb]{1.00,0.00,0.00}{1,1,1}{\strut ##1}}}}
\@namedef{PY@tok@kc}{\let\PY@bf=\textbf\def\PY@tc##1{\textcolor[rgb]{0.00,0.50,0.00}{##1}}}
\@namedef{PY@tok@kd}{\let\PY@bf=\textbf\def\PY@tc##1{\textcolor[rgb]{0.00,0.50,0.00}{##1}}}
\@namedef{PY@tok@kn}{\let\PY@bf=\textbf\def\PY@tc##1{\textcolor[rgb]{0.00,0.50,0.00}{##1}}}
\@namedef{PY@tok@kr}{\let\PY@bf=\textbf\def\PY@tc##1{\textcolor[rgb]{0.00,0.50,0.00}{##1}}}
\@namedef{PY@tok@bp}{\def\PY@tc##1{\textcolor[rgb]{0.00,0.50,0.00}{##1}}}
\@namedef{PY@tok@fm}{\def\PY@tc##1{\textcolor[rgb]{0.00,0.00,1.00}{##1}}}
\@namedef{PY@tok@vc}{\def\PY@tc##1{\textcolor[rgb]{0.10,0.09,0.49}{##1}}}
\@namedef{PY@tok@vg}{\def\PY@tc##1{\textcolor[rgb]{0.10,0.09,0.49}{##1}}}
\@namedef{PY@tok@vi}{\def\PY@tc##1{\textcolor[rgb]{0.10,0.09,0.49}{##1}}}
\@namedef{PY@tok@vm}{\def\PY@tc##1{\textcolor[rgb]{0.10,0.09,0.49}{##1}}}
\@namedef{PY@tok@sa}{\def\PY@tc##1{\textcolor[rgb]{0.73,0.13,0.13}{##1}}}
\@namedef{PY@tok@sb}{\def\PY@tc##1{\textcolor[rgb]{0.73,0.13,0.13}{##1}}}
\@namedef{PY@tok@sc}{\def\PY@tc##1{\textcolor[rgb]{0.73,0.13,0.13}{##1}}}
\@namedef{PY@tok@dl}{\def\PY@tc##1{\textcolor[rgb]{0.73,0.13,0.13}{##1}}}
\@namedef{PY@tok@s2}{\def\PY@tc##1{\textcolor[rgb]{0.73,0.13,0.13}{##1}}}
\@namedef{PY@tok@sh}{\def\PY@tc##1{\textcolor[rgb]{0.73,0.13,0.13}{##1}}}
\@namedef{PY@tok@s1}{\def\PY@tc##1{\textcolor[rgb]{0.73,0.13,0.13}{##1}}}
\@namedef{PY@tok@mb}{\def\PY@tc##1{\textcolor[rgb]{0.40,0.40,0.40}{##1}}}
\@namedef{PY@tok@mf}{\def\PY@tc##1{\textcolor[rgb]{0.40,0.40,0.40}{##1}}}
\@namedef{PY@tok@mh}{\def\PY@tc##1{\textcolor[rgb]{0.40,0.40,0.40}{##1}}}
\@namedef{PY@tok@mi}{\def\PY@tc##1{\textcolor[rgb]{0.40,0.40,0.40}{##1}}}
\@namedef{PY@tok@il}{\def\PY@tc##1{\textcolor[rgb]{0.40,0.40,0.40}{##1}}}
\@namedef{PY@tok@mo}{\def\PY@tc##1{\textcolor[rgb]{0.40,0.40,0.40}{##1}}}
\@namedef{PY@tok@ch}{\let\PY@it=\textit\def\PY@tc##1{\textcolor[rgb]{0.24,0.48,0.48}{##1}}}
\@namedef{PY@tok@cm}{\let\PY@it=\textit\def\PY@tc##1{\textcolor[rgb]{0.24,0.48,0.48}{##1}}}
\@namedef{PY@tok@cpf}{\let\PY@it=\textit\def\PY@tc##1{\textcolor[rgb]{0.24,0.48,0.48}{##1}}}
\@namedef{PY@tok@c1}{\let\PY@it=\textit\def\PY@tc##1{\textcolor[rgb]{0.24,0.48,0.48}{##1}}}
\@namedef{PY@tok@cs}{\let\PY@it=\textit\def\PY@tc##1{\textcolor[rgb]{0.24,0.48,0.48}{##1}}}

\def\PYZbs{\char`\\}
\def\PYZus{\char`\_}
\def\PYZob{\char`\{}
\def\PYZcb{\char`\}}
\def\PYZca{\char`\^}
\def\PYZam{\char`\&}
\def\PYZlt{\char`\<}
\def\PYZgt{\char`\>}
\def\PYZsh{\char`\#}
\def\PYZpc{\char`\%}
\def\PYZdl{\char`\$}
\def\PYZhy{\char`\-}
\def\PYZsq{\char`\'}
\def\PYZdq{\char`\"}
\def\PYZti{\char`\~}
% for compatibility with earlier versions
\def\PYZat{@}
\def\PYZlb{[}
\def\PYZrb{]}
\makeatother


    % For linebreaks inside Verbatim environment from package fancyvrb.
    \makeatletter
        \newbox\Wrappedcontinuationbox
        \newbox\Wrappedvisiblespacebox
        \newcommand*\Wrappedvisiblespace {\textcolor{red}{\textvisiblespace}}
        \newcommand*\Wrappedcontinuationsymbol {\textcolor{red}{\llap{\tiny$\m@th\hookrightarrow$}}}
        \newcommand*\Wrappedcontinuationindent {3ex }
        \newcommand*\Wrappedafterbreak {\kern\Wrappedcontinuationindent\copy\Wrappedcontinuationbox}
        % Take advantage of the already applied Pygments mark-up to insert
        % potential linebreaks for TeX processing.
        %        {, <, #, %, $, ' and ": go to next line.
        %        _, }, ^, &, >, - and ~: stay at end of broken line.
        % Use of \textquotesingle for straight quote.
        \newcommand*\Wrappedbreaksatspecials {%
            \def\PYGZus{\discretionary{\char`\_}{\Wrappedafterbreak}{\char`\_}}%
            \def\PYGZob{\discretionary{}{\Wrappedafterbreak\char`\{}{\char`\{}}%
            \def\PYGZcb{\discretionary{\char`\}}{\Wrappedafterbreak}{\char`\}}}%
            \def\PYGZca{\discretionary{\char`\^}{\Wrappedafterbreak}{\char`\^}}%
            \def\PYGZam{\discretionary{\char`\&}{\Wrappedafterbreak}{\char`\&}}%
            \def\PYGZlt{\discretionary{}{\Wrappedafterbreak\char`\<}{\char`\<}}%
            \def\PYGZgt{\discretionary{\char`\>}{\Wrappedafterbreak}{\char`\>}}%
            \def\PYGZsh{\discretionary{}{\Wrappedafterbreak\char`\#}{\char`\#}}%
            \def\PYGZpc{\discretionary{}{\Wrappedafterbreak\char`\%}{\char`\%}}%
            \def\PYGZdl{\discretionary{}{\Wrappedafterbreak\char`\$}{\char`\$}}%
            \def\PYGZhy{\discretionary{\char`\-}{\Wrappedafterbreak}{\char`\-}}%
            \def\PYGZsq{\discretionary{}{\Wrappedafterbreak\textquotesingle}{\textquotesingle}}%
            \def\PYGZdq{\discretionary{}{\Wrappedafterbreak\char`\"}{\char`\"}}%
            \def\PYGZti{\discretionary{\char`\~}{\Wrappedafterbreak}{\char`\~}}%
        }
        % Some characters . , ; ? ! / are not pygmentized.
        % This macro makes them "active" and they will insert potential linebreaks
        \newcommand*\Wrappedbreaksatpunct {%
            \lccode`\~`\.\lowercase{\def~}{\discretionary{\hbox{\char`\.}}{\Wrappedafterbreak}{\hbox{\char`\.}}}%
            \lccode`\~`\,\lowercase{\def~}{\discretionary{\hbox{\char`\,}}{\Wrappedafterbreak}{\hbox{\char`\,}}}%
            \lccode`\~`\;\lowercase{\def~}{\discretionary{\hbox{\char`\;}}{\Wrappedafterbreak}{\hbox{\char`\;}}}%
            \lccode`\~`\:\lowercase{\def~}{\discretionary{\hbox{\char`\:}}{\Wrappedafterbreak}{\hbox{\char`\:}}}%
            \lccode`\~`\?\lowercase{\def~}{\discretionary{\hbox{\char`\?}}{\Wrappedafterbreak}{\hbox{\char`\?}}}%
            \lccode`\~`\!\lowercase{\def~}{\discretionary{\hbox{\char`\!}}{\Wrappedafterbreak}{\hbox{\char`\!}}}%
            \lccode`\~`\/\lowercase{\def~}{\discretionary{\hbox{\char`\/}}{\Wrappedafterbreak}{\hbox{\char`\/}}}%
            \catcode`\.\active
            \catcode`\,\active
            \catcode`\;\active
            \catcode`\:\active
            \catcode`\?\active
            \catcode`\!\active
            \catcode`\/\active
            \lccode`\~`\~
        }
    \makeatother

    \let\OriginalVerbatim=\Verbatim
    \makeatletter
    \renewcommand{\Verbatim}[1][1]{%
        %\parskip\z@skip
        \sbox\Wrappedcontinuationbox {\Wrappedcontinuationsymbol}%
        \sbox\Wrappedvisiblespacebox {\FV@SetupFont\Wrappedvisiblespace}%
        \def\FancyVerbFormatLine ##1{\hsize\linewidth
            \vtop{\raggedright\hyphenpenalty\z@\exhyphenpenalty\z@
                \doublehyphendemerits\z@\finalhyphendemerits\z@
                \strut ##1\strut}%
        }%
        % If the linebreak is at a space, the latter will be displayed as visible
        % space at end of first line, and a continuation symbol starts next line.
        % Stretch/shrink are however usually zero for typewriter font.
        \def\FV@Space {%
            \nobreak\hskip\z@ plus\fontdimen3\font minus\fontdimen4\font
            \discretionary{\copy\Wrappedvisiblespacebox}{\Wrappedafterbreak}
            {\kern\fontdimen2\font}%
        }%

        % Allow breaks at special characters using \PYG... macros.
        \Wrappedbreaksatspecials
        % Breaks at punctuation characters . , ; ? ! and / need catcode=\active
        \OriginalVerbatim[#1,codes*=\Wrappedbreaksatpunct]%
    }
    \makeatother

    % Exact colors from NB
    \definecolor{incolor}{HTML}{303F9F}
    \definecolor{outcolor}{HTML}{D84315}
    \definecolor{cellborder}{HTML}{CFCFCF}
    \definecolor{cellbackground}{HTML}{F7F7F7}

    % prompt
    \makeatletter
    \newcommand{\boxspacing}{\kern\kvtcb@left@rule\kern\kvtcb@boxsep}
    \makeatother
    \newcommand{\prompt}[4]{
        {\ttfamily\llap{{\color{#2}[#3]:\hspace{3pt}#4}}\vspace{-\baselineskip}}
    }
    

    
    % Prevent overflowing lines due to hard-to-break entities
    \sloppy
    % Setup hyperref package
    \hypersetup{
      breaklinks=true,  % so long urls are correctly broken across lines
      colorlinks=true,
      urlcolor=urlcolor,
      linkcolor=linkcolor,
      citecolor=citecolor,
      }
    % Slightly bigger margins than the latex defaults
    
    \geometry{verbose,tmargin=1in,bmargin=1in,lmargin=1in,rmargin=1in}
    
    

\begin{document}
    
    \maketitle
    
    

    
    \section{Summary}\label{summary}

本報告旨在實作一個雙層神經網路,並應用於 MNIST 手寫數字資料集的分類任務
。為符合要求,整個神經網路模型,包括其訓練過程,均使用 Python 和 NumPy
函式庫從零開始搭建,不依賴任何現有的機器學習框架。

實作流程始於資料的獲取與前處理。首先,由於原始 MNIST
資料集網站的下載問題,我改由穩定的 GitHub mirror 來源獲取資料
。接著,對資料進行標準化前處理,包含將 28×28 的二維圖片資料拉平為 784
維的特徵向量 ,將像素值 normalize 至 0 到 1 的區間 ,並對分類標籤進行
One-Hot Encode,以便於後續損失函數的計算。

神經網路的核心架構包含一個使用 ReLU 作為激活函數的隱藏層,以及一個使用
Softmax 函數進行多類別機率輸出的輸出層。這份 ipynb
完整地實作了模型訓練的關鍵演算法,包括:

\begin{itemize}
\tightlist
\item
  前向傳播 (Forward Propagation):根據輸入資料與當前參數計算預測輸出 。
\item
  損失函數 (Loss Function):採用 Cross-Entropy Loss
  來衡量預測與正解間的差異。
\item
  反向傳播 (Backward Propagation):計算損失函數對各層參數的梯度。
\item
  參數更新 (Update
  Parameters):根據計算出的梯度,透過梯度下降法來迭代優化模型權重。
\end{itemize}

整個訓練過程採用小批次梯度下降 (Mini-batch Gradient Descent)
的方式,並在每個訓練世代 (epoch)
開始前對資料進行隨機洗牌,以提升模型的泛化能力。

    \section{Step 0:
下載資料集}\label{step-0-ux4e0bux8f09ux8cc7ux6599ux96c6}

    \begin{tcolorbox}[breakable, size=fbox, boxrule=1pt, pad at break*=1mm,colback=cellbackground, colframe=cellborder]
\prompt{In}{incolor}{1}{\boxspacing}
\begin{Verbatim}[commandchars=\\\{\}]
\PY{c+c1}{\PYZsh{} 從 Github mirror 下載資料集(原作者 http://yann.lecun.com/exdb/mnist/ 似乎資料已刪除)}


\PY{k}{def}\PY{+w}{ }\PY{n+nf}{get\PYZus{}mnist}\PY{p}{(}\PY{p}{)}\PY{p}{:}
\PY{+w}{    }\PY{l+s+sd}{\PYZdq{}\PYZdq{}\PYZdq{}}
\PY{l+s+sd}{    The code to download the mnist data original came from}
\PY{l+s+sd}{    https://cntk.ai/pythondocs/CNTK\PYZus{}103A\PYZus{}MNIST\PYZus{}DataLoader.html}
\PY{l+s+sd}{    \PYZdq{}\PYZdq{}\PYZdq{}}

    \PY{k+kn}{import}\PY{+w}{ }\PY{n+nn}{gzip}
    \PY{k+kn}{import}\PY{+w}{ }\PY{n+nn}{numpy}\PY{+w}{ }\PY{k}{as}\PY{+w}{ }\PY{n+nn}{np}
    \PY{k+kn}{import}\PY{+w}{ }\PY{n+nn}{os}
    \PY{k+kn}{import}\PY{+w}{ }\PY{n+nn}{struct}

    \PY{k+kn}{from}\PY{+w}{ }\PY{n+nn}{urllib}\PY{n+nn}{.}\PY{n+nn}{request}\PY{+w}{ }\PY{k+kn}{import} \PY{n}{urlretrieve}

    \PY{k}{def}\PY{+w}{ }\PY{n+nf}{load\PYZus{}data}\PY{p}{(}\PY{n}{src}\PY{p}{,} \PY{n}{num\PYZus{}samples}\PY{p}{)}\PY{p}{:}
        \PY{n+nb}{print}\PY{p}{(}\PY{l+s+s2}{\PYZdq{}}\PY{l+s+s2}{Downloading }\PY{l+s+s2}{\PYZdq{}} \PY{o}{+} \PY{n}{src}\PY{p}{)}
        \PY{n}{gzfname}\PY{p}{,} \PY{n}{h} \PY{o}{=} \PY{n}{urlretrieve}\PY{p}{(}\PY{n}{src}\PY{p}{,} \PY{l+s+s2}{\PYZdq{}}\PY{l+s+s2}{./delete.me}\PY{l+s+s2}{\PYZdq{}}\PY{p}{)}
        \PY{n+nb}{print}\PY{p}{(}\PY{l+s+s2}{\PYZdq{}}\PY{l+s+s2}{Done.}\PY{l+s+s2}{\PYZdq{}}\PY{p}{)}
        \PY{k}{try}\PY{p}{:}
            \PY{k}{with} \PY{n}{gzip}\PY{o}{.}\PY{n}{open}\PY{p}{(}\PY{n}{gzfname}\PY{p}{)} \PY{k}{as} \PY{n}{gz}\PY{p}{:}
                \PY{n}{n} \PY{o}{=} \PY{n}{struct}\PY{o}{.}\PY{n}{unpack}\PY{p}{(}\PY{l+s+s2}{\PYZdq{}}\PY{l+s+s2}{I}\PY{l+s+s2}{\PYZdq{}}\PY{p}{,} \PY{n}{gz}\PY{o}{.}\PY{n}{read}\PY{p}{(}\PY{l+m+mi}{4}\PY{p}{)}\PY{p}{)}
                \PY{c+c1}{\PYZsh{} Read magic number.}
                \PY{k}{if} \PY{n}{n}\PY{p}{[}\PY{l+m+mi}{0}\PY{p}{]} \PY{o}{!=} \PY{l+m+mh}{0x3080000}\PY{p}{:}
                    \PY{k}{raise} \PY{n+ne}{Exception}\PY{p}{(}\PY{l+s+s2}{\PYZdq{}}\PY{l+s+s2}{Invalid file: unexpected magic number.}\PY{l+s+s2}{\PYZdq{}}\PY{p}{)}
                \PY{c+c1}{\PYZsh{} Read number of entries.}
                \PY{n}{n} \PY{o}{=} \PY{n}{struct}\PY{o}{.}\PY{n}{unpack}\PY{p}{(}\PY{l+s+s2}{\PYZdq{}}\PY{l+s+s2}{\PYZgt{}I}\PY{l+s+s2}{\PYZdq{}}\PY{p}{,} \PY{n}{gz}\PY{o}{.}\PY{n}{read}\PY{p}{(}\PY{l+m+mi}{4}\PY{p}{)}\PY{p}{)}\PY{p}{[}\PY{l+m+mi}{0}\PY{p}{]}
                \PY{k}{if} \PY{n}{n} \PY{o}{!=} \PY{n}{num\PYZus{}samples}\PY{p}{:}
                    \PY{k}{raise} \PY{n+ne}{Exception}\PY{p}{(}\PY{l+s+sa}{f}\PY{l+s+s2}{\PYZdq{}}\PY{l+s+s2}{Invalid file: expected }\PY{l+s+si}{\PYZob{}}\PY{n}{num\PYZus{}samples}\PY{l+s+si}{\PYZcb{}}\PY{l+s+s2}{ entries.}\PY{l+s+s2}{\PYZdq{}}\PY{p}{)}
                \PY{n}{crow} \PY{o}{=} \PY{n}{struct}\PY{o}{.}\PY{n}{unpack}\PY{p}{(}\PY{l+s+s2}{\PYZdq{}}\PY{l+s+s2}{\PYZgt{}I}\PY{l+s+s2}{\PYZdq{}}\PY{p}{,} \PY{n}{gz}\PY{o}{.}\PY{n}{read}\PY{p}{(}\PY{l+m+mi}{4}\PY{p}{)}\PY{p}{)}\PY{p}{[}\PY{l+m+mi}{0}\PY{p}{]}
                \PY{n}{ccol} \PY{o}{=} \PY{n}{struct}\PY{o}{.}\PY{n}{unpack}\PY{p}{(}\PY{l+s+s2}{\PYZdq{}}\PY{l+s+s2}{\PYZgt{}I}\PY{l+s+s2}{\PYZdq{}}\PY{p}{,} \PY{n}{gz}\PY{o}{.}\PY{n}{read}\PY{p}{(}\PY{l+m+mi}{4}\PY{p}{)}\PY{p}{)}\PY{p}{[}\PY{l+m+mi}{0}\PY{p}{]}
                \PY{k}{if} \PY{n}{crow} \PY{o}{!=} \PY{l+m+mi}{28} \PY{o+ow}{or} \PY{n}{ccol} \PY{o}{!=} \PY{l+m+mi}{28}\PY{p}{:}
                    \PY{k}{raise} \PY{n+ne}{Exception}\PY{p}{(}\PY{l+s+s2}{\PYZdq{}}\PY{l+s+s2}{Invalid file: expected 28 rows/cols per image.}\PY{l+s+s2}{\PYZdq{}}\PY{p}{)}
                \PY{c+c1}{\PYZsh{} Read data.}
                \PY{n}{res} \PY{o}{=} \PY{n}{np}\PY{o}{.}\PY{n}{frombuffer}\PY{p}{(}\PY{n}{gz}\PY{o}{.}\PY{n}{read}\PY{p}{(}\PY{n}{num\PYZus{}samples} \PY{o}{*} \PY{n}{crow} \PY{o}{*} \PY{n}{ccol}\PY{p}{)}\PY{p}{,} \PY{n}{dtype}\PY{o}{=}\PY{n}{np}\PY{o}{.}\PY{n}{uint8}\PY{p}{)}
        \PY{k}{finally}\PY{p}{:}
            \PY{n}{os}\PY{o}{.}\PY{n}{remove}\PY{p}{(}\PY{n}{gzfname}\PY{p}{)}
        \PY{k}{return} \PY{n}{res}\PY{o}{.}\PY{n}{reshape}\PY{p}{(}\PY{p}{(}\PY{n}{num\PYZus{}samples}\PY{p}{,} \PY{n}{crow}\PY{p}{,} \PY{n}{ccol}\PY{p}{)}\PY{p}{)} \PY{o}{/} \PY{l+m+mi}{256} \PY{c+c1}{\PYZsh{} 順便做了 normalize}

    \PY{k}{def}\PY{+w}{ }\PY{n+nf}{load\PYZus{}labels}\PY{p}{(}\PY{n}{src}\PY{p}{,} \PY{n}{num\PYZus{}samples}\PY{p}{)}\PY{p}{:}
        \PY{n+nb}{print}\PY{p}{(}\PY{l+s+s2}{\PYZdq{}}\PY{l+s+s2}{Downloading }\PY{l+s+s2}{\PYZdq{}} \PY{o}{+} \PY{n}{src}\PY{p}{)}
        \PY{n}{gzfname}\PY{p}{,} \PY{n}{h} \PY{o}{=} \PY{n}{urlretrieve}\PY{p}{(}\PY{n}{src}\PY{p}{,} \PY{l+s+s2}{\PYZdq{}}\PY{l+s+s2}{./delete.me}\PY{l+s+s2}{\PYZdq{}}\PY{p}{)}
        \PY{n+nb}{print}\PY{p}{(}\PY{l+s+s2}{\PYZdq{}}\PY{l+s+s2}{Done.}\PY{l+s+s2}{\PYZdq{}}\PY{p}{)}
        \PY{k}{try}\PY{p}{:}
            \PY{k}{with} \PY{n}{gzip}\PY{o}{.}\PY{n}{open}\PY{p}{(}\PY{n}{gzfname}\PY{p}{)} \PY{k}{as} \PY{n}{gz}\PY{p}{:}
                \PY{n}{n} \PY{o}{=} \PY{n}{struct}\PY{o}{.}\PY{n}{unpack}\PY{p}{(}\PY{l+s+s2}{\PYZdq{}}\PY{l+s+s2}{I}\PY{l+s+s2}{\PYZdq{}}\PY{p}{,} \PY{n}{gz}\PY{o}{.}\PY{n}{read}\PY{p}{(}\PY{l+m+mi}{4}\PY{p}{)}\PY{p}{)}
                \PY{c+c1}{\PYZsh{} Read magic number.}
                \PY{k}{if} \PY{n}{n}\PY{p}{[}\PY{l+m+mi}{0}\PY{p}{]} \PY{o}{!=} \PY{l+m+mh}{0x1080000}\PY{p}{:}
                    \PY{k}{raise} \PY{n+ne}{Exception}\PY{p}{(}\PY{l+s+s2}{\PYZdq{}}\PY{l+s+s2}{Invalid file: unexpected magic number.}\PY{l+s+s2}{\PYZdq{}}\PY{p}{)}
                \PY{c+c1}{\PYZsh{} Read number of entries.}
                \PY{n}{n} \PY{o}{=} \PY{n}{struct}\PY{o}{.}\PY{n}{unpack}\PY{p}{(}\PY{l+s+s2}{\PYZdq{}}\PY{l+s+s2}{\PYZgt{}I}\PY{l+s+s2}{\PYZdq{}}\PY{p}{,} \PY{n}{gz}\PY{o}{.}\PY{n}{read}\PY{p}{(}\PY{l+m+mi}{4}\PY{p}{)}\PY{p}{)}
                \PY{k}{if} \PY{n}{n}\PY{p}{[}\PY{l+m+mi}{0}\PY{p}{]} \PY{o}{!=} \PY{n}{num\PYZus{}samples}\PY{p}{:}
                    \PY{k}{raise} \PY{n+ne}{Exception}\PY{p}{(}\PY{l+s+sa}{f}\PY{l+s+s2}{\PYZdq{}}\PY{l+s+s2}{Invalid file: expected }\PY{l+s+si}{\PYZob{}}\PY{n}{num\PYZus{}samples}\PY{l+s+si}{\PYZcb{}}\PY{l+s+s2}{ rows.}\PY{l+s+s2}{\PYZdq{}}\PY{p}{)}
                \PY{c+c1}{\PYZsh{} Read labels.}
                \PY{n}{res} \PY{o}{=} \PY{n}{np}\PY{o}{.}\PY{n}{frombuffer}\PY{p}{(}\PY{n}{gz}\PY{o}{.}\PY{n}{read}\PY{p}{(}\PY{n}{num\PYZus{}samples}\PY{p}{)}\PY{p}{,} \PY{n}{dtype}\PY{o}{=}\PY{n}{np}\PY{o}{.}\PY{n}{uint8}\PY{p}{)}
        \PY{k}{finally}\PY{p}{:}
            \PY{n}{os}\PY{o}{.}\PY{n}{remove}\PY{p}{(}\PY{n}{gzfname}\PY{p}{)}
        \PY{k}{return} \PY{n}{res}\PY{o}{.}\PY{n}{reshape}\PY{p}{(}\PY{n}{num\PYZus{}samples}\PY{p}{)}

    \PY{k}{def}\PY{+w}{ }\PY{n+nf}{try\PYZus{}download}\PY{p}{(}\PY{n}{data\PYZus{}source}\PY{p}{,} \PY{n}{label\PYZus{}source}\PY{p}{,} \PY{n}{num\PYZus{}samples}\PY{p}{)}\PY{p}{:}
        \PY{n}{data} \PY{o}{=} \PY{n}{load\PYZus{}data}\PY{p}{(}\PY{n}{data\PYZus{}source}\PY{p}{,} \PY{n}{num\PYZus{}samples}\PY{p}{)}
        \PY{n}{labels} \PY{o}{=} \PY{n}{load\PYZus{}labels}\PY{p}{(}\PY{n}{label\PYZus{}source}\PY{p}{,} \PY{n}{num\PYZus{}samples}\PY{p}{)}
        \PY{k}{return} \PY{n}{data}\PY{p}{,} \PY{n}{labels}

    \PY{c+c1}{\PYZsh{} Not sure why, but yann lecun\PYZsq{}s website does no longer support}
    \PY{c+c1}{\PYZsh{} simple downloader. (e.g. urlretrieve and wget fail, while curl work)}
    \PY{c+c1}{\PYZsh{} Since not everyone has linux, use a mirror from uni server.}
    \PY{c+c1}{\PYZsh{}     server = \PYZsq{}http://yann.lecun.com/exdb/mnist\PYZsq{}}
    \PY{n}{server} \PY{o}{=} \PY{l+s+s2}{\PYZdq{}}\PY{l+s+s2}{https://raw.githubusercontent.com/fgnt/mnist/master}\PY{l+s+s2}{\PYZdq{}}

    \PY{c+c1}{\PYZsh{} URLs for the train image and label data}
    \PY{n}{url\PYZus{}train\PYZus{}image} \PY{o}{=} \PY{l+s+sa}{f}\PY{l+s+s2}{\PYZdq{}}\PY{l+s+si}{\PYZob{}}\PY{n}{server}\PY{l+s+si}{\PYZcb{}}\PY{l+s+s2}{/train\PYZhy{}images\PYZhy{}idx3\PYZhy{}ubyte.gz}\PY{l+s+s2}{\PYZdq{}}
    \PY{n}{url\PYZus{}train\PYZus{}labels} \PY{o}{=} \PY{l+s+sa}{f}\PY{l+s+s2}{\PYZdq{}}\PY{l+s+si}{\PYZob{}}\PY{n}{server}\PY{l+s+si}{\PYZcb{}}\PY{l+s+s2}{/train\PYZhy{}labels\PYZhy{}idx1\PYZhy{}ubyte.gz}\PY{l+s+s2}{\PYZdq{}}
    \PY{n}{num\PYZus{}train\PYZus{}samples} \PY{o}{=} \PY{l+m+mi}{60000}

    \PY{n+nb}{print}\PY{p}{(}\PY{l+s+s2}{\PYZdq{}}\PY{l+s+s2}{Downloading train data}\PY{l+s+s2}{\PYZdq{}}\PY{p}{)}
    \PY{n}{train\PYZus{}features}\PY{p}{,} \PY{n}{train\PYZus{}labels} \PY{o}{=} \PY{n}{try\PYZus{}download}\PY{p}{(}\PY{n}{url\PYZus{}train\PYZus{}image}\PY{p}{,} \PY{n}{url\PYZus{}train\PYZus{}labels}\PY{p}{,} \PY{n}{num\PYZus{}train\PYZus{}samples}\PY{p}{)}

    \PY{c+c1}{\PYZsh{} URLs for the test image and label data}
    \PY{n}{url\PYZus{}test\PYZus{}image} \PY{o}{=} \PY{l+s+sa}{f}\PY{l+s+s2}{\PYZdq{}}\PY{l+s+si}{\PYZob{}}\PY{n}{server}\PY{l+s+si}{\PYZcb{}}\PY{l+s+s2}{/t10k\PYZhy{}images\PYZhy{}idx3\PYZhy{}ubyte.gz}\PY{l+s+s2}{\PYZdq{}}
    \PY{n}{url\PYZus{}test\PYZus{}labels} \PY{o}{=} \PY{l+s+sa}{f}\PY{l+s+s2}{\PYZdq{}}\PY{l+s+si}{\PYZob{}}\PY{n}{server}\PY{l+s+si}{\PYZcb{}}\PY{l+s+s2}{/t10k\PYZhy{}labels\PYZhy{}idx1\PYZhy{}ubyte.gz}\PY{l+s+s2}{\PYZdq{}}
    \PY{n}{num\PYZus{}test\PYZus{}samples} \PY{o}{=} \PY{l+m+mi}{10000}

    \PY{n+nb}{print}\PY{p}{(}\PY{l+s+s2}{\PYZdq{}}\PY{l+s+s2}{Downloading test data}\PY{l+s+s2}{\PYZdq{}}\PY{p}{)}
    \PY{n}{test\PYZus{}features}\PY{p}{,} \PY{n}{test\PYZus{}labels} \PY{o}{=} \PY{n}{try\PYZus{}download}\PY{p}{(}\PY{n}{url\PYZus{}test\PYZus{}image}\PY{p}{,} \PY{n}{url\PYZus{}test\PYZus{}labels}\PY{p}{,} \PY{n}{num\PYZus{}test\PYZus{}samples}\PY{p}{)}

    \PY{k}{return} \PY{n}{train\PYZus{}features}\PY{p}{,} \PY{n}{train\PYZus{}labels}\PY{p}{,} \PY{n}{test\PYZus{}features}\PY{p}{,} \PY{n}{test\PYZus{}labels}
\end{Verbatim}
\end{tcolorbox}

    \begin{tcolorbox}[breakable, size=fbox, boxrule=1pt, pad at break*=1mm,colback=cellbackground, colframe=cellborder]
\prompt{In}{incolor}{2}{\boxspacing}
\begin{Verbatim}[commandchars=\\\{\}]
\PY{n}{train\PYZus{}features}\PY{p}{,} \PY{n}{train\PYZus{}labels}\PY{p}{,} \PY{n}{test\PYZus{}features}\PY{p}{,} \PY{n}{test\PYZus{}labels} \PY{o}{=} \PY{n}{get\PYZus{}mnist}\PY{p}{(}\PY{p}{)}
\end{Verbatim}
\end{tcolorbox}

    \begin{Verbatim}[commandchars=\\\{\}]
Downloading train data
Downloading https://raw.githubusercontent.com/fgnt/mnist/master/train-images-
idx3-ubyte.gz
Done.
Downloading https://raw.githubusercontent.com/fgnt/mnist/master/train-labels-
idx1-ubyte.gz
Done.
Downloading test data
Downloading https://raw.githubusercontent.com/fgnt/mnist/master/t10k-images-
idx3-ubyte.gz
Done.
Downloading https://raw.githubusercontent.com/fgnt/mnist/master/t10k-labels-
idx1-ubyte.gz
Done.
    \end{Verbatim}

    \begin{tcolorbox}[breakable, size=fbox, boxrule=1pt, pad at break*=1mm,colback=cellbackground, colframe=cellborder]
\prompt{In}{incolor}{3}{\boxspacing}
\begin{Verbatim}[commandchars=\\\{\}]
\PY{k+kn}{import}\PY{+w}{ }\PY{n+nn}{os}

\PY{n}{os}\PY{o}{.}\PY{n}{makedirs}\PY{p}{(}\PY{l+s+s2}{\PYZdq{}}\PY{l+s+s2}{data}\PY{l+s+s2}{\PYZdq{}}\PY{p}{,} \PY{n}{exist\PYZus{}ok}\PY{o}{=}\PY{k+kc}{True}\PY{p}{)}  \PY{c+c1}{\PYZsh{} 確保 data 資料夾存在}
\end{Verbatim}
\end{tcolorbox}

    \begin{tcolorbox}[breakable, size=fbox, boxrule=1pt, pad at break*=1mm,colback=cellbackground, colframe=cellborder]
\prompt{In}{incolor}{4}{\boxspacing}
\begin{Verbatim}[commandchars=\\\{\}]
\PY{k+kn}{import}\PY{+w}{ }\PY{n+nn}{numpy}\PY{+w}{ }\PY{k}{as}\PY{+w}{ }\PY{n+nn}{np}

\PY{n}{np}\PY{o}{.}\PY{n}{savez\PYZus{}compressed}\PY{p}{(}
    \PY{l+s+s2}{\PYZdq{}}\PY{l+s+s2}{data/mnist\PYZus{}dataset.npz}\PY{l+s+s2}{\PYZdq{}}\PY{p}{,} \PY{n}{train\PYZus{}x}\PY{o}{=}\PY{n}{train\PYZus{}features}\PY{p}{,} \PY{n}{train\PYZus{}y}\PY{o}{=}\PY{n}{train\PYZus{}labels}\PY{p}{,} \PY{n}{test\PYZus{}x}\PY{o}{=}\PY{n}{test\PYZus{}features}\PY{p}{,} \PY{n}{test\PYZus{}y}\PY{o}{=}\PY{n}{test\PYZus{}labels}
\PY{p}{)}
\end{Verbatim}
\end{tcolorbox}

    \section{Step 1: Data Processing}\label{step-1-data-processing}

    \subsection{Step 1-1: Read data and
reshape}\label{step-1-1-read-data-and-reshape}

    \begin{tcolorbox}[breakable, size=fbox, boxrule=1pt, pad at break*=1mm,colback=cellbackground, colframe=cellborder]
\prompt{In}{incolor}{5}{\boxspacing}
\begin{Verbatim}[commandchars=\\\{\}]
\PY{c+c1}{\PYZsh{} 讀取已儲存的 .npz 檔案}
\PY{n}{mnist\PYZus{}data} \PY{o}{=} \PY{n}{np}\PY{o}{.}\PY{n}{load}\PY{p}{(}\PY{l+s+s2}{\PYZdq{}}\PY{l+s+s2}{data/mnist\PYZus{}dataset.npz}\PY{l+s+s2}{\PYZdq{}}\PY{p}{)}

\PY{c+c1}{\PYZsh{} 透過儲存時設定的關鍵字名稱,來取出各個 NDArray}
\PY{n}{train\PYZus{}x\PYZus{}orig} \PY{o}{=} \PY{n}{mnist\PYZus{}data}\PY{p}{[}\PY{l+s+s2}{\PYZdq{}}\PY{l+s+s2}{train\PYZus{}x}\PY{l+s+s2}{\PYZdq{}}\PY{p}{]}
\PY{n}{train\PYZus{}y\PYZus{}orig} \PY{o}{=} \PY{n}{mnist\PYZus{}data}\PY{p}{[}\PY{l+s+s2}{\PYZdq{}}\PY{l+s+s2}{train\PYZus{}y}\PY{l+s+s2}{\PYZdq{}}\PY{p}{]}
\PY{n}{test\PYZus{}x\PYZus{}orig} \PY{o}{=} \PY{n}{mnist\PYZus{}data}\PY{p}{[}\PY{l+s+s2}{\PYZdq{}}\PY{l+s+s2}{test\PYZus{}x}\PY{l+s+s2}{\PYZdq{}}\PY{p}{]}
\PY{n}{test\PYZus{}y\PYZus{}orig} \PY{o}{=} \PY{n}{mnist\PYZus{}data}\PY{p}{[}\PY{l+s+s2}{\PYZdq{}}\PY{l+s+s2}{test\PYZus{}y}\PY{l+s+s2}{\PYZdq{}}\PY{p}{]}

\PY{c+c1}{\PYZsh{} 驗證一下 shape}
\PY{n+nb}{print}\PY{p}{(}\PY{l+s+s2}{\PYZdq{}}\PY{l+s+s2}{train\PYZus{}x\PYZus{}orig.shape:}\PY{l+s+s2}{\PYZdq{}}\PY{p}{,} \PY{n}{train\PYZus{}x\PYZus{}orig}\PY{o}{.}\PY{n}{shape}\PY{p}{)}
\PY{n+nb}{print}\PY{p}{(}\PY{l+s+s2}{\PYZdq{}}\PY{l+s+s2}{train\PYZus{}y\PYZus{}orig.shape:}\PY{l+s+s2}{\PYZdq{}}\PY{p}{,} \PY{n}{train\PYZus{}y\PYZus{}orig}\PY{o}{.}\PY{n}{shape}\PY{p}{)}
\PY{n+nb}{print}\PY{p}{(}\PY{l+s+s2}{\PYZdq{}}\PY{l+s+s2}{test\PYZus{}x\PYZus{}orig.shape:}\PY{l+s+s2}{\PYZdq{}}\PY{p}{,} \PY{n}{test\PYZus{}x\PYZus{}orig}\PY{o}{.}\PY{n}{shape}\PY{p}{)}
\PY{n+nb}{print}\PY{p}{(}\PY{l+s+s2}{\PYZdq{}}\PY{l+s+s2}{test\PYZus{}y\PYZus{}orig.shape:}\PY{l+s+s2}{\PYZdq{}}\PY{p}{,} \PY{n}{test\PYZus{}y\PYZus{}orig}\PY{o}{.}\PY{n}{shape}\PY{p}{)}
\end{Verbatim}
\end{tcolorbox}

    \begin{Verbatim}[commandchars=\\\{\}]
train\_x\_orig.shape: (60000, 28, 28)
train\_y\_orig.shape: (60000,)
test\_x\_orig.shape: (10000, 28, 28)
test\_y\_orig.shape: (10000,)
    \end{Verbatim}

    \begin{tcolorbox}[breakable, size=fbox, boxrule=1pt, pad at break*=1mm,colback=cellbackground, colframe=cellborder]
\prompt{In}{incolor}{6}{\boxspacing}
\begin{Verbatim}[commandchars=\\\{\}]
\PY{k+kn}{import}\PY{+w}{ }\PY{n+nn}{matplotlib}\PY{n+nn}{.}\PY{n+nn}{pyplot}\PY{+w}{ }\PY{k}{as}\PY{+w}{ }\PY{n+nn}{plt}

\PY{c+c1}{\PYZsh{} 建立一個 12x5 的畫布}
\PY{n}{plt}\PY{o}{.}\PY{n}{figure}\PY{p}{(}\PY{n}{figsize}\PY{o}{=}\PY{p}{(}\PY{l+m+mi}{12}\PY{p}{,} \PY{l+m+mi}{5}\PY{p}{)}\PY{p}{)}

\PY{c+c1}{\PYZsh{} 迴圈顯示前 10 張圖片}
\PY{k}{for} \PY{n}{i} \PY{o+ow}{in} \PY{n+nb}{range}\PY{p}{(}\PY{l+m+mi}{10}\PY{p}{)}\PY{p}{:}
    \PY{c+c1}{\PYZsh{} 建立一個 2x5 的子圖網格,並定位到第 i+1 的位置}
    \PY{n}{plt}\PY{o}{.}\PY{n}{subplot}\PY{p}{(}\PY{l+m+mi}{2}\PY{p}{,} \PY{l+m+mi}{5}\PY{p}{,} \PY{n}{i} \PY{o}{+} \PY{l+m+mi}{1}\PY{p}{)}

    \PY{c+c1}{\PYZsh{} 顯示 train\PYZus{}x\PYZus{}orig 中的第 i 張圖片,使用灰階色彩}
    \PY{n}{plt}\PY{o}{.}\PY{n}{imshow}\PY{p}{(}\PY{n}{train\PYZus{}x\PYZus{}orig}\PY{p}{[}\PY{n}{i}\PY{p}{]}\PY{p}{,} \PY{n}{cmap}\PY{o}{=}\PY{l+s+s2}{\PYZdq{}}\PY{l+s+s2}{gray}\PY{l+s+s2}{\PYZdq{}}\PY{p}{)}

    \PY{c+c1}{\PYZsh{} 設定子圖的標題為該圖片的正解標籤}
    \PY{n}{plt}\PY{o}{.}\PY{n}{title}\PY{p}{(}\PY{l+s+sa}{f}\PY{l+s+s2}{\PYZdq{}}\PY{l+s+s2}{Label: }\PY{l+s+si}{\PYZob{}}\PY{n}{train\PYZus{}y\PYZus{}orig}\PY{p}{[}\PY{n}{i}\PY{p}{]}\PY{l+s+si}{\PYZcb{}}\PY{l+s+s2}{\PYZdq{}}\PY{p}{)}

    \PY{c+c1}{\PYZsh{} 關閉座標軸}
    \PY{n}{plt}\PY{o}{.}\PY{n}{axis}\PY{p}{(}\PY{l+s+s2}{\PYZdq{}}\PY{l+s+s2}{off}\PY{l+s+s2}{\PYZdq{}}\PY{p}{)}

\PY{c+c1}{\PYZsh{} 為整個圖表加上一個大標題}
\PY{n}{plt}\PY{o}{.}\PY{n}{suptitle}\PY{p}{(}\PY{l+s+s2}{\PYZdq{}}\PY{l+s+s2}{Display of the First 10 Images}\PY{l+s+s2}{\PYZdq{}}\PY{p}{,} \PY{n}{fontsize}\PY{o}{=}\PY{l+m+mi}{16}\PY{p}{)}

\PY{c+c1}{\PYZsh{} 顯示圖片}
\PY{n}{plt}\PY{o}{.}\PY{n}{show}\PY{p}{(}\PY{p}{)}
\end{Verbatim}
\end{tcolorbox}

    \begin{center}
    \adjustimage{max size={0.9\linewidth}{0.9\paperheight}}{output_9_0.png}
    \end{center}
    { \hspace*{\fill} \\}
    
    \begin{tcolorbox}[breakable, size=fbox, boxrule=1pt, pad at break*=1mm,colback=cellbackground, colframe=cellborder]
\prompt{In}{incolor}{7}{\boxspacing}
\begin{Verbatim}[commandchars=\\\{\}]
\PY{c+c1}{\PYZsh{} 將 28x28 的圖片拉平成 784 的向量}
\PY{c+c1}{\PYZsh{} \PYZhy{}1 會讓 numpy 自動計算該維度的數量}
\PY{n}{train\PYZus{}x} \PY{o}{=} \PY{n}{train\PYZus{}x\PYZus{}orig}\PY{o}{.}\PY{n}{reshape}\PY{p}{(}\PY{n}{train\PYZus{}x\PYZus{}orig}\PY{o}{.}\PY{n}{shape}\PY{p}{[}\PY{l+m+mi}{0}\PY{p}{]}\PY{p}{,} \PY{o}{\PYZhy{}}\PY{l+m+mi}{1}\PY{p}{)}
\PY{n}{test\PYZus{}x} \PY{o}{=} \PY{n}{test\PYZus{}x\PYZus{}orig}\PY{o}{.}\PY{n}{reshape}\PY{p}{(}\PY{n}{test\PYZus{}x\PYZus{}orig}\PY{o}{.}\PY{n}{shape}\PY{p}{[}\PY{l+m+mi}{0}\PY{p}{]}\PY{p}{,} \PY{o}{\PYZhy{}}\PY{l+m+mi}{1}\PY{p}{)}

\PY{n+nb}{print}\PY{p}{(}\PY{l+s+s2}{\PYZdq{}}\PY{l+s+s2}{tran\PYZus{}x.shape:}\PY{l+s+s2}{\PYZdq{}}\PY{p}{,} \PY{n}{train\PYZus{}x}\PY{o}{.}\PY{n}{shape}\PY{p}{)}  \PY{c+c1}{\PYZsh{} (60000, 784)}
\PY{n+nb}{print}\PY{p}{(}\PY{l+s+s2}{\PYZdq{}}\PY{l+s+s2}{test\PYZus{}x.shape:}\PY{l+s+s2}{\PYZdq{}}\PY{p}{,} \PY{n}{test\PYZus{}x}\PY{o}{.}\PY{n}{shape}\PY{p}{)}  \PY{c+c1}{\PYZsh{} (10000, 784)}
\end{Verbatim}
\end{tcolorbox}

    \begin{Verbatim}[commandchars=\\\{\}]
tran\_x.shape: (60000, 784)
test\_x.shape: (10000, 784)
    \end{Verbatim}

    \begin{quote}
在下載的函數中已經將圖片的每個像素值除以 256,已經做完 Normalization
\end{quote}

    \subsection{Step 1-2: 對 Label 進行 One-Hot
Encode}\label{step-1-2-ux5c0d-label-ux9032ux884c-one-hot-encode}

    \begin{tcolorbox}[breakable, size=fbox, boxrule=1pt, pad at break*=1mm,colback=cellbackground, colframe=cellborder]
\prompt{In}{incolor}{8}{\boxspacing}
\begin{Verbatim}[commandchars=\\\{\}]
\PY{k}{def}\PY{+w}{ }\PY{n+nf}{one\PYZus{}hot\PYZus{}encode}\PY{p}{(}\PY{n}{labels}\PY{p}{,} \PY{n}{num\PYZus{}classes}\PY{p}{)}\PY{p}{:}
    \PY{c+c1}{\PYZsh{} 創建一個全為 0 的矩陣,shape 為 (樣本數, 類別數)}
    \PY{n}{one\PYZus{}hot} \PY{o}{=} \PY{n}{np}\PY{o}{.}\PY{n}{zeros}\PY{p}{(}\PY{p}{(}\PY{n}{labels}\PY{o}{.}\PY{n}{size}\PY{p}{,} \PY{n}{num\PYZus{}classes}\PY{p}{)}\PY{p}{)}
    \PY{c+c1}{\PYZsh{} 在對應的類別位置上填上 1}
    \PY{c+c1}{\PYZsh{} np.arange(labels.size) 會產生 [0, 1, 2, ...] 用於定位行}
    \PY{c+c1}{\PYZsh{} labels 本身 ([5, 0, 4, ...]) 用於定位列}
    \PY{n}{one\PYZus{}hot}\PY{p}{[}\PY{n}{np}\PY{o}{.}\PY{n}{arange}\PY{p}{(}\PY{n}{labels}\PY{o}{.}\PY{n}{size}\PY{p}{)}\PY{p}{,} \PY{n}{labels}\PY{p}{]} \PY{o}{=} \PY{l+m+mi}{1}
    \PY{k}{return} \PY{n}{one\PYZus{}hot}


\PY{c+c1}{\PYZsh{} 我們的類別是數字 0\PYZhy{}9,所以有 10 個類別}
\PY{n}{num\PYZus{}classes} \PY{o}{=} \PY{l+m+mi}{10}
\PY{n}{train\PYZus{}y} \PY{o}{=} \PY{n}{one\PYZus{}hot\PYZus{}encode}\PY{p}{(}\PY{n}{train\PYZus{}y\PYZus{}orig}\PY{p}{,} \PY{n}{num\PYZus{}classes}\PY{p}{)}
\PY{n}{test\PYZus{}y} \PY{o}{=} \PY{n}{one\PYZus{}hot\PYZus{}encode}\PY{p}{(}\PY{n}{test\PYZus{}y\PYZus{}orig}\PY{p}{,} \PY{n}{num\PYZus{}classes}\PY{p}{)}

\PY{n+nb}{print}\PY{p}{(}\PY{l+s+s2}{\PYZdq{}}\PY{l+s+s2}{One\PYZhy{}Hot encode 前的 train\PYZus{}y.shape:}\PY{l+s+s2}{\PYZdq{}}\PY{p}{,} \PY{n}{train\PYZus{}y\PYZus{}orig}\PY{o}{.}\PY{n}{shape}\PY{p}{)}  \PY{c+c1}{\PYZsh{} (60000,)}
\PY{n+nb}{print}\PY{p}{(}\PY{l+s+s2}{\PYZdq{}}\PY{l+s+s2}{One\PYZhy{}Hot 後的 train\PYZus{}y.shape:}\PY{l+s+s2}{\PYZdq{}}\PY{p}{,} \PY{n}{train\PYZus{}y}\PY{o}{.}\PY{n}{shape}\PY{p}{)}  \PY{c+c1}{\PYZsh{} (60000, 10)}
\PY{n+nb}{print}\PY{p}{(}\PY{l+s+sa}{f}\PY{l+s+s2}{\PYZdq{}}\PY{l+s+s2}{一個 One\PYZhy{}Hot 編碼範例 (原始 label 是 }\PY{l+s+si}{\PYZob{}}\PY{n}{train\PYZus{}y\PYZus{}orig}\PY{p}{[}\PY{l+m+mi}{0}\PY{p}{]}\PY{l+s+si}{\PYZcb{}}\PY{l+s+s2}{): }\PY{l+s+si}{\PYZob{}}\PY{n}{train\PYZus{}y}\PY{p}{[}\PY{l+m+mi}{0}\PY{p}{]}\PY{l+s+si}{\PYZcb{}}\PY{l+s+s2}{\PYZdq{}}\PY{p}{)}
\end{Verbatim}
\end{tcolorbox}

    \begin{Verbatim}[commandchars=\\\{\}]
One-Hot encode 前的 train\_y.shape: (60000,)
One-Hot 後的 train\_y.shape: (60000, 10)
一個 One-Hot 編碼範例 (原始 label 是 5): [0. 0. 0. 0. 0. 1. 0. 0. 0. 0.]
    \end{Verbatim}

    \section{Step 2:
實作神經網路}\label{step-2-ux5be6ux4f5cux795eux7d93ux7db2ux8def}

    \subsection{Step 2-1:
初始化權重}\label{step-2-1-ux521dux59cbux5316ux6b0aux91cd}

初始化兩層神經網路的參數

參數:

\begin{itemize}
\tightlist
\item
  n\_x -- 輸入層的神經元數量(784)
\item
  n\_h -- 隱藏層的神經元數量(例如 128)
\item
  n\_y -- 輸出層的神經元數量(10)
\end{itemize}

Return:

\begin{itemize}
\tightlist
\item
  parameters -- 一個包含 W1, b1, W2, b2 的 Python dict
\end{itemize}

    \begin{tcolorbox}[breakable, size=fbox, boxrule=1pt, pad at break*=1mm,colback=cellbackground, colframe=cellborder]
\prompt{In}{incolor}{9}{\boxspacing}
\begin{Verbatim}[commandchars=\\\{\}]
\PY{k}{def}\PY{+w}{ }\PY{n+nf}{initialize\PYZus{}parameters}\PY{p}{(}\PY{n}{n\PYZus{}x}\PY{p}{,} \PY{n}{n\PYZus{}h}\PY{p}{,} \PY{n}{n\PYZus{}y}\PY{p}{)}\PY{p}{:}
    \PY{c+c1}{\PYZsh{} 為了讓每次的隨機結果都一樣,設定一個隨機種子}
    \PY{n}{np}\PY{o}{.}\PY{n}{random}\PY{o}{.}\PY{n}{seed}\PY{p}{(}\PY{l+m+mi}{42}\PY{p}{)}

    \PY{c+c1}{\PYZsh{} 初始化權重 W1 和 W2,使用常態分佈的隨機數並乘上一個小數 (如 0.01)}
    \PY{c+c1}{\PYZsh{} 這樣可以避免起始權重過大}
    \PY{n}{W1} \PY{o}{=} \PY{n}{np}\PY{o}{.}\PY{n}{random}\PY{o}{.}\PY{n}{randn}\PY{p}{(}\PY{n}{n\PYZus{}x}\PY{p}{,} \PY{n}{n\PYZus{}h}\PY{p}{)} \PY{o}{*} \PY{l+m+mf}{0.01}
    \PY{n}{W2} \PY{o}{=} \PY{n}{np}\PY{o}{.}\PY{n}{random}\PY{o}{.}\PY{n}{randn}\PY{p}{(}\PY{n}{n\PYZus{}h}\PY{p}{,} \PY{n}{n\PYZus{}y}\PY{p}{)} \PY{o}{*} \PY{l+m+mf}{0.01}

    \PY{c+c1}{\PYZsh{} 將 b1 初始化為一個微小的正數,例如 0.01,避免 dying ReLU}
    \PY{n}{b1} \PY{o}{=} \PY{n}{np}\PY{o}{.}\PY{n}{ones}\PY{p}{(}\PY{p}{(}\PY{l+m+mi}{1}\PY{p}{,} \PY{n}{n\PYZus{}h}\PY{p}{)}\PY{p}{)} \PY{o}{*} \PY{l+m+mf}{0.01}

    \PY{c+c1}{\PYZsh{} b2 初始化為 0(它影響的是 softmax)}
    \PY{n}{b2} \PY{o}{=} \PY{n}{np}\PY{o}{.}\PY{n}{zeros}\PY{p}{(}\PY{p}{(}\PY{l+m+mi}{1}\PY{p}{,} \PY{n}{n\PYZus{}y}\PY{p}{)}\PY{p}{)}

    \PY{c+c1}{\PYZsh{} 將所有參數打包到一個字典中}
    \PY{n}{parameters} \PY{o}{=} \PY{p}{\PYZob{}}\PY{l+s+s2}{\PYZdq{}}\PY{l+s+s2}{W1}\PY{l+s+s2}{\PYZdq{}}\PY{p}{:} \PY{n}{W1}\PY{p}{,} \PY{l+s+s2}{\PYZdq{}}\PY{l+s+s2}{b1}\PY{l+s+s2}{\PYZdq{}}\PY{p}{:} \PY{n}{b1}\PY{p}{,} \PY{l+s+s2}{\PYZdq{}}\PY{l+s+s2}{W2}\PY{l+s+s2}{\PYZdq{}}\PY{p}{:} \PY{n}{W2}\PY{p}{,} \PY{l+s+s2}{\PYZdq{}}\PY{l+s+s2}{b2}\PY{l+s+s2}{\PYZdq{}}\PY{p}{:} \PY{n}{b2}\PY{p}{\PYZcb{}}

    \PY{k}{return} \PY{n}{parameters}
\end{Verbatim}
\end{tcolorbox}

    \begin{tcolorbox}[breakable, size=fbox, boxrule=1pt, pad at break*=1mm,colback=cellbackground, colframe=cellborder]
\prompt{In}{incolor}{10}{\boxspacing}
\begin{Verbatim}[commandchars=\\\{\}]
\PY{c+c1}{\PYZsh{} 測試函數}
\PY{n}{parameters} \PY{o}{=} \PY{n}{initialize\PYZus{}parameters}\PY{p}{(}\PY{l+m+mi}{784}\PY{p}{,} \PY{l+m+mi}{128}\PY{p}{,} \PY{l+m+mi}{10}\PY{p}{)}

\PY{c+c1}{\PYZsh{} print 出參數的名稱和其維度,確認是否正確}
\PY{n+nb}{print}\PY{p}{(}\PY{l+s+s2}{\PYZdq{}}\PY{l+s+s2}{W1 Shape: }\PY{l+s+s2}{\PYZdq{}} \PY{o}{+} \PY{n+nb}{str}\PY{p}{(}\PY{n}{parameters}\PY{p}{[}\PY{l+s+s2}{\PYZdq{}}\PY{l+s+s2}{W1}\PY{l+s+s2}{\PYZdq{}}\PY{p}{]}\PY{o}{.}\PY{n}{shape}\PY{p}{)}\PY{p}{)}
\PY{n+nb}{print}\PY{p}{(}\PY{l+s+s2}{\PYZdq{}}\PY{l+s+s2}{b1 Shape: }\PY{l+s+s2}{\PYZdq{}} \PY{o}{+} \PY{n+nb}{str}\PY{p}{(}\PY{n}{parameters}\PY{p}{[}\PY{l+s+s2}{\PYZdq{}}\PY{l+s+s2}{b1}\PY{l+s+s2}{\PYZdq{}}\PY{p}{]}\PY{o}{.}\PY{n}{shape}\PY{p}{)}\PY{p}{)}
\PY{n+nb}{print}\PY{p}{(}\PY{l+s+s2}{\PYZdq{}}\PY{l+s+s2}{W2 Shape: }\PY{l+s+s2}{\PYZdq{}} \PY{o}{+} \PY{n+nb}{str}\PY{p}{(}\PY{n}{parameters}\PY{p}{[}\PY{l+s+s2}{\PYZdq{}}\PY{l+s+s2}{W2}\PY{l+s+s2}{\PYZdq{}}\PY{p}{]}\PY{o}{.}\PY{n}{shape}\PY{p}{)}\PY{p}{)}
\PY{n+nb}{print}\PY{p}{(}\PY{l+s+s2}{\PYZdq{}}\PY{l+s+s2}{b2 Shape: }\PY{l+s+s2}{\PYZdq{}} \PY{o}{+} \PY{n+nb}{str}\PY{p}{(}\PY{n}{parameters}\PY{p}{[}\PY{l+s+s2}{\PYZdq{}}\PY{l+s+s2}{b2}\PY{l+s+s2}{\PYZdq{}}\PY{p}{]}\PY{o}{.}\PY{n}{shape}\PY{p}{)}\PY{p}{)}
\end{Verbatim}
\end{tcolorbox}

    \begin{Verbatim}[commandchars=\\\{\}]
W1 Shape: (784, 128)
b1 Shape: (1, 128)
W2 Shape: (128, 10)
b2 Shape: (1, 10)
    \end{Verbatim}

    \subsection{Step 2-2: Activation
Function}\label{step-2-2-activation-function}

    \begin{tcolorbox}[breakable, size=fbox, boxrule=1pt, pad at break*=1mm,colback=cellbackground, colframe=cellborder]
\prompt{In}{incolor}{11}{\boxspacing}
\begin{Verbatim}[commandchars=\\\{\}]
\PY{k}{def}\PY{+w}{ }\PY{n+nf}{relu}\PY{p}{(}\PY{n}{Z}\PY{p}{)}\PY{p}{:}
\PY{+w}{    }\PY{l+s+sd}{\PYZdq{}\PYZdq{}\PYZdq{}}
\PY{l+s+sd}{    ReLU 激活函數}
\PY{l+s+sd}{    \PYZdq{}\PYZdq{}\PYZdq{}}
    \PY{k}{return} \PY{n}{np}\PY{o}{.}\PY{n}{maximum}\PY{p}{(}\PY{l+m+mi}{0}\PY{p}{,} \PY{n}{Z}\PY{p}{)}


\PY{k}{def}\PY{+w}{ }\PY{n+nf}{softmax}\PY{p}{(}\PY{n}{Z}\PY{p}{)}\PY{p}{:}
\PY{+w}{    }\PY{l+s+sd}{\PYZdq{}\PYZdq{}\PYZdq{}}
\PY{l+s+sd}{    Softmax 激活函數}

\PY{l+s+sd}{    為了數值穩定性(避免因指數運算產生極大值),先減去 Z 中的最大值}
\PY{l+s+sd}{    \PYZdq{}\PYZdq{}\PYZdq{}}
    \PY{c+c1}{\PYZsh{} Z.shape: (N, 10), N 是樣本數}
    \PY{c+c1}{\PYZsh{} np.max(Z, axis=1, keepdims=True) 會找到每個樣本的最大值}
    \PY{c+c1}{\PYZsh{} keepdims=True 確保結果的維度是 (N, 1),這樣才能正確地進行廣播 (broadcasting)}
    \PY{n}{exp\PYZus{}scores} \PY{o}{=} \PY{n}{np}\PY{o}{.}\PY{n}{exp}\PY{p}{(}\PY{n}{Z} \PY{o}{\PYZhy{}} \PY{n}{np}\PY{o}{.}\PY{n}{max}\PY{p}{(}\PY{n}{Z}\PY{p}{,} \PY{n}{axis}\PY{o}{=}\PY{l+m+mi}{1}\PY{p}{,} \PY{n}{keepdims}\PY{o}{=}\PY{k+kc}{True}\PY{p}{)}\PY{p}{)}
    \PY{k}{return} \PY{n}{exp\PYZus{}scores} \PY{o}{/} \PY{n}{np}\PY{o}{.}\PY{n}{sum}\PY{p}{(}\PY{n}{exp\PYZus{}scores}\PY{p}{,} \PY{n}{axis}\PY{o}{=}\PY{l+m+mi}{1}\PY{p}{,} \PY{n}{keepdims}\PY{o}{=}\PY{k+kc}{True}\PY{p}{)}
\end{Verbatim}
\end{tcolorbox}

    \subsection{Step 2-3: Forward
Propagation}\label{step-2-3-forward-propagation}

實作完整的前向傳播過程

參數:

\begin{itemize}
\tightlist
\item
  X -- 輸入資料,維度為 (樣本數, 784)
\item
  parameters -- 包含 W1, b1, W2, b2 的字典(由剛剛實現的
  initialize\_parameters 回傳值可得)
\end{itemize}

Return:

\begin{itemize}
\tightlist
\item
  A2 -- 輸出層的結果 (Softmax 的輸出),維度為 (樣本數, 10)
\item
  cache -- 一個包含 Z1, A1, W1, b1, Z2, A2, W2, b2
  的字典,供反向傳播使用
\end{itemize}

    \begin{tcolorbox}[breakable, size=fbox, boxrule=1pt, pad at break*=1mm,colback=cellbackground, colframe=cellborder]
\prompt{In}{incolor}{12}{\boxspacing}
\begin{Verbatim}[commandchars=\\\{\}]
\PY{k}{def}\PY{+w}{ }\PY{n+nf}{forward\PYZus{}propagation}\PY{p}{(}\PY{n}{X}\PY{p}{,} \PY{n}{parameters}\PY{p}{)}\PY{p}{:}
    \PY{c+c1}{\PYZsh{} 從 parameters 字典中取出權重和 bias}
    \PY{n}{W1} \PY{o}{=} \PY{n}{parameters}\PY{p}{[}\PY{l+s+s2}{\PYZdq{}}\PY{l+s+s2}{W1}\PY{l+s+s2}{\PYZdq{}}\PY{p}{]}
    \PY{n}{b1} \PY{o}{=} \PY{n}{parameters}\PY{p}{[}\PY{l+s+s2}{\PYZdq{}}\PY{l+s+s2}{b1}\PY{l+s+s2}{\PYZdq{}}\PY{p}{]}
    \PY{n}{W2} \PY{o}{=} \PY{n}{parameters}\PY{p}{[}\PY{l+s+s2}{\PYZdq{}}\PY{l+s+s2}{W2}\PY{l+s+s2}{\PYZdq{}}\PY{p}{]}
    \PY{n}{b2} \PY{o}{=} \PY{n}{parameters}\PY{p}{[}\PY{l+s+s2}{\PYZdq{}}\PY{l+s+s2}{b2}\PY{l+s+s2}{\PYZdq{}}\PY{p}{]}

    \PY{c+c1}{\PYZsh{} 1. 從輸入層到隱藏層}
    \PY{n}{Z1} \PY{o}{=} \PY{n}{np}\PY{o}{.}\PY{n}{dot}\PY{p}{(}\PY{n}{X}\PY{p}{,} \PY{n}{W1}\PY{p}{)} \PY{o}{+} \PY{n}{b1}
    \PY{n}{A1} \PY{o}{=} \PY{n}{relu}\PY{p}{(}\PY{n}{Z1}\PY{p}{)}

    \PY{c+c1}{\PYZsh{} 2. 從隱藏層到輸出層}
    \PY{n}{Z2} \PY{o}{=} \PY{n}{np}\PY{o}{.}\PY{n}{dot}\PY{p}{(}\PY{n}{A1}\PY{p}{,} \PY{n}{W2}\PY{p}{)} \PY{o}{+} \PY{n}{b2}
    \PY{n}{A2} \PY{o}{=} \PY{n}{softmax}\PY{p}{(}\PY{n}{Z2}\PY{p}{)}  \PY{c+c1}{\PYZsh{} A2 就是我們的預測機率分佈}

    \PY{c+c1}{\PYZsh{} 儲存計算的中間值,以便反向傳播使用}
    \PY{n}{cache} \PY{o}{=} \PY{p}{\PYZob{}}\PY{l+s+s2}{\PYZdq{}}\PY{l+s+s2}{Z1}\PY{l+s+s2}{\PYZdq{}}\PY{p}{:} \PY{n}{Z1}\PY{p}{,} \PY{l+s+s2}{\PYZdq{}}\PY{l+s+s2}{A1}\PY{l+s+s2}{\PYZdq{}}\PY{p}{:} \PY{n}{A1}\PY{p}{,} \PY{l+s+s2}{\PYZdq{}}\PY{l+s+s2}{Z2}\PY{l+s+s2}{\PYZdq{}}\PY{p}{:} \PY{n}{Z2}\PY{p}{,} \PY{l+s+s2}{\PYZdq{}}\PY{l+s+s2}{A2}\PY{l+s+s2}{\PYZdq{}}\PY{p}{:} \PY{n}{A2}\PY{p}{\PYZcb{}}

    \PY{k}{return} \PY{n}{A2}\PY{p}{,} \PY{n}{cache}
\end{Verbatim}
\end{tcolorbox}

    \subsection{Step 2-4: 損失函數}\label{step-2-4-ux640dux5931ux51fdux6578}

對於 Softmax 輸出的多分類問題,最標準的損失函數是 Cross Entropy Loss。

它的核心思想是:

\begin{itemize}
\tightlist
\item
  如果模型對「正確答案」預測的機率越高,損失就越小。
\item
  如果模型對「正確答案」預測的機率越低(甚至以很高的機率猜測錯誤的答案)損失就越大。
\end{itemize}

公式是把所有樣本的損失加總後取平均:

\(J=-\frac{1}{m}\sum_{i=1}^{m}\sum_{j=1}^{C}y_{ij}\log(\hat{y}_{ij})\)

其中:

\begin{itemize}
\tightlist
\item
  \(m\) 是樣本數量
\item
  \(C\) 是分類數量(以手寫辨識字為例,是 10)
\item
  \(y_{ij}\):第 \(i\) 個樣本\textbf{是否}屬於第 \(j\)
  個分類的正確標籤(One-Hot Encoding),正確為 1 否則為 0
\item
  \(\hat{y}_{ij}\) 是模型對於第 \(i\) 個樣本屬於第 \(j\)
  個分類的預測機率(也就是 \texttt{A2} 的值)
\end{itemize}

\begin{center}\rule{0.5\linewidth}{0.5pt}\end{center}

計算 cross-entropy 損失

參數:

\begin{itemize}
\tightlist
\item
  A2 -- Forward propagation 的輸出 (模型的預測機率),維度為 (樣本數, 10)
\item
  Y -- 正解的標籤 (One-Hot 編碼),維度為 (樣本數, 10)
\end{itemize}

Return:

\begin{itemize}
\tightlist
\item
  loss -- Cross Entropy 損失值 (一個純量)
\end{itemize}

    \begin{tcolorbox}[breakable, size=fbox, boxrule=1pt, pad at break*=1mm,colback=cellbackground, colframe=cellborder]
\prompt{In}{incolor}{13}{\boxspacing}
\begin{Verbatim}[commandchars=\\\{\}]
\PY{k}{def}\PY{+w}{ }\PY{n+nf}{compute\PYZus{}loss}\PY{p}{(}\PY{n}{A2}\PY{p}{,} \PY{n}{Y}\PY{p}{)}\PY{p}{:}
    \PY{c+c1}{\PYZsh{} 獲取樣本數量}
    \PY{n}{m} \PY{o}{=} \PY{n}{Y}\PY{o}{.}\PY{n}{shape}\PY{p}{[}\PY{l+m+mi}{0}\PY{p}{]}

    \PY{c+c1}{\PYZsh{} 計算 Cross\PYZhy{}Entropy 損失}
    \PY{c+c1}{\PYZsh{} 加上一個極小值 1e\PYZhy{}9 是為了避免 log(0) 的情況,確保數值穩定}
    \PY{n}{log\PYZus{}probs} \PY{o}{=} \PY{n}{np}\PY{o}{.}\PY{n}{log}\PY{p}{(}\PY{n}{A2} \PY{o}{+} \PY{l+m+mf}{1e\PYZhy{}9}\PY{p}{)}

    \PY{c+c1}{\PYZsh{} Y 是 one\PYZhy{}hot encoding 後的標籤,所以 Y * log\PYZus{}probs 會巧妙地只選出正確類別的 log(機率)}
    \PY{n}{loss} \PY{o}{=} \PY{o}{\PYZhy{}}\PY{n}{np}\PY{o}{.}\PY{n}{sum}\PY{p}{(}\PY{n}{Y} \PY{o}{*} \PY{n}{log\PYZus{}probs}\PY{p}{)}

    \PY{c+c1}{\PYZsh{} 取平均}
    \PY{n}{cost} \PY{o}{=} \PY{n}{loss} \PY{o}{/} \PY{n}{m}

    \PY{c+c1}{\PYZsh{} np.squeeze() 可以移除多餘的維度,確保結果是純量}
    \PY{n}{cost} \PY{o}{=} \PY{n}{np}\PY{o}{.}\PY{n}{squeeze}\PY{p}{(}\PY{n}{cost}\PY{p}{)}

    \PY{k}{return} \PY{n}{cost}
\end{Verbatim}
\end{tcolorbox}

    \subsection{Step 2-5: Backward
Propagation}\label{step-2-5-backward-propagation}

步驟:

\begin{enumerate}
\def\labelenumi{\arabic{enumi}.}
\tightlist
\item
  計算輸出層的梯度(\texttt{dZ2}): \(dZ_2=A_2-Y\)
\item
  計算 \texttt{W2} 和 \texttt{b2} 的梯度:

  \begin{itemize}
  \tightlist
  \item
    \(dW_2=\frac{1}{m}A^T_1*dZ_2\)
  \item
    \(db_2=\frac{1}{m}\sum_{}^{}{dZ_2}\)
  \end{itemize}
\item
  計算隱藏層的梯度(\texttt{dZ1}):

  \begin{itemize}
  \tightlist
  \item
    \(dZ_1=dZ_2*W_2^T*g'(Z_1)\)
  \item
    這裡的 \(g'(Z_1)\) 是隱藏層激活函數的導數
  \end{itemize}
\item
  計算 \texttt{W1} 和 \texttt{b1} 的梯度:

  \begin{itemize}
  \tightlist
  \item
    \(dW_1=\frac{1}{m}X^T*dZ_1\)
  \item
    \(db_1=\frac{1}{m}\sum_{}^{}{dZ_1}\)
  \end{itemize}
\end{enumerate}

\begin{center}\rule{0.5\linewidth}{0.5pt}\end{center}

實作反向傳播,計算 W1, b1, W2, b2 的梯度

參數:

\begin{itemize}
\tightlist
\item
  parameters -- 包含 W1, W2 的字典
\item
  cache -- 包含 A1, A2, Z1, Z2 的字典,來自前向傳播
\item
  X -- 輸入資料,維度 (樣本數, 784)
\item
  Y -- 正解標籤 (one-hot),維度 (樣本數, 10)
\end{itemize}

Return:

\begin{itemize}
\tightlist
\item
  grads -- 一個包含 dW1, db1, dW2, db2 的梯度字典
\end{itemize}

    \begin{tcolorbox}[breakable, size=fbox, boxrule=1pt, pad at break*=1mm,colback=cellbackground, colframe=cellborder]
\prompt{In}{incolor}{14}{\boxspacing}
\begin{Verbatim}[commandchars=\\\{\}]
\PY{k}{def}\PY{+w}{ }\PY{n+nf}{backward\PYZus{}propagation}\PY{p}{(}\PY{n}{parameters}\PY{p}{,} \PY{n}{cache}\PY{p}{,} \PY{n}{X}\PY{p}{,} \PY{n}{Y}\PY{p}{)}\PY{p}{:}
    \PY{n}{m} \PY{o}{=} \PY{n}{X}\PY{o}{.}\PY{n}{shape}\PY{p}{[}\PY{l+m+mi}{0}\PY{p}{]}  \PY{c+c1}{\PYZsh{} 樣本數量}

    \PY{c+c1}{\PYZsh{} 從 parameters 和 cache 中取出需要的變數}
    \PY{n}{W2} \PY{o}{=} \PY{n}{parameters}\PY{p}{[}\PY{l+s+s2}{\PYZdq{}}\PY{l+s+s2}{W2}\PY{l+s+s2}{\PYZdq{}}\PY{p}{]}
    \PY{n}{A1} \PY{o}{=} \PY{n}{cache}\PY{p}{[}\PY{l+s+s2}{\PYZdq{}}\PY{l+s+s2}{A1}\PY{l+s+s2}{\PYZdq{}}\PY{p}{]}
    \PY{n}{A2} \PY{o}{=} \PY{n}{cache}\PY{p}{[}\PY{l+s+s2}{\PYZdq{}}\PY{l+s+s2}{A2}\PY{l+s+s2}{\PYZdq{}}\PY{p}{]}
    \PY{n}{Z1} \PY{o}{=} \PY{n}{cache}\PY{p}{[}\PY{l+s+s2}{\PYZdq{}}\PY{l+s+s2}{Z1}\PY{l+s+s2}{\PYZdq{}}\PY{p}{]}

    \PY{c+c1}{\PYZsh{} 1. 輸出層的梯度}
    \PY{c+c1}{\PYZsh{} dZ2 的維度是 (m, 10)}
    \PY{n}{dZ2} \PY{o}{=} \PY{n}{A2} \PY{o}{\PYZhy{}} \PY{n}{Y}

    \PY{c+c1}{\PYZsh{} 2. W2 和 b2 的梯度}
    \PY{c+c1}{\PYZsh{} A1.T 的維度是 (128, m), dZ2 的維度是 (m, 10) \PYZhy{}\PYZgt{} dW2 維度是 (128, 10)}
    \PY{n}{dW2} \PY{o}{=} \PY{p}{(}\PY{l+m+mi}{1} \PY{o}{/} \PY{n}{m}\PY{p}{)} \PY{o}{*} \PY{n}{np}\PY{o}{.}\PY{n}{dot}\PY{p}{(}\PY{n}{A1}\PY{o}{.}\PY{n}{T}\PY{p}{,} \PY{n}{dZ2}\PY{p}{)}
    \PY{n}{db2} \PY{o}{=} \PY{p}{(}\PY{l+m+mi}{1} \PY{o}{/} \PY{n}{m}\PY{p}{)} \PY{o}{*} \PY{n}{np}\PY{o}{.}\PY{n}{sum}\PY{p}{(}\PY{n}{dZ2}\PY{p}{,} \PY{n}{axis}\PY{o}{=}\PY{l+m+mi}{0}\PY{p}{,} \PY{n}{keepdims}\PY{o}{=}\PY{k+kc}{True}\PY{p}{)}

    \PY{c+c1}{\PYZsh{} 3. 隱藏層的梯度}
    \PY{c+c1}{\PYZsh{} W2.T 的維度是 (10, 128)}
    \PY{c+c1}{\PYZsh{} dZ2.dot(W2.T) 的維度是 (m, 128)}
    \PY{c+c1}{\PYZsh{} Z1 \PYZgt{} 0 會回傳一個布林矩陣,當作 ReLU 的導數 (True=1, False=0),剛好符合 ReLU 的導數特性}
    \PY{n}{dZ1} \PY{o}{=} \PY{n}{np}\PY{o}{.}\PY{n}{dot}\PY{p}{(}\PY{n}{dZ2}\PY{p}{,} \PY{n}{W2}\PY{o}{.}\PY{n}{T}\PY{p}{)} \PY{o}{*} \PY{p}{(}\PY{n}{Z1} \PY{o}{\PYZgt{}} \PY{l+m+mi}{0}\PY{p}{)}

    \PY{c+c1}{\PYZsh{} 4. W1 和 b1 的梯度}
    \PY{c+c1}{\PYZsh{} X.T 的維度是 (784, m), dZ1 的維度是 (m, 128) \PYZhy{}\PYZgt{} dW1 維度是 (784, 128)}
    \PY{n}{dW1} \PY{o}{=} \PY{p}{(}\PY{l+m+mi}{1} \PY{o}{/} \PY{n}{m}\PY{p}{)} \PY{o}{*} \PY{n}{np}\PY{o}{.}\PY{n}{dot}\PY{p}{(}\PY{n}{X}\PY{o}{.}\PY{n}{T}\PY{p}{,} \PY{n}{dZ1}\PY{p}{)}
    \PY{n}{db1} \PY{o}{=} \PY{p}{(}\PY{l+m+mi}{1} \PY{o}{/} \PY{n}{m}\PY{p}{)} \PY{o}{*} \PY{n}{np}\PY{o}{.}\PY{n}{sum}\PY{p}{(}\PY{n}{dZ1}\PY{p}{,} \PY{n}{axis}\PY{o}{=}\PY{l+m+mi}{0}\PY{p}{,} \PY{n}{keepdims}\PY{o}{=}\PY{k+kc}{True}\PY{p}{)}

    \PY{c+c1}{\PYZsh{} 將所有梯度打包到一個字典中}
    \PY{n}{grads} \PY{o}{=} \PY{p}{\PYZob{}}\PY{l+s+s2}{\PYZdq{}}\PY{l+s+s2}{dW1}\PY{l+s+s2}{\PYZdq{}}\PY{p}{:} \PY{n}{dW1}\PY{p}{,} \PY{l+s+s2}{\PYZdq{}}\PY{l+s+s2}{db1}\PY{l+s+s2}{\PYZdq{}}\PY{p}{:} \PY{n}{db1}\PY{p}{,} \PY{l+s+s2}{\PYZdq{}}\PY{l+s+s2}{dW2}\PY{l+s+s2}{\PYZdq{}}\PY{p}{:} \PY{n}{dW2}\PY{p}{,} \PY{l+s+s2}{\PYZdq{}}\PY{l+s+s2}{db2}\PY{l+s+s2}{\PYZdq{}}\PY{p}{:} \PY{n}{db2}\PY{p}{\PYZcb{}}

    \PY{k}{return} \PY{n}{grads}
\end{Verbatim}
\end{tcolorbox}

    \subsection{Step 2-6: 更新權重}\label{step-2-6-ux66f4ux65b0ux6b0aux91cd}

使用梯度下降法更新模型的參數

參數:

\begin{itemize}
\tightlist
\item
  parameters -- 包含 W1, b1, W2, b2 的字典
\item
  grads -- 包含 dW1, db1, dW2, db2 的梯度字典
\item
  learning\_rate -- 學習率 (alpha),一個純量
\end{itemize}

Return:

\begin{itemize}
\tightlist
\item
  parameters -- 更新後的參數字典
\end{itemize}

    \begin{tcolorbox}[breakable, size=fbox, boxrule=1pt, pad at break*=1mm,colback=cellbackground, colframe=cellborder]
\prompt{In}{incolor}{15}{\boxspacing}
\begin{Verbatim}[commandchars=\\\{\}]
\PY{k}{def}\PY{+w}{ }\PY{n+nf}{update\PYZus{}parameters}\PY{p}{(}\PY{n}{parameters}\PY{p}{,} \PY{n}{grads}\PY{p}{,} \PY{n}{learning\PYZus{}rate}\PY{p}{)}\PY{p}{:}
    \PY{c+c1}{\PYZsh{} 從字典中取出參數}
    \PY{n}{W1} \PY{o}{=} \PY{n}{parameters}\PY{p}{[}\PY{l+s+s2}{\PYZdq{}}\PY{l+s+s2}{W1}\PY{l+s+s2}{\PYZdq{}}\PY{p}{]}
    \PY{n}{b1} \PY{o}{=} \PY{n}{parameters}\PY{p}{[}\PY{l+s+s2}{\PYZdq{}}\PY{l+s+s2}{b1}\PY{l+s+s2}{\PYZdq{}}\PY{p}{]}
    \PY{n}{W2} \PY{o}{=} \PY{n}{parameters}\PY{p}{[}\PY{l+s+s2}{\PYZdq{}}\PY{l+s+s2}{W2}\PY{l+s+s2}{\PYZdq{}}\PY{p}{]}
    \PY{n}{b2} \PY{o}{=} \PY{n}{parameters}\PY{p}{[}\PY{l+s+s2}{\PYZdq{}}\PY{l+s+s2}{b2}\PY{l+s+s2}{\PYZdq{}}\PY{p}{]}

    \PY{c+c1}{\PYZsh{} 從字典中取出梯度}
    \PY{n}{dW1} \PY{o}{=} \PY{n}{grads}\PY{p}{[}\PY{l+s+s2}{\PYZdq{}}\PY{l+s+s2}{dW1}\PY{l+s+s2}{\PYZdq{}}\PY{p}{]}
    \PY{n}{db1} \PY{o}{=} \PY{n}{grads}\PY{p}{[}\PY{l+s+s2}{\PYZdq{}}\PY{l+s+s2}{db1}\PY{l+s+s2}{\PYZdq{}}\PY{p}{]}
    \PY{n}{dW2} \PY{o}{=} \PY{n}{grads}\PY{p}{[}\PY{l+s+s2}{\PYZdq{}}\PY{l+s+s2}{dW2}\PY{l+s+s2}{\PYZdq{}}\PY{p}{]}
    \PY{n}{db2} \PY{o}{=} \PY{n}{grads}\PY{p}{[}\PY{l+s+s2}{\PYZdq{}}\PY{l+s+s2}{db2}\PY{l+s+s2}{\PYZdq{}}\PY{p}{]}

    \PY{c+c1}{\PYZsh{} 根據梯度下降規則更新每一個參數}
    \PY{n}{W1} \PY{o}{=} \PY{n}{W1} \PY{o}{\PYZhy{}} \PY{n}{learning\PYZus{}rate} \PY{o}{*} \PY{n}{dW1}
    \PY{n}{b1} \PY{o}{=} \PY{n}{b1} \PY{o}{\PYZhy{}} \PY{n}{learning\PYZus{}rate} \PY{o}{*} \PY{n}{db1}
    \PY{n}{W2} \PY{o}{=} \PY{n}{W2} \PY{o}{\PYZhy{}} \PY{n}{learning\PYZus{}rate} \PY{o}{*} \PY{n}{dW2}
    \PY{n}{b2} \PY{o}{=} \PY{n}{b2} \PY{o}{\PYZhy{}} \PY{n}{learning\PYZus{}rate} \PY{o}{*} \PY{n}{db2}

    \PY{c+c1}{\PYZsh{} 將更新後的參數存回字典}
    \PY{n}{updated\PYZus{}parameters} \PY{o}{=} \PY{p}{\PYZob{}}\PY{l+s+s2}{\PYZdq{}}\PY{l+s+s2}{W1}\PY{l+s+s2}{\PYZdq{}}\PY{p}{:} \PY{n}{W1}\PY{p}{,} \PY{l+s+s2}{\PYZdq{}}\PY{l+s+s2}{b1}\PY{l+s+s2}{\PYZdq{}}\PY{p}{:} \PY{n}{b1}\PY{p}{,} \PY{l+s+s2}{\PYZdq{}}\PY{l+s+s2}{W2}\PY{l+s+s2}{\PYZdq{}}\PY{p}{:} \PY{n}{W2}\PY{p}{,} \PY{l+s+s2}{\PYZdq{}}\PY{l+s+s2}{b2}\PY{l+s+s2}{\PYZdq{}}\PY{p}{:} \PY{n}{b2}\PY{p}{\PYZcb{}}

    \PY{k}{return} \PY{n}{updated\PYZus{}parameters}
\end{Verbatim}
\end{tcolorbox}

    \subsection{Step 2-7: 組合成
nn\_model}\label{step-2-7-ux7d44ux5408ux6210-nn_model}

參數:

\begin{itemize}
\tightlist
\item
  X -- 訓練資料 (60000, 784)
\item
  Y -- 訓練標籤 (60000, 10)
\item
  n\_h -- 隱藏層大小
\item
  learning\_rate -- 學習率
\item
  num\_epochs -- 訓練的世代數
\item
  batch\_size -- 每個小批次的大小
\end{itemize}

Return:

\begin{itemize}
\tightlist
\item
  parameters -- 訓練完成後的模型參數
\end{itemize}

    \begin{tcolorbox}[breakable, size=fbox, boxrule=1pt, pad at break*=1mm,colback=cellbackground, colframe=cellborder]
\prompt{In}{incolor}{16}{\boxspacing}
\begin{Verbatim}[commandchars=\\\{\}]
\PY{k}{def}\PY{+w}{ }\PY{n+nf}{nn\PYZus{}model}\PY{p}{(}\PY{n}{X}\PY{p}{,} \PY{n}{Y}\PY{p}{,} \PY{n}{n\PYZus{}h}\PY{p}{,} \PY{n}{learning\PYZus{}rate}\PY{o}{=}\PY{l+m+mf}{0.1}\PY{p}{,} \PY{n}{num\PYZus{}epochs}\PY{o}{=}\PY{l+m+mi}{10}\PY{p}{,} \PY{n}{batch\PYZus{}size}\PY{o}{=}\PY{l+m+mi}{128}\PY{p}{)}\PY{p}{:}
    \PY{n}{n\PYZus{}x} \PY{o}{=} \PY{n}{X}\PY{o}{.}\PY{n}{shape}\PY{p}{[}\PY{l+m+mi}{1}\PY{p}{]}
    \PY{n}{n\PYZus{}y} \PY{o}{=} \PY{n}{Y}\PY{o}{.}\PY{n}{shape}\PY{p}{[}\PY{l+m+mi}{1}\PY{p}{]}
    \PY{n}{m} \PY{o}{=} \PY{n}{X}\PY{o}{.}\PY{n}{shape}\PY{p}{[}\PY{l+m+mi}{0}\PY{p}{]}

    \PY{n}{parameters} \PY{o}{=} \PY{n}{initialize\PYZus{}parameters}\PY{p}{(}\PY{n}{n\PYZus{}x}\PY{p}{,} \PY{n}{n\PYZus{}h}\PY{p}{,} \PY{n}{n\PYZus{}y}\PY{p}{)}

    \PY{c+c1}{\PYZsh{} 儲存每個 epoch 的 loss,用於後續繪圖}
    \PY{n}{costs} \PY{o}{=} \PY{p}{[}\PY{p}{]}

    \PY{c+c1}{\PYZsh{} 2. 訓練迴圈}
    \PY{k}{for} \PY{n}{i} \PY{o+ow}{in} \PY{n+nb}{range}\PY{p}{(}\PY{n}{num\PYZus{}epochs}\PY{p}{)}\PY{p}{:}
        \PY{n}{epoch\PYZus{}cost} \PY{o}{=} \PY{l+m+mf}{0.0}

        \PY{c+c1}{\PYZsh{} 將資料隨機打亂,這有助於訓練}
        \PY{n}{permutation} \PY{o}{=} \PY{n}{np}\PY{o}{.}\PY{n}{random}\PY{o}{.}\PY{n}{permutation}\PY{p}{(}\PY{n}{m}\PY{p}{)}
        \PY{n}{shuffled\PYZus{}X} \PY{o}{=} \PY{n}{X}\PY{p}{[}\PY{n}{permutation}\PY{p}{,} \PY{p}{:}\PY{p}{]}
        \PY{n}{shuffled\PYZus{}Y} \PY{o}{=} \PY{n}{Y}\PY{p}{[}\PY{n}{permutation}\PY{p}{,} \PY{p}{:}\PY{p}{]}

        \PY{c+c1}{\PYZsh{} Mini\PYZhy{}batch 處理}
        \PY{n}{num\PYZus{}minibatches} \PY{o}{=} \PY{n}{m} \PY{o}{/}\PY{o}{/} \PY{n}{batch\PYZus{}size}
        \PY{k}{for} \PY{n}{j} \PY{o+ow}{in} \PY{n+nb}{range}\PY{p}{(}\PY{n}{num\PYZus{}minibatches}\PY{p}{)}\PY{p}{:}
            \PY{n}{start} \PY{o}{=} \PY{n}{j} \PY{o}{*} \PY{n}{batch\PYZus{}size}
            \PY{n}{end} \PY{o}{=} \PY{n}{start} \PY{o}{+} \PY{n}{batch\PYZus{}size}

            \PY{n}{minibatch\PYZus{}X} \PY{o}{=} \PY{n}{shuffled\PYZus{}X}\PY{p}{[}\PY{n}{start}\PY{p}{:}\PY{n}{end}\PY{p}{,} \PY{p}{:}\PY{p}{]}
            \PY{n}{minibatch\PYZus{}Y} \PY{o}{=} \PY{n}{shuffled\PYZus{}Y}\PY{p}{[}\PY{n}{start}\PY{p}{:}\PY{n}{end}\PY{p}{,} \PY{p}{:}\PY{p}{]}

            \PY{c+c1}{\PYZsh{} \PYZhy{}\PYZhy{}\PYZhy{} 單次學習週期 \PYZhy{}\PYZhy{}\PYZhy{}}
            \PY{c+c1}{\PYZsh{} a. 前向傳播}
            \PY{n}{A2}\PY{p}{,} \PY{n}{cache} \PY{o}{=} \PY{n}{forward\PYZus{}propagation}\PY{p}{(}\PY{n}{minibatch\PYZus{}X}\PY{p}{,} \PY{n}{parameters}\PY{p}{)}

            \PY{c+c1}{\PYZsh{} b. 計算損失}
            \PY{n}{cost} \PY{o}{=} \PY{n}{compute\PYZus{}loss}\PY{p}{(}\PY{n}{A2}\PY{p}{,} \PY{n}{minibatch\PYZus{}Y}\PY{p}{)}
            \PY{n}{epoch\PYZus{}cost} \PY{o}{+}\PY{o}{=} \PY{n}{cost}

            \PY{c+c1}{\PYZsh{} c. 反向傳播}
            \PY{n}{grads} \PY{o}{=} \PY{n}{backward\PYZus{}propagation}\PY{p}{(}\PY{n}{parameters}\PY{p}{,} \PY{n}{cache}\PY{p}{,} \PY{n}{minibatch\PYZus{}X}\PY{p}{,} \PY{n}{minibatch\PYZus{}Y}\PY{p}{)}

            \PY{c+c1}{\PYZsh{} d. 更新參數}
            \PY{n}{parameters} \PY{o}{=} \PY{n}{update\PYZus{}parameters}\PY{p}{(}\PY{n}{parameters}\PY{p}{,} \PY{n}{grads}\PY{p}{,} \PY{n}{learning\PYZus{}rate}\PY{p}{)}

        \PY{c+c1}{\PYZsh{} 印出每個 epoch 的平均損失}
        \PY{n}{avg\PYZus{}epoch\PYZus{}cost} \PY{o}{=} \PY{n}{epoch\PYZus{}cost} \PY{o}{/} \PY{n}{num\PYZus{}minibatches}
        \PY{n}{costs}\PY{o}{.}\PY{n}{append}\PY{p}{(}\PY{n}{avg\PYZus{}epoch\PYZus{}cost}\PY{p}{)}
        \PY{n+nb}{print}\PY{p}{(}\PY{l+s+sa}{f}\PY{l+s+s2}{\PYZdq{}}\PY{l+s+s2}{Epoch }\PY{l+s+si}{\PYZob{}}\PY{n}{i}\PY{+w}{ }\PY{o}{+}\PY{+w}{ }\PY{l+m+mi}{1}\PY{l+s+si}{\PYZcb{}}\PY{l+s+s2}{/}\PY{l+s+si}{\PYZob{}}\PY{n}{num\PYZus{}epochs}\PY{l+s+si}{\PYZcb{}}\PY{l+s+s2}{ \PYZhy{} Cost: }\PY{l+s+si}{\PYZob{}}\PY{n}{avg\PYZus{}epoch\PYZus{}cost}\PY{l+s+si}{:}\PY{l+s+s2}{.6f}\PY{l+s+si}{\PYZcb{}}\PY{l+s+s2}{\PYZdq{}}\PY{p}{)}

    \PY{k}{return} \PY{n}{parameters}\PY{p}{,} \PY{n}{costs}
\end{Verbatim}
\end{tcolorbox}

    \section{Step 3:
訓練與評估}\label{step-3-ux8a13ux7df4ux8207ux8a55ux4f30}

    \subsection{Step 3-1: 實作 predict
函數}\label{step-3-1-ux5be6ux4f5c-predict-ux51fdux6578}

參數:

\begin{itemize}
\tightlist
\item
  X -- 要預測的資料集,維度 (樣本數, 784)
\item
  parameters -- 訓練好的參數字典
\end{itemize}

Return:

\begin{itemize}
\tightlist
\item
  predictions -- 預測的標籤,維度 (樣本數,)
\end{itemize}

    \begin{tcolorbox}[breakable, size=fbox, boxrule=1pt, pad at break*=1mm,colback=cellbackground, colframe=cellborder]
\prompt{In}{incolor}{17}{\boxspacing}
\begin{Verbatim}[commandchars=\\\{\}]
\PY{k}{def}\PY{+w}{ }\PY{n+nf}{predict}\PY{p}{(}\PY{n}{X}\PY{p}{,} \PY{n}{parameters}\PY{p}{)}\PY{p}{:}
    \PY{c+c1}{\PYZsh{} 執行前向傳播}
    \PY{n}{A2}\PY{p}{,} \PY{n}{\PYZus{}} \PY{o}{=} \PY{n}{forward\PYZus{}propagation}\PY{p}{(}\PY{n}{X}\PY{p}{,} \PY{n}{parameters}\PY{p}{)}

    \PY{c+c1}{\PYZsh{} 找出每個樣本中,機率最高的那個類別的索引 (0\PYZhy{}9)}
    \PY{c+c1}{\PYZsh{} np.argmax(..., axis=1) 會回傳每一行最大值的索引}
    \PY{n}{predictions} \PY{o}{=} \PY{n}{np}\PY{o}{.}\PY{n}{argmax}\PY{p}{(}\PY{n}{A2}\PY{p}{,} \PY{n}{axis}\PY{o}{=}\PY{l+m+mi}{1}\PY{p}{)}

    \PY{k}{return} \PY{n}{predictions}
\end{Verbatim}
\end{tcolorbox}

    \subsection{Step 3-2: 訓練模型}\label{step-3-2-ux8a13ux7df4ux6a21ux578b}

    \begin{tcolorbox}[breakable, size=fbox, boxrule=1pt, pad at break*=1mm,colback=cellbackground, colframe=cellborder]
\prompt{In}{incolor}{18}{\boxspacing}
\begin{Verbatim}[commandchars=\\\{\}]
\PY{n}{trained\PYZus{}parameters}\PY{p}{,} \PY{n}{costs} \PY{o}{=} \PY{n}{nn\PYZus{}model}\PY{p}{(}
    \PY{n}{train\PYZus{}x}\PY{p}{,}
    \PY{n}{train\PYZus{}y}\PY{p}{,}
    \PY{n}{n\PYZus{}h}\PY{o}{=}\PY{l+m+mi}{256}\PY{p}{,}
    \PY{n}{num\PYZus{}epochs}\PY{o}{=}\PY{l+m+mi}{50}\PY{p}{,}
    \PY{n}{batch\PYZus{}size}\PY{o}{=}\PY{l+m+mi}{128}\PY{p}{,}
\PY{p}{)}
\end{Verbatim}
\end{tcolorbox}

    \begin{Verbatim}[commandchars=\\\{\}]
Epoch 1/50 - Cost: 0.711293
Epoch 2/50 - Cost: 0.310791
Epoch 3/50 - Cost: 0.256582
Epoch 4/50 - Cost: 0.216052
Epoch 5/50 - Cost: 0.186137
Epoch 6/50 - Cost: 0.163056
Epoch 7/50 - Cost: 0.144624
Epoch 8/50 - Cost: 0.129664
Epoch 9/50 - Cost: 0.117931
Epoch 10/50 - Cost: 0.107521
Epoch 11/50 - Cost: 0.098650
Epoch 12/50 - Cost: 0.091640
Epoch 13/50 - Cost: 0.085038
Epoch 14/50 - Cost: 0.079125
Epoch 15/50 - Cost: 0.074194
Epoch 16/50 - Cost: 0.069123
Epoch 17/50 - Cost: 0.065229
Epoch 18/50 - Cost: 0.061586
Epoch 19/50 - Cost: 0.057991
Epoch 20/50 - Cost: 0.055033
Epoch 21/50 - Cost: 0.051809
Epoch 22/50 - Cost: 0.049203
Epoch 23/50 - Cost: 0.046869
Epoch 24/50 - Cost: 0.044499
Epoch 25/50 - Cost: 0.042497
Epoch 26/50 - Cost: 0.040451
Epoch 27/50 - Cost: 0.038547
Epoch 28/50 - Cost: 0.036872
Epoch 29/50 - Cost: 0.035224
Epoch 30/50 - Cost: 0.033556
Epoch 31/50 - Cost: 0.032152
Epoch 32/50 - Cost: 0.030781
Epoch 33/50 - Cost: 0.029663
Epoch 34/50 - Cost: 0.028193
Epoch 35/50 - Cost: 0.027238
Epoch 36/50 - Cost: 0.025979
Epoch 37/50 - Cost: 0.024899
Epoch 38/50 - Cost: 0.023973
Epoch 39/50 - Cost: 0.022911
Epoch 40/50 - Cost: 0.022065
Epoch 41/50 - Cost: 0.021322
Epoch 42/50 - Cost: 0.020312
Epoch 43/50 - Cost: 0.019671
Epoch 44/50 - Cost: 0.018996
Epoch 45/50 - Cost: 0.018260
Epoch 46/50 - Cost: 0.017636
Epoch 47/50 - Cost: 0.016940
Epoch 48/50 - Cost: 0.016291
Epoch 49/50 - Cost: 0.015741
Epoch 50/50 - Cost: 0.015181
    \end{Verbatim}

    \begin{tcolorbox}[breakable, size=fbox, boxrule=1pt, pad at break*=1mm,colback=cellbackground, colframe=cellborder]
\prompt{In}{incolor}{19}{\boxspacing}
\begin{Verbatim}[commandchars=\\\{\}]
\PY{k+kn}{import}\PY{+w}{ }\PY{n+nn}{matplotlib}\PY{n+nn}{.}\PY{n+nn}{pyplot}\PY{+w}{ }\PY{k}{as}\PY{+w}{ }\PY{n+nn}{plt}
\PY{k+kn}{import}\PY{+w}{ }\PY{n+nn}{numpy}\PY{+w}{ }\PY{k}{as}\PY{+w}{ }\PY{n+nn}{np}

\PY{n}{plt}\PY{o}{.}\PY{n}{figure}\PY{p}{(}\PY{n}{figsize}\PY{o}{=}\PY{p}{(}\PY{l+m+mi}{10}\PY{p}{,} \PY{l+m+mi}{6}\PY{p}{)}\PY{p}{)}
\PY{n}{plt}\PY{o}{.}\PY{n}{plot}\PY{p}{(}\PY{n}{costs}\PY{p}{)}
\PY{n}{plt}\PY{o}{.}\PY{n}{title}\PY{p}{(}\PY{l+s+s2}{\PYZdq{}}\PY{l+s+s2}{Learning Curve: Cost vs. Epochs}\PY{l+s+s2}{\PYZdq{}}\PY{p}{)}
\PY{n}{plt}\PY{o}{.}\PY{n}{xlabel}\PY{p}{(}\PY{l+s+s2}{\PYZdq{}}\PY{l+s+s2}{Epoch}\PY{l+s+s2}{\PYZdq{}}\PY{p}{)}
\PY{n}{plt}\PY{o}{.}\PY{n}{ylabel}\PY{p}{(}\PY{l+s+s2}{\PYZdq{}}\PY{l+s+s2}{Cost}\PY{l+s+s2}{\PYZdq{}}\PY{p}{)}
\PY{n}{plt}\PY{o}{.}\PY{n}{grid}\PY{p}{(}\PY{k+kc}{True}\PY{p}{)}
\PY{n}{plt}\PY{o}{.}\PY{n}{show}\PY{p}{(}\PY{p}{)}
\end{Verbatim}
\end{tcolorbox}

    \begin{center}
    \adjustimage{max size={0.9\linewidth}{0.9\paperheight}}{output_35_0.png}
    \end{center}
    { \hspace*{\fill} \\}
    
    \subsection{Step 3-3:
用測試集檢驗}\label{step-3-3-ux7528ux6e2cux8a66ux96c6ux6aa2ux9a57}

    \begin{tcolorbox}[breakable, size=fbox, boxrule=1pt, pad at break*=1mm,colback=cellbackground, colframe=cellborder]
\prompt{In}{incolor}{20}{\boxspacing}
\begin{Verbatim}[commandchars=\\\{\}]
\PY{c+c1}{\PYZsh{} 對測試集進行預測}
\PY{n}{test\PYZus{}predictions} \PY{o}{=} \PY{n}{predict}\PY{p}{(}\PY{n}{test\PYZus{}x}\PY{p}{,} \PY{n}{trained\PYZus{}parameters}\PY{p}{)}

\PY{c+c1}{\PYZsh{} test\PYZus{}y\PYZus{}orig 是原始的、非 one\PYZhy{}hot 的標籤 (0, 1, 2...)}
\PY{c+c1}{\PYZsh{} 比較預測結果和真實標籤}
\PY{n}{accuracy} \PY{o}{=} \PY{n}{np}\PY{o}{.}\PY{n}{mean}\PY{p}{(}\PY{n}{test\PYZus{}predictions} \PY{o}{==} \PY{n}{test\PYZus{}y\PYZus{}orig}\PY{p}{)} \PY{o}{*} \PY{l+m+mi}{100}

\PY{n+nb}{print}\PY{p}{(}\PY{l+s+sa}{f}\PY{l+s+s2}{\PYZdq{}}\PY{l+s+s2}{模型在測試集上的準確率 (Accuracy): }\PY{l+s+si}{\PYZob{}}\PY{n}{accuracy}\PY{l+s+si}{:}\PY{l+s+s2}{.2f}\PY{l+s+si}{\PYZcb{}}\PY{l+s+s2}{\PYZpc{}}\PY{l+s+s2}{\PYZdq{}}\PY{p}{)}
\end{Verbatim}
\end{tcolorbox}

    \begin{Verbatim}[commandchars=\\\{\}]
模型在測試集上的準確率 (Accuracy): 98.03\%
    \end{Verbatim}

    \subsection{Step 3-4: 計算 Precision 與
Recall}\label{step-3-4-ux8a08ux7b97-precision-ux8207-recall}

    \begin{tcolorbox}[breakable, size=fbox, boxrule=1pt, pad at break*=1mm,colback=cellbackground, colframe=cellborder]
\prompt{In}{incolor}{21}{\boxspacing}
\begin{Verbatim}[commandchars=\\\{\}]
\PY{n}{num\PYZus{}classes} \PY{o}{=} \PY{l+m+mi}{10}
\PY{c+c1}{\PYZsh{} 建立一個 10x10 的矩陣,用來存放混淆矩陣的結果}
\PY{n}{confusion\PYZus{}matrix} \PY{o}{=} \PY{n}{np}\PY{o}{.}\PY{n}{zeros}\PY{p}{(}\PY{p}{(}\PY{n}{num\PYZus{}classes}\PY{p}{,} \PY{n}{num\PYZus{}classes}\PY{p}{)}\PY{p}{,} \PY{n}{dtype}\PY{o}{=}\PY{n+nb}{int}\PY{p}{)}

\PY{c+c1}{\PYZsh{} 遍歷所有測試集樣本,填充混淆矩陣}
\PY{k}{for} \PY{n}{i} \PY{o+ow}{in} \PY{n+nb}{range}\PY{p}{(}\PY{n+nb}{len}\PY{p}{(}\PY{n}{test\PYZus{}y\PYZus{}orig}\PY{p}{)}\PY{p}{)}\PY{p}{:}
    \PY{n}{true\PYZus{}label} \PY{o}{=} \PY{n}{test\PYZus{}y\PYZus{}orig}\PY{p}{[}\PY{n}{i}\PY{p}{]}
    \PY{n}{predicted\PYZus{}label} \PY{o}{=} \PY{n}{test\PYZus{}predictions}\PY{p}{[}\PY{n}{i}\PY{p}{]}
    \PY{n}{confusion\PYZus{}matrix}\PY{p}{[}\PY{n}{true\PYZus{}label}\PY{p}{,} \PY{n}{predicted\PYZus{}label}\PY{p}{]} \PY{o}{+}\PY{o}{=} \PY{l+m+mi}{1}

\PY{n+nb}{print}\PY{p}{(}\PY{l+s+s2}{\PYZdq{}}\PY{l+s+s2}{混淆矩陣 (Confusion Matrix):}\PY{l+s+s2}{\PYZdq{}}\PY{p}{)}
\PY{n+nb}{print}\PY{p}{(}\PY{n}{confusion\PYZus{}matrix}\PY{p}{)}
\PY{n+nb}{print}\PY{p}{(}\PY{l+s+s2}{\PYZdq{}}\PY{l+s+se}{\PYZbs{}n}\PY{l+s+s2}{\PYZdq{}} \PY{o}{+} \PY{l+s+s2}{\PYZdq{}}\PY{l+s+s2}{=}\PY{l+s+s2}{\PYZdq{}} \PY{o}{*} \PY{l+m+mi}{30}\PY{p}{)}
\end{Verbatim}
\end{tcolorbox}

    \begin{Verbatim}[commandchars=\\\{\}]
混淆矩陣 (Confusion Matrix):
[[ 970    0    0    0    1    1    3    1    3    1]
 [   0 1123    3    1    0    1    2    1    4    0]
 [   4    2 1010    3    3    0    1    4    5    0]
 [   0    0    5  990    0    5    0    2    4    4]
 [   1    0    3    0  965    0    2    3    0    8]
 [   3    0    0    7    0  871    4    1    5    1]
 [   4    3    1    1    3    3  941    0    2    0]
 [   1    3    8    1    0    0    0 1006    4    5]
 [   5    0    3    3    3    1    3    2  952    2]
 [   1    2    0    6   10    3    1    5    6  975]]

==============================
    \end{Verbatim}

    \begin{tcolorbox}[breakable, size=fbox, boxrule=1pt, pad at break*=1mm,colback=cellbackground, colframe=cellborder]
\prompt{In}{incolor}{22}{\boxspacing}
\begin{Verbatim}[commandchars=\\\{\}]
\PY{k+kn}{import}\PY{+w}{ }\PY{n+nn}{seaborn}\PY{+w}{ }\PY{k}{as}\PY{+w}{ }\PY{n+nn}{sns}

\PY{n}{plt}\PY{o}{.}\PY{n}{figure}\PY{p}{(}\PY{n}{figsize}\PY{o}{=}\PY{p}{(}\PY{l+m+mi}{10}\PY{p}{,} \PY{l+m+mi}{8}\PY{p}{)}\PY{p}{)}
\PY{n}{sns}\PY{o}{.}\PY{n}{heatmap}\PY{p}{(}
    \PY{n}{confusion\PYZus{}matrix}\PY{p}{,}
    \PY{n}{annot}\PY{o}{=}\PY{k+kc}{True}\PY{p}{,}  \PY{c+c1}{\PYZsh{} 在格子中顯示數字}
    \PY{n}{fmt}\PY{o}{=}\PY{l+s+s2}{\PYZdq{}}\PY{l+s+s2}{d}\PY{l+s+s2}{\PYZdq{}}\PY{p}{,}  \PY{c+c1}{\PYZsh{} 將數字格式化為整數}
    \PY{n}{cmap}\PY{o}{=}\PY{l+s+s2}{\PYZdq{}}\PY{l+s+s2}{Blues}\PY{l+s+s2}{\PYZdq{}}\PY{p}{,}  \PY{c+c1}{\PYZsh{} 使用藍色系的色盤}
    \PY{n}{xticklabels}\PY{o}{=}\PY{n+nb}{range}\PY{p}{(}\PY{l+m+mi}{10}\PY{p}{)}\PY{p}{,} \PY{c+c1}{\PYZsh{} type: ignore}
    \PY{n}{yticklabels}\PY{o}{=}\PY{n+nb}{range}\PY{p}{(}\PY{l+m+mi}{10}\PY{p}{)}\PY{p}{,} \PY{c+c1}{\PYZsh{} type: ignore}
\PY{p}{)}
\PY{n}{plt}\PY{o}{.}\PY{n}{title}\PY{p}{(}\PY{l+s+s2}{\PYZdq{}}\PY{l+s+s2}{Confusion Matrix Heatmap}\PY{l+s+s2}{\PYZdq{}}\PY{p}{)}
\PY{n}{plt}\PY{o}{.}\PY{n}{xlabel}\PY{p}{(}\PY{l+s+s2}{\PYZdq{}}\PY{l+s+s2}{Predicted Label}\PY{l+s+s2}{\PYZdq{}}\PY{p}{)}
\PY{n}{plt}\PY{o}{.}\PY{n}{ylabel}\PY{p}{(}\PY{l+s+s2}{\PYZdq{}}\PY{l+s+s2}{True Label}\PY{l+s+s2}{\PYZdq{}}\PY{p}{)}
\PY{n}{plt}\PY{o}{.}\PY{n}{show}\PY{p}{(}\PY{p}{)}
\end{Verbatim}
\end{tcolorbox}

    \begin{center}
    \adjustimage{max size={0.9\linewidth}{0.9\paperheight}}{output_40_0.png}
    \end{center}
    { \hspace*{\fill} \\}
    
    \begin{tcolorbox}[breakable, size=fbox, boxrule=1pt, pad at break*=1mm,colback=cellbackground, colframe=cellborder]
\prompt{In}{incolor}{23}{\boxspacing}
\begin{Verbatim}[commandchars=\\\{\}]
\PY{k}{for} \PY{n}{i} \PY{o+ow}{in} \PY{n+nb}{range}\PY{p}{(}\PY{n}{num\PYZus{}classes}\PY{p}{)}\PY{p}{:}
    \PY{n}{true\PYZus{}positives} \PY{o}{=} \PY{n}{confusion\PYZus{}matrix}\PY{p}{[}\PY{n}{i}\PY{p}{,} \PY{n}{i}\PY{p}{]}
    \PY{n}{false\PYZus{}positives} \PY{o}{=} \PY{n}{np}\PY{o}{.}\PY{n}{sum}\PY{p}{(}\PY{n}{confusion\PYZus{}matrix}\PY{p}{[}\PY{p}{:}\PY{p}{,} \PY{n}{i}\PY{p}{]}\PY{p}{)} \PY{o}{\PYZhy{}} \PY{n}{true\PYZus{}positives}
    \PY{n}{false\PYZus{}negatives} \PY{o}{=} \PY{n}{np}\PY{o}{.}\PY{n}{sum}\PY{p}{(}\PY{n}{confusion\PYZus{}matrix}\PY{p}{[}\PY{n}{i}\PY{p}{,} \PY{p}{:}\PY{p}{]}\PY{p}{)} \PY{o}{\PYZhy{}} \PY{n}{true\PYZus{}positives}

    \PY{c+c1}{\PYZsh{} 避免分母為零的情況}
    \PY{n}{precision} \PY{o}{=} \PY{n}{true\PYZus{}positives} \PY{o}{/} \PY{p}{(}\PY{n}{true\PYZus{}positives} \PY{o}{+} \PY{n}{false\PYZus{}positives}\PY{p}{)} \PY{k}{if} \PY{p}{(}\PY{n}{true\PYZus{}positives} \PY{o}{+} \PY{n}{false\PYZus{}positives}\PY{p}{)} \PY{o}{\PYZgt{}} \PY{l+m+mi}{0} \PY{k}{else} \PY{l+m+mi}{0}
    \PY{n}{recall} \PY{o}{=} \PY{n}{true\PYZus{}positives} \PY{o}{/} \PY{p}{(}\PY{n}{true\PYZus{}positives} \PY{o}{+} \PY{n}{false\PYZus{}negatives}\PY{p}{)} \PY{k}{if} \PY{p}{(}\PY{n}{true\PYZus{}positives} \PY{o}{+} \PY{n}{false\PYZus{}negatives}\PY{p}{)} \PY{o}{\PYZgt{}} \PY{l+m+mi}{0} \PY{k}{else} \PY{l+m+mi}{0}

    \PY{n+nb}{print}\PY{p}{(}\PY{l+s+sa}{f}\PY{l+s+s2}{\PYZdq{}}\PY{l+s+s2}{數字 }\PY{l+s+s2}{\PYZsq{}}\PY{l+s+si}{\PYZob{}}\PY{n}{i}\PY{l+s+si}{\PYZcb{}}\PY{l+s+s2}{\PYZsq{}}\PY{l+s+s2}{:}\PY{l+s+s2}{\PYZdq{}}\PY{p}{)}
    \PY{n+nb}{print}\PY{p}{(}\PY{l+s+sa}{f}\PY{l+s+s2}{\PYZdq{}}\PY{l+s+s2}{  精確率 (Precision): }\PY{l+s+si}{\PYZob{}}\PY{n}{precision}\PY{l+s+si}{:}\PY{l+s+s2}{.4f}\PY{l+s+si}{\PYZcb{}}\PY{l+s+s2}{\PYZdq{}}\PY{p}{)}
    \PY{n+nb}{print}\PY{p}{(}\PY{l+s+sa}{f}\PY{l+s+s2}{\PYZdq{}}\PY{l+s+s2}{  召回率 (Recall):    }\PY{l+s+si}{\PYZob{}}\PY{n}{recall}\PY{l+s+si}{:}\PY{l+s+s2}{.4f}\PY{l+s+si}{\PYZcb{}}\PY{l+s+s2}{\PYZdq{}}\PY{p}{)}
\end{Verbatim}
\end{tcolorbox}

    \begin{Verbatim}[commandchars=\\\{\}]
數字 '0':
  精確率 (Precision): 0.9808
  召回率 (Recall):    0.9898
數字 '1':
  精確率 (Precision): 0.9912
  召回率 (Recall):    0.9894
數字 '2':
  精確率 (Precision): 0.9777
  召回率 (Recall):    0.9787
數字 '3':
  精確率 (Precision): 0.9783
  召回率 (Recall):    0.9802
數字 '4':
  精確率 (Precision): 0.9797
  召回率 (Recall):    0.9827
數字 '5':
  精確率 (Precision): 0.9842
  召回率 (Recall):    0.9765
數字 '6':
  精確率 (Precision): 0.9833
  召回率 (Recall):    0.9823
數字 '7':
  精確率 (Precision): 0.9815
  召回率 (Recall):    0.9786
數字 '8':
  精確率 (Precision): 0.9665
  召回率 (Recall):    0.9774
數字 '9':
  精確率 (Precision): 0.9789
  召回率 (Recall):    0.9663
    \end{Verbatim}

    \section{Step 4: Conclusion}\label{step-4-conclusion}

本次作業成功地從零開始,僅使用 NumPy 實現了一個能夠對 MNIST
手寫數字進行高精度分類的雙層神經網路。在使用 256 個隱藏層神經元,經過 50
個 epoch 的訓練後,模型在測試集上達到了 98.03\% 的優異準確率
。我們進一步透過混淆矩陣分析了模型對每個數字的詳細預測情況
,並計算了各類別的精確率 (Precision) 與召回率 (Recall)
。結果顯示,模型對於所有數字(0-9)均有穩健且高水準的辨識能力,例如數字
`1' 的精確率為 0.9912,召回率為 0.9894。並且所有類別的 Precision 和
Recall 皆維持在 0.96 以上。

透過 Confusion Matrix Heapmap
能更深入地分析模型的具體表現。圖中清晰的對角線顯示,絕大多數的樣本都被正確分類。例如,在
`1' 的測試樣本中,模型成功辨識了 1123
個。同時,從非對角線的數值中也能發現一些有趣的混淆情況,例如有 10
次將真實的 `9' 預測為 `4' ,以及 8 次將 `4' 預測為 `9'
。這些特定的錯誤,為未來優化模型提供了改進的方向。


    % Add a bibliography block to the postdoc
    
    
    
\end{document}
